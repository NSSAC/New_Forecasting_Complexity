%%%%%%%%%%%%%%%%%%%%%%%%%%%%%%%%%%%%%%%%%%%%%%%%%%%%%%%%%%%%%%%%%%%%%%%  
%%% Short-Term Forecasting in SIR Systems - Section 5:  
%%% Last edited by Ravi (Nov. 6, 2016).  
%%%%%%%%%%%%%%%%%%%%%%%%%%%%%%%%%%%%%%%%%%%%%%%%%%%%%%%%%%%%%%%%%%%%%%%%  
  
\section{Complexity Results for More Realistic Social Networks}
\label{sec:realistic}


In this section, we shall show how the results of the 
previous section can be extended to SIR systems whose
underlying graphs are closer to actual social networks. 
For brevity, we present these extensions for the 
\tNewInfv{} and \tTotInfv{} problems. 
%%As in Section \ref{sec:general_results},
%%we will consider these problems for set $S=V$, which will not be specified below.
%%\textbf{(Must verify that these extensions also hold for
%%other problems.)}


\subsection{Making All Edge Probabilities Below 1}

%% \textsf{We may not need this subsection since the next
%% subsection shows how all the probability values can be 
%% made 1/2.}

\begin{theorem}\label{thm:gen_hardness_prob_not_one}
The following forecasting problems are \cnump-hard,
even when all the transmission probabilities are less than one:
(1) \TwoNewInfv{} ~and~
(2)  \TwoTotInfv{}.
\end{theorem}

\noindent
\textbf{Proof:}~  

\noindent
\textbf{Proof of (1):}
%For the proof for the \TwoNewInfv{} problem. 
We use the same construction as in the proof for the \TwoNewInfv{} problem 
in Part~(1) of Theorem \ref{thm:gen_hardness}, 
except that each edge incident on a node in $V_2$ is 
assigned a transmission probability $p_c$ such that 
$( 1 - 1/2^n)^{1/m} < p_c <1$.  
As before, any infection pattern (i.e., a subset of $V_1$)
corresponding to a non-satisfying
assignment results in zero probability of event \cale{}.  
Now consider an infection pattern, say $\alpha$,
corresponding to a satisfying assignment.
This infection pattern  gives each
node in $V_2$ at least one neighbor that is infected at time $t =
1$.  Therefore, the conditional probability, given infection pattern
$\alpha$, that any given node in $V_2$ is infected at time $t = 2$
is at least $p_c$ and less than 1.  
The conditional probability,
given infection pattern $\alpha$, that all $m$ nodes in $V_2$ are
infected at time $t = 2$ is at least $p_c^m$ and less than 1.

Let $N$ be the number of satisfying assignments, and $P_m$ be the
total probability that there are at least $m$ nodes infected at
time $t = 2$.  
Since each infection pattern corresponding to a
satisfying assignment has probability $1/2^n$ of occurring, we
obtain $$p_c^m N/2^n \leq P_m < N/2^n.$$
Since  $p_c > ( 1 - 1/2^n)^{1/m}$, we have that 
$$( 1 - 1/2^n) N/2^n < P_m < N/2^n.$$
Since there is at least one clause, the assignment where all variables
have value {\tt false} is non-satisfying, so $N < 2^n$. Therefore,
$$( 1 - 1/2^n) N = N - N/2^n > N-1.$$
This yields 
$$(N-1)/2^n < P_m < N/2^n.$$
Consequently, 
$\displaystyle{N = \left\lceil 2^n P_m \right\rceil}$.
Thus, if the probability value $P_m$ can be computed efficiently,
then $N$, the number of satisfying assignments to the 
\mtsat{} instance can also be computed efficiently.
This completes the proof of Part~(1) of the theorem.

\medskip
\noindent
\textbf{Proof of (2):}
%We start with the proof for the \TwoTotInfv{} problem.
We use the same construction as in the proof of
Part~(2) of  Theorem \ref{thm:gen_hardness},
except that each edge incident on a node in $V_2$ is assigned a
transmission probability $p_c$ such that 
$( 1 - 1/2^n)^{1/{mn}} < p_c <1$.  
As before, any infection pattern corresponding to a
non-satisfying assignment results in zero probability of event \cale{}.  
Now consider an infection pattern corresponding to a
satisfying assignment, say infection pattern $\alpha$.  
The conditional probability, given infection pattern $\alpha$, 
that all $mn$ nodes in $V_2$ are infected at time 
$t = 2$ is at least $p_c^{mn}$ and less than 1.

Let $N$ be the number of satisfying assignments, and $P_q$ be the
total probability that there are at least $q$ nodes infected at
time $t = 2$.  
Since each infection pattern corresponding to a
satisfying assignment has probability $1/2^n$ of occurring, we obtain
$$p_c^{mn} N/2^n \leq P_q < N/2^n.$$
Since  $p_c > ( 1 - 1/2^n)^{1/{mn}}$, we have  
an equation similar to that in Part~(1):
$$( 1 - 1/2^n) N/2^n < P_q < N/2^n.$$
Thus, 
$N = \displaystyle{\left\lceil 2^n P_q \right\rceil}$,
and this completes the proof of Part (2) of the theorem.
\QED


\subsection{Making All Edge Probabilities Equal to Half}

\begin{theorem}\label{thm:gen_hardness_prob_half}
For any $t \geq 3$, the following forecasting problems are \cnump-hard,
even when the transmission probabilities are all equal to $1/2$:
(1)  \tNewInfv{} ~and~ 
(2)  \tTotInfv{}.
\end{theorem}

\noindent
\textbf{Proof:}~  

\noindent
\textbf{Proof of (1):}
We start with the proof for the \TwoNewInfv{} problem.
Recall that $n$ and $m$ denote respectively the number of variables and
clauses in the given instance of~ \mtsat.
Let
\begin{equation}\label{eqn:k_new_defn}
k = - \left\lfloor \frac{\ln (1- (1-1/2^n)^{1/m} )}{\ln(4/3)} 
        \right\rfloor + 1.
\end{equation}
The following claim establishes a bound on the value of $k$.

\medskip
\noindent
\textbf{Claim 5:} The value of
$k$ defined by Equation~(\ref{eqn:k_new_defn})
is bounded by a linear function of $n$.

\medskip
\noindent
\textbf{Proof of Claim 5:}~
Note that for a given $n$, the value of $k$ increases with $m$.
Also note that since each clause contains two variables and
no clause is repeated, $m < n^2$.

Consider the quantity $\epsilon = 1/(n^2 2^n)$.
We want to show that $(1 - \epsilon)^{n^2} > (1-1/2^n)$.
From the Binomial Theorem \cite{Rosen-2011},
$$(1 - \epsilon)^{n^2} = \sum_{i=0}^{n^2} {n^2 \choose i}{(- \epsilon)}^i,$$
where 
$${n^2 \choose i} = \frac{n^2!}{i! \, (n^2-i)!}.$$
Note that 
$${n^2 \choose i+1} ~=~ 
{n^2 \choose i} \frac{n^2-i}{i+1} ~\leq~ {n^2 \choose i} n^2$$.
Thus,
$${n^2 \choose i+1} {\epsilon}^{i+1} < {n^2 \choose i} {\epsilon}^i,$$
so the magnitude of the terms in the above summation is 
monotonically decreasing.
Consequently, the value of the summation is greater than 
the sum of the first two terms in the summation.
Thus, $(1 - \epsilon)^{n^2} > (1 - n^2 \epsilon)$, 
so  $(1 - \epsilon)^{n^2} > 1-1/2^n$.

As a consequence, $(1 - \epsilon) > (1-1/2^n)^{1/{n^2}}$.
More generally, for any $m < n^2$,
$(1 - \epsilon)$ $>$ \mbox{$(1-1/2^n)^{1/m}$}.
Thus, $(1- (1-1/2^n)^{1/m} ) > \epsilon$.
Consequently, $\ln (1- (1-1/2^n)^{1/m} ) > \ln (\epsilon)$,
so \mbox{$- \ln (1- (1-1/2^n)^{1/m})$} $<$ $-\ln (\epsilon)$.
But $\ln (\epsilon)$ = $-n \ln (2) + 2  \ln (n)$.
Thus, 
$$k <  \left\lceil \frac{n \ln (2) + 2  \ln (n)}{\ln(4/3)} \right\rceil+ 1,$$
establishing Claim~5. \hfill $\Box$

\medskip
We now continue the proof for the \ThrNewInfv{} problem.
We modify the proof for the \TwoNewInfv{} problem from Part (1) of 
Theorem~\ref{thm:gen_hardness_prob_not_one}.  
The graph $G(V,E)$ is
modified by replacing each edge in $E_2$ with a set of
$k$ edge-disjoint paths, as follows.  
We add a new set of $2km$ nodes $Y$.  
We envision these $2km$ nodes as being organized 
into $2m$ groups, with each group containing $k$ nodes.  
Each of these groups can be envisioned
as being associated with one of the $2m$ edges in $E_2$.  
Consider an edge $\{v_a, w_j\}$ in $E_2$.  
We denote the associated group
of nodes in $Y$ as $\{ y_{a, j,d}$, $1 \leq d \leq k \}$.

Each of the $2m$ edges in $E_2$ is replaced by $2k$ new edges.
Consider an edge $\{v_a, w_j\}$ in $E_2$.
This edge is deleted, and replaced by the following $2k$ new edges:
$\{v_a,  y_{a, j,d}\}$ and $\{y_{a, j,d},  w_j\}$, 
where $1 \leq d \leq k$.
Each of these new edges is assigned the transmission probability $1/2$.

We now consider the probability that event \cale{} occurs at time $t = 3$.
Any infection pattern corresponding to a non-satisfying assignment
results in zero probability of event \cale{}.
Now consider an infection pattern corresponding to a satisfying assignment, 
say infection pattern $\alpha$.
Consider a clause $C_j$.
Since the assignment satisfies all clauses,
there is at least one node $v_a$ in $V_1$,
such that $v_a$ corresponds to a variable occurring in clause $C_j$,
and for this infection pattern,
node $v_a$ is in state $\istate$ at time $t = 1$.
Each of the $k$ disjoint paths 
$( \{v_a,  y_{a, j,d}\}, \{y_{a, j,d},  w_j\} )$
has probability $1/4$ of causing node $w_j$ to be 
in state $\istate$ at time $t = 3$.
Therefore, the conditional probability, given infection pattern $\alpha$,
that any given node in $V_2$ is infected at time $t = 3$ 
is at least $1 - (3/4)^k$ and less than 1.
The conditional probability, given infection pattern $\alpha$,
that all $m$ nodes in $V_2$ are infected at time $t = 3$ 
is at least $(1 - (3/4)^k)^m$ and less than 1.

\medskip
Note that 
$$k > - \frac{\ln (1- (1-1/2^n)^{1/m} )}{\ln(4/3)}.$$ 
So 
$$k \, \ln(4/3) > -  \ln (1- (1-1/2^n)^{1/m} ).$$
Thus, $$(4/3)^k > \frac{1}{1- (1-1/2^n)^{1/m}}.$$
Equivalently,
$$(3/4)^k < 1- (1-1/2^n)^{1/m}.$$
Thus, $(1-1/2^n)^{1/m} < 1 - (3/4)^k$, yielding
$$(1-1/2^n) < (1 - (3/4)^k)^m.$$

Let $N$ be the number of satisfying assignments, and $P_m$ be the
total probability that there are at least $m$ nodes infected at
time $t = 3$.  Since each infection pattern corresponding to a
satisfying assignment has probability $1/2^n$ of occurring, we
obtain $$(1 - (3/4)^k)^m N/2^n \leq P_m < N/2^n.$$

Since  $(1 - (3/4)^k)^m > ( 1 - 1/2^n)$, we have have that
$$( 1 - 1/2^n) N/2^n < P_m < N/2^n.$$
Thus, 
$N = \displaystyle{\lceil 2^n P_m \rceil}$,
and this completes the proof of the \cnump-hardness of \ThrNewInfv{}.

The \cnump-hardness of \tNewInfv{} for any $t \geq 4$ can be 
proven through a simple modification to the above construction,
similar to the modification used in the proof of Part~(1)
of Theorem~\ref{thm:gen_hardness}.
The graph $G(V,E)$ constructed in the proof for \ThrNewInfv{} ~is
augmented as follows.
Given an integer $t \geq 4$, we construct a simple path 
$\langle s_0, s_1, \ldots, s_{t-4}\rangle$
consisting of $t-3$ new nodes, and add an edge from $s_{t-4}$ to $s$.
The transmission probability for all the new edges is set to $1/2$.
At time 0, node $s_0$ is in state \istate{} and all 
other nodes are in state \sstate. 
The new nodes and edges added to $G$ ensure that 
node $s$ gets infected at time $t-3$ with probability $(1/2)^{t-3}$.
The requirement is to compute the probability that at least $m$
nodes in $V_2$ get infected at time $t$.  
Since each satisfying assignment has at least one variable equal to 
\texttt{True},
each infection pattern 
corresponding to a satisfying assignment has probability 
$(1/2)^{n+t-3}$ of occurring.
Thus, $N = \displaystyle{\lceil 2^{n+t-3} P_m \rceil}$, and
this completes the proof of Part~(1) of the theorem.



\medskip
\noindent
\textbf{Proof of (2):}
We start with the proof for the \TwoTotInfv{} problem.
Recall that $n$ and $m$ denote respectively the number of variables and
clauses in the given instance of~ \mtsat.
Let
\begin{equation}\label{eqn:k_tot_defn}
k = - \left\lfloor \frac{\ln (1- (1-1/2^n)^{1/{m n^2}} )}{\ln(4/3)} 
        \right\rfloor + 1.
\end{equation}
The following claim establishes a bound on the value of $k$.

\medskip
\noindent
\textbf{Claim 6:} The value of
$k$ defined by Equation~(\ref{eqn:k_tot_defn})
is bounded by a linear function of $n$.

\medskip
\noindent
\textbf{Proof of Claim 6:}~
Note that for a given $n$, the value of $k$ increases with $m$.
Also note that since each clause contains two variables and no
variable is repeated, $m < n^2$.

Consider the quantity $\epsilon = 1/(n^4 2^n)$.
We want to show that $(1 - \epsilon)^{n^4} > (1-1/2^n)$.
From the Binomial Theorem \cite{Rosen-2011},
$$(1 - \epsilon)^{n^4} = \sum_{i=0}^{n^4} {n^4 \choose i}{(- \epsilon)}^i,$$
where 
$${n^4 \choose i} = \frac{n^4!}{i! \, (n^4-i)!}.$$
Note that 
$${n^4 \choose i+1} ~=~ 
{n^4 \choose i} \frac{n^4-i}{i+1} ~\leq~ {n^4 \choose i} n^4.$$
Thus,
$${n^4 \choose i+1} {\epsilon}^{i+1} < {n^4 \choose i} {\epsilon}^i$$.
So the magnitude of the terms in the above summation is 
monotonically decreasing.
Consequently, the value of the summation is greater than 
the sum of the first two terms in the summation.
Thus, $(1 - \epsilon)^{n^4} > (1 - n^4 \epsilon)$,~ 
so  $(1 - \epsilon)^{n^4} > 1-1/2^n$.

As a consequence, $(1 - \epsilon) > (1-1/2^n)^{1/{n^4}}$.
More generally, for any $m < n^2$,
$(1 - \epsilon)$ $>$ \mbox{$(1-1/2^n)^{1/{m n^2}}$}.
Thus, $(1- (1-1/2^n)^{1/{m n^2}} ) > \epsilon$.
Consequently, $\ln (1- (1-1/2^n)^{1/{m n^2}} ) > \ln (\epsilon)$,
so \mbox{$- \ln (1- (1-1/2^n)^{1/{m n^2}})$} $<$ $-\ln (\epsilon)$.
But $\ln (\epsilon)$ = $-n \ln (2) - 4  \ln (n)$.
Thus, 
$$k <  \left\lceil \frac{n \ln (2) + 4  \ln (n)}{\ln(4/3)} \right\rceil+ 1,$$
establishing Claim~6. \hfill $\Box$

We modify the proof for the \TwoTotInfv{} problem from 
Part~(2) of Theorem \ref{thm:gen_hardness_prob_not_one}. 
The graph $G(V,E)$ is modified as follows. 
As before, there is a node for each variable.
However, for each of the $m$ clauses,  there are  now $n^3$ clause nodes.
We denote the set of $m n^3$ clause nodes as $\{ w_i \mid 1 \leq i \leq m n^3 \}$.
%We refer to the set of clause nodes associated with a given clause $c_j$
%as $W_j = \{w_j^1, w_j^2, \ldots, w_j^{n^2}\}$. 
There are no direct edges between variable nodes and clause nodes.
For each of the $n$ variables, we add $k$ new nodes, 
which we refer to as {\bf auxiliary nodes} for that variable.
There is an edge from each of the $n$ variable nodes, 
to each of the $k$ auxiliary nodes for that variable.
There is also an edge from each auxiliary node for a given variable 
to each clause node for a clause containing that variable.
Each of edge is assigned the transmission probability $1/2$.
The value of $q$ in specifying event \cale{} is 
given by $q = mn^3$.

We now consider the probability that event \cale{} occurs at time 3.

Consider an infection pattern corresponding to a non-satisfying assignment.
Since there is at least one clause that is unsatisfied,
at most $(m-1) n^3$ clause nodes get infected by time 3.
The number of nodes that are not clause nodes is $1 + n +kn$.
Thus, the total number of nodes that are infected by time 3 is at most  $(m-1) n^3 + (k+1) n + 1$.
As per Claim 6, let $\beta$ be a constant such that $k \leq \beta n$.
Then the total number of nodes that are infected by time 3 is at most  $q - n^3 + \beta n^2 + n + 1$.
Thus there is a constant $n_0$ such that if the number of Boolean variables $n$ is at least $n_0$,
any infection pattern corresponding to a non-satisfying assignment 
results in zero probability of event \cale{}.

We will now utilize a general principle of probabilities 
known as the FKG Inequality \cite{FKG_1971, srinivasan_1999}.
For an event $\gamma$, let $Pr[\gamma]$ denote the probability of event $\gamma$,
and for a set of events $\Gamma$, let $Pr[\Gamma]$ denote the 
probability of $\Gamma$.
Suppose we are considering events over a given set of 
Boolean-valued independent random variables.
For a given subset $A$ of these random variables,
let $\bar{A}$ be the event that those variables in $A$ have value \texttt{True} 
and those not in $A$ have value \texttt{False}. 
An event $\gamma$ is said to be {\bf increasing} in these random variables 
if for any subsets $A$ and $B$
of these random variables such that $A \subset B$,
if $\gamma$ holds when $\bar{A}$ holds, then $\gamma$ holds when $\bar{B}$ holds.
The FKG Inequality says that for an event $\gamma$ and set of other events $\Gamma$
that are all increasing over the same set of random variables,
$Pr[\gamma \:|\: \Gamma] \geq Pr[\gamma]$.


Now consider an infection pattern corresponding to a satisfying assignment, 
say infection pattern $\alpha$.
Let $\cale_i$ denote the event that node $w_i$ is infected at time 3. 
%Let $\cale^i$ denote the event that the set of nodes $\{  w_1, \dots ,  w_i \}$
%are all infected at time 3. 
%Let $Pr[ \cale_i ]$ denote the probability of event $\cale_i$.
%Let $Pr[ \cale^i ]$ denote the probability of event $\cale^i$.
Consider a clause $C_j$.
Since assignment $\alpha$ satisfies all clauses,
there is at least one variable node $v_a$,
such that $v_a$ corresponds to a variable occurring in clause $C_j$,
and in infection pattern $\alpha$,
node $v_a$ is in state $\istate$ at time $t = 1$.
For any given clause node $w_i$ corresponding to clause $c_j$,
there are $k$ disjoint length 2 paths between nodes $v_a$ and $w_i$.
%$( \{v_a,  y_{a, j,d}\}, \{y_{a, j,d},  w_j\} )$
Each of these $k$ paths
has probability $1/4$ of causing node $w_i$ to be 
in state $\istate$ at time $t = 3$.
Therefore, the conditional probability of event $\cale_i$ 
given infection pattern $\alpha$
is at least $1 - (3/4)^k$ and less than 1.

Let us call an edge incident on an auxiliary node an {\bf auxiliary edge}.
The transmissibility of each auxiliary edge is an independent random variable, 
with probability $1/2$.
Note that each event $\cale_i$ is increasing in these random variables.
Because these events are all increasing, the FKG Inequality
applies to the probabilities of the $m n^2$ events of the form $\cale_i$.
From the FKG Inequality, for $1 < i \leq m n^2$,
$$Pr[ \cale_i \mid \alpha \, \bigwedge_{1\leq i' < i} \, \cale_{i'} ] 
   ~\geq~ Pr[ \cale_i \mid \alpha].$$

\noindent
For each $\cale_i$, $1 \leq i \leq m n^2$, 
$$Pr[ \cale_i \:|\:  \alpha] ~\geq~ 1 - (3/4)^k.$$
Thus, using induction, it follows that
$$Pr[ \bigwedge_{1 \leq i' \leq i} \, \cale_{i'} \;|\;  \alpha] 
  ~\geq~ [1 - (3/4)^k]^i.$$

\noindent
Thus, the conditional probability, given infection pattern $\alpha$,
that all $m n^2$ clause nodes are infected at time $t = 3$ 
is at least $[1 - (3/4)^k]^{m n^2}$ and less than 1.

\medskip
Note that 
$$k > - \frac{\ln (1- (1-1/2^n)^{1/{m n^2}} )}{\ln(4/3)}.$$ 
Thus, 
$$k \, \ln(4/3) > -  \ln (1- (1-1/2^n)^{1/{m n^2}} ).$$
In other words, $$(4/3)^k > \frac{1}{1- (1-1/2^n)^{1/{m n^2}}}.$$
Equivalently,
$$(3/4)^k < 1- (1-1/2^n)^{1/{m n^2}}.$$
Thus, $(1-1/2^n)^{1/{m n^2}} < 1 - (3/4)^k$, yielding
$$(1-1/2^n) < [1 - (3/4)^k]^{m n^2}.$$

Let $n$ be greater than $n_0$.
Let $N$ be the number of satisfying assignments, and $P_q$ be the
total probability that there are at least $q$ nodes infected by
time $t = 3$.  Since each infection pattern corresponding to a
satisfying assignment has probability $1/2^n$ of occurring, we
obtain $$[1 - (3/4)^k]^{m n^2} N/2^n \leq P_q < N/2^n.$$

Since  $[1 - (3/4)^k]^{m n^2} > ( 1 - 1/2^n)$, we have that
$$( 1 - 1/2^n) N/2^n < P_q < N/2^n.$$
Thus, $N = \displaystyle{\lceil 2^n P_q \rceil}$.
In other words,
if the probability of event \cale{} can be computed 
efficiently, then for any \mtsat{} instance with at least $n_0$ variables,
the number of satisfying assignments can also be computed efficiently. 
This establishes the \cnump-hardness of \ThrTotInfv.
The \cnump-hardness of \tTotInfv{} for any $t \geq 4$ can be 
proven in a manner similar to that used \\ for \tNewInfv{}.
\QED


\subsection{Graphs with Small Diameter and Large Average Clustering
Coefficient}
\label{sse:dia_cc}

Here, we show that \TwoNewInfv{} and \TwoTotInfv{} problems
remain computationally intractable even when the underlying
graph is a \textbf{clique}; that is, its  diameter 
and average clustering coefficient are both 1.

%%{\bf NEW VERSION OF THEOREM}

\begin{theorem}\label{thm:dia_cc}
The following forecasting problems are \cnump-hard,
even for networks whose diameter and average clustering
coefficient are both 1: 
(1)  \TwoNewInfv{} ~and~ 
(2)  \TwoTotInfv.
\end{theorem}

\noindent
\textbf{Proof of Part (1):}~ We obtain this result by modifying
the construction presented in the proof of Part~(1) 
of Theorem~\ref{thm:gen_hardness}.
We add additional edges to make the graph into a clique, 
thereby ensuring that  the diameter and average clustering coefficient are both 1.
Recall that $n$ and $m$ denote the number of variables and clauses
in the given instance of \mtsat,
node set $V_1$ consists of $n$ variable nodes,
and node set $V_2$ consists of $m$ clause nodes.
New edge set $E_3$ consists of $m$ edges, and connects node $s$ to each of the clause nodes.
New edge set $E_4$ consists of $m(n-2)$ edges, 
and connects each clause node to each of the $n-2$ variable nodes 
that it is not connected to via edges in $E_2$.
New edge set $E_5$ consists of $n(n-1)/2$ edges, and connects each pair of variable nodes.
New edge set $E_6$ consists of $m(m-1)/2$ edges, and connects each pair of edge nodes.
We let $r$  be the number of edges in $E_3 \cup E_4 \cup E_5$, viz. $r = n(n-1)/2 + m(n-1)$.

The transmission probability of each edge in $E_3 \cup E_4 \cup E_5$ 
is set to a value $\epsilon$ such that 
\[
\epsilon ~<~ \min\{1 - (1 - 1/{2^{n+1}})^{1/r},~ 1 - (1/2)^{1/m}\}.
\]
We refer to the edges in $E_3 \cup E_4 \cup E_5$ as {\bf weak edges}.

The transmission probability of each edge in $E_6$ is set to the value 1.

Consider an infection pattern, say $\alpha$,
corresponding to a satisfying assignment.
Consider the event ${\cal E}_{\alpha}$ that none of the 
edges in $E_3$ are transmitting,
and the edges in $E_1$ that are transmitting 
have endpoints in $V_1$ corresponding to infection pattern $\alpha$.
Note that the probability of event ${\cal E}_{\alpha}$ is $(1 -  \epsilon)^m/{2^n}$.
Event ${\cal E}_{\alpha}$ ensures that each
clause node is uninfected at time $t=1$,
and has at least one neighbor that is infected at time $t =1$.  
Therefore, event ${\cal E}_{\alpha}$ ensures
that all $m$ clause nodes become newly infected at time $t = 2$.  

Let $N$ be the number of satisfying assignments, 
so that there are $N$ events of the form ${\cal E}_{\alpha}$,
corresponding to the $N$ satisfying assignments.
Note that these $N$ events are disjoint.
Let ${\cal E}_{sat}$ denote the event that one of these $N$ events occurs.
The probability that event ${\cal E}_{sat}$ occurs,
which we denote as  $P[{\cal E}_{sat}]$, is $(1 -  \epsilon)^m N/2^n$.

Now consider the event ${\cal E}_{weak}$ that at least one of the $r$ weak edges
(i.e., the edges in $E_3 \cup E_4 \cup E_5$), is transmitting.
Note that events ${\cal E}_{sat}$ and ${\cal E}_{weak}$ are not disjoint.
The probability that event ${\cal E}_{weak}$ occurs,
which we denote as  $P[{\cal E}_{weak}]$, is $1 -(1 -  \epsilon)^r$.
There are instances when event ${\cal E}_{weak}$ occurs 
and there are $m$ new infections at time 2;
and there are instances when event ${\cal E}_{weak}$ occurs 
but there are fewer than $m$ new infections at time 2.

Suppose that neither event ${\cal E}_{sat}$ nor event ${\cal E}_{weak}$ occurs.
Then the only possible new infections that can occur at time $t=1$ are in
variable nodes, and the set of these variables that are infected correspond 
to a non-satisfying infection pattern.
Since no weak edges are transmitting, there less than $m$ 
new infections at time $t=2$.

Let $P_m$ be the
total probability that there are at least $m$ new nodes infected at
time $t = 2$, i.e., the probability of event ${\cal E}$.  
Since event ${\cal E}$ occurs for sure if event ${\cal E}_{sat}$ occurs,
and can only occur if at least one of events ${\cal E}_{sat}$ or event ${\cal E}_{weak}$ occurs,
we have
$$P[{\cal E}_{sat}]  ~<~ P_m  ~<~  P[{\cal E}_{sat}]  + P[{\cal E}_{weak}].$$
Thus,
$$(1 -  \epsilon)^m N/2^n ~<~ P_m  ~<~  (1 -  \epsilon)^m N/2^n  
                                        + 1 - (1 -  \epsilon)^r.$$
This yields
$$ N  ~<~  2^n/(1 -  \epsilon)^m  P_m    ~<~   N  + 
                              [1 - (1 -  \epsilon)^r] 2^n/(1 -  \epsilon)^m.$$

Recall that 
 $\epsilon < \min\{1 - (1 - 1/{2^{n+1}})^{1/r},~1 - (1/2)^{1/m}\}$.
 The fact that $\epsilon < 1 - (1 - 1/{2^{n+1}})^{1/r}$ ensures that 
 $(1 -  \epsilon) > (1 - 1/{2^{n+1}})^{1/r}$, so that
 $(1 -  \epsilon)^r  > 1 - 1/{2^{n+1}}$,
 and consequently $2^n (1 -  \epsilon)^r  > 2^n-1/2$.
The fact that $\epsilon < 1 - (1/2)^{1/m}$ ensures that 
 $(1 -  \epsilon) > (1/2)^{1/m}$, so that
 $(1 -  \epsilon)^m  > 1/2$.
We thus have 
 $$2^n (1 -  \epsilon)^r + (1 -  \epsilon)^m  > 2^n.$$
So,
$$[ 1 - (1 -  \epsilon)^r] 2^n < (1 -  \epsilon)^m.$$
 Consequently,
 $$[1 - (1 -  \epsilon)^r] 2^n/(1 -  \epsilon)^m < 1.$$
This yields
$$ N ~<~ 2^n/(1 -  \epsilon)^m  P_m   ~<~  N  + 1.$$

Consequently, $N = \lfloor 2^n/(1 -  \epsilon)^m  P_m \rfloor.$
Thus, if the probability value $P_m$ can be computed efficiently,
then $N$, the number of satisfying assignments to the 
\mtsat{} instance, can also be computed efficiently.
This completes the proof of Part~(1) of the theorem.


\medskip
\noindent
\textbf{Proof of Part (2):}~ We obtain this result by a modification
to the construction presented in the proof of Part~(1) 
of  this Theorem.
The necessary modifications are as follows.

The value of $q$ is $n+m+1$.

The node and edge sets are the same as in Part~(1),
but some of the edge transmission probabilities are different.
Each edge in $E_1$ has transmission probability $1/2$.
Each edge in $E_2 \cup E_5 \cup E_6$ has transmission probability 1.

We let $r$  be the number of edges in $E_3 \cup E_4$, viz., $r = m(n-1)$.

The transmission probability of each edge in $E_3 \cup E_4$ 
is set to a value $\epsilon$ such that $\epsilon < 1 - (1 - 1/{2^n})^{1/r}$.
Note that such a value of $\epsilon$ ensures that $(1 - (1 -  \epsilon)^r) 2^n < 1$.
We refer to the edges in $E_3 \cup E_4$ as {\bf weak edges}.

Consider an infection pattern, say $\alpha$,
corresponding to a satisfying assignment.
Consider the event ${\cal E}_{\alpha}$ that the edges in $E_1$ that are transmitting 
have endpoints in $V_1$ corresponding to infection pattern $\alpha$.
Note that the probability of event ${\cal E}_{\alpha}$ is $1/{2^n}$.
Since $\alpha$ is a satisfying assignment,
at least one clause node is infected at time $t = 1$.
Edge set $E_5$ then ensures that every variable node is infected by time $t=2$.
Therefore, event ${\cal E}_{\alpha}$ ensures
that all $q$ nodes are infected by time $t = 2$.  

Let $N$ be the number of satisfying assignments, 
so that there are $N$ events of the form ${\cal E}_{\alpha}$,
corresponding to the $N$ satisfying assignments.
Note that these $N$ events are disjoint.
Let ${\cal E}_{sat}$ denote the event that one of these $N$ events occurs.
The probability that event ${\cal E}_{sat}$ occurs,
which we denote as  $P[{\cal E}_{sat}]$, is $N/2^n$.

Now consider the event ${\cal E}_{weak}$ that at least one of the $r$ weak edges,
i.e., the edges in $E_3 \cup E_4$, is transmitting.
Note that events ${\cal E}_{sat}$ and ${\cal E}_{weak}$ are not disjoint.
The probability that event ${\cal E}_{weak}$ occurs,
which we denote as  $P[{\cal E}_{weak}]$, is $1 -(1 -  \epsilon)^r$.
There are instances when event ${\cal E}_{weak}$ occurs 
and there are $q$ total infections by time $t=2$;
and there are instances when event ${\cal E}_{weak}$ occurs 
but there are fewer than $q$ total infections by time $t=2$.

Suppose that neither event ${\cal E}_{sat}$ nor event ${\cal E}_{weak}$ occurs.
Then the only possible new infections that can occur at time $t=1$ are in
variable nodes, and the set of these variables that are infected correspond 
to an non-satisfying infection pattern.
Since, no weak edges are transmitting, 
there is at least one clause node that is uninfected by time $t=2$.
Thus, there are fewer than $q$ total infections by time $t=2$.

Let $P_q$ be the
total probability that there are at least $q$ new nodes infected by
time $t = 2$, i.e., the probability of event ${\cal E}$.  
Since event ${\cal E}$ occurs for sure if event ${\cal E}_{sat}$ occurs,
and can only occur if at least one of events ${\cal E}_{sat}$ 
or event ${\cal E}_{weak}$ occurs, we have
$$P[{\cal E}_{sat}]  ~<~ P_q  ~<~  P[{\cal E}_{sat}]  + P[{\cal E}_{weak}].$$
Thus,
$$N/2^n ~<~ P_q  ~<~  N/2^n  + 1 - (1 -  \epsilon)^r.$$
This yields
$$ N < 2^n  P_q   <  N  + [1 - (1 -  \epsilon)^r] 2^n.$$

Since $\epsilon$ has been assigned a value so that 
$[1 - (1 -  \epsilon)^r] 2^n < 1$, we have
$$ N < 2^n  P_q   <  N  + 1.$$

Consequently, $N = \lfloor 2^n  P_q \rfloor.$
Thus, if the probability value $P_q$ can be computed efficiently,
then $N$, the number of satisfying assignments to the 
\mtsat{} instance, can also be computed efficiently.
This completes the proof of Part~(2) of the theorem. \QED

\subsection{Results for Power Law Graphs}
\label{sse:power_law}

Some authors have formalized power-law graphs as follows
(see \cite{ferrante_etal_2008,shen_etal_2012}), 
a formulation that we call a {\em strict} power-law graph.

\begin{definition} \label{def:power-law-graph}
A graph $G$ is called a {\bf strict} $\beta$  \textbf{power-law graph}
if there exists an $\alpha$ such that the maximum node degree is $\Delta = \lfloor e^{\alpha/\beta} \rfloor$
and the number $y_i$ of nodes of degree $i$ is given by 
\[
y_i = \left\{ \begin{array}{r@{\quad \quad}l}
\lfloor e^{\alpha} / i^{\beta} \rfloor  & \mbox{if}~~  ~i > 1~ ~\mbox{or}~~ 
      \sum_{i=1}^{\Delta}  \lfloor e^{\alpha} / i^{\beta} \rfloor~~  \mbox{is~ even}   \\
\lfloor e^{\alpha} \rfloor  + 1 & \mbox{otherwise}.
\end{array} \right.
\]
\end{definition}
However, often a more relaxed requirement for a graph to be deemed
a power-law graph is used, with more flexibility on the number of
nodes of small degree.  
For instance, Newman \cite{newman_2010} says the following:
\begin{quotation}
``A common situation is that power law is obeyed in the
tail of the distribution, for large values of $k$, but not in the
small-$k$ regime.  When one says that a network has a power law
degree distribution, one normally means that only the tail of the
distribution has this form."
\end{quotation}
In particular, Newman~\cite{newman_2010} uses the following definition for power
law degree distribution. 
Let $p_k$ denote the fraction of nodes with degree $k$.
When $p_k$  follows a power law distribution with exponent $\alpha$, we have
$$p_k ~=~ c k^{-\alpha}~~ ~\mbox{for}~~~  k ~\geq~ k_{min}$$
for some constants $c$ and $k_{min}$.
We formalize this definition as follows.

\begin{definition} \label{def:degree-power-law-graph}
For real $\beta > 0$ and integer $k \geq 1$, a graph $G$ is called
a \textbf{degree}-$k$ $\beta$  \textbf{power-law graph} if there exists an
$\alpha$ such that the maximum node degree is $\Delta = \lfloor
e^{\alpha/\beta} \rfloor$ and for $i \geq k$, the number $y_i$ of nodes
of degree $i$ is:
\[
y_i = \left\{ \begin{array}{r@{\quad \quad}l}
\lfloor e^{\alpha} / i^{\beta} \rfloor  & \mbox{if}~~~  i > 1~  ~\mbox{or}~~
\sum_{i=1}^{\Delta}  \lfloor e^{\alpha} / i^{\beta} \rfloor  ~~\mbox{is~ even}   \\
\lfloor e^{\alpha} \rfloor  + 1 & \mbox{otherwise}.
\end{array} \right.
\]
\end{definition}
Note that there is no constraint on the number of nodes of degree
$i$ ~if~ $i < k$.  Also note that a degree-1 $\beta$ power-law graph
satisfies Definition \ref{def:power-law-graph}, and so is a strict
$\beta$ power-law graph.
We use the following result\footnote{We have stated
Proposition~\ref{prop:power-law-embedding} as it appears
in \cite{ferrante_etal_2008,shen_etal_2012}.
The phrase ``maximal connected component" is used in the statement
to refer to a ``connected component".}
from \cite{ferrante_etal_2008,shen_etal_2012}.

\begin{proposition}  \label{prop:power-law-embedding}
There is a polynomial time algorithm that given a $\beta >0$ and a
simple undirected graph $G$, constructs a simple undirected graph
$G'$ such that $G$ is a set of maximal connected components of $G'$ and
$G'$ is a strict $\beta$  power-law graph.
\end{proposition}   

Using Proposition \ref{prop:power-law-embedding}, for any fixed
$\beta > 0$, all the results in Section 4 hold, even if the underlying
graph of the given SIR system is required to be a  strict $\beta$
power-law graph.  For instance, we have the following stronger
version of Part~(1) of Theorem~\ref{thm:gen_hardness}.

\newcommand{\calsp}{\mbox{$\mathcal{S}'$}}  

\begin{theorem}\label{thm:hardness_power_law}
For any $\beta > 0$ and any $t \geq 2$, the \tNewInfv{} problem is
\cnump-hard, even when the underlying graph of the given SIR system
is a  strict $\beta$  power-law graph.
\end{theorem}

\noindent
\textbf{Proof:}~  
Let $\cals{}$ be the SIR system constructed by the reduction used
in the proof of Part~(1) of Theorem~\ref{thm:gen_hardness}.  
Let $G$ be the underlying graph
of $\cals{}$.  Let $G'$ be the graph constructed from $G$ by the
algorithm from Proposition \ref{prop:power-law-embedding}.  Let
$\calsp{}$ be a SIR system with underlying graph $G'$, the same
transmission probabilities as $\cals{}$ on the edges that were also
in $G$, and arbitrary nonzero transmission probabilities on the new
edges.  We extend the initial configuration by having all the added
nodes be in state \sstate.  It can be seen that the reduction is
still valid.
\QED

Note that in general, the underlying graph constructed in the proof
of Theorem~\ref{thm:hardness_power_law} contains multiple
connected components. In general this is unavoidable, since a power
law graph that is required to be a simple graph may have too many
nodes of degree one to be connected.
We now show that if the underlying graph of the given SIR system
is permitted to be a degree-2 power law graph, the reduction can
be done so that the constructed graph is connected.
We do this through three kinds of graph transformations, which we refer
to collectively as {\bf component-merger} transformations.  Each
of these transformations leaves all node degrees unchanged, while
reducing the number of connected components that contain more than
two nodes.

\medskip
\noindent
\textbf{Transformation~1~ (cycle-merger):}~
Suppose that $F_1$ and $F_2$ are two distinct connected
components of a simple graph such that $F_1$ and $F_2$ each contain
a cycle.  Let $\{u, v\}$ be an edge in a cycle in $F_1$, and $\{x, y\}$
be an edge in a cycle in $F_2$.  A {\bf cycle-merger} via edges
$\{u, v\}$ and $\{x, y\}$ deletes these two edges, and replaces them
with the two new edges $\{u, x\}$ and $\{v, y\}$.

\medskip
\noindent
\textbf{Transformation~2~ (tree-merger):}~
Suppose that $F_1$ and $F_2$ are two distinct connected
components of a simple graph such that $F_1$ and $F_2$ are each
trees (i.e., cycle-free), and each contains more than two nodes.
Let $\{u, v\}$ be an edge in $F_1$ such that $v$ is a leaf node, and
$\{x, y\}$ be an edge in $F_2$ such that $y$ is a leaf node.  A {\bf
tree-merger} via edges $\{u, v\}$ and $\{x, y\}$ deletes these two
edges, and replaces them with the two new edges $\{u, x\}$ and $\{v, y\}$.

\medskip
\noindent
\textbf{Transformation~3~ (cycle-tree-merger):}~
Suppose that $F_1$ and $F_2$ are two distinct connected
components of a simple graph such that $F_1$ contains a cycle and
$F_2$ is a tree containing more than two nodes.  Let $\{u, v\}$ be
an edge in a cycle in $F_1$.  Let $\{x, y\}$ and $\{w, z\}$ be distinct
edges in $F_2$ such that $y$ and $z$ are leaf nodes.  A {\bf
cycle-tree-merger} via edges $\{u, v\}$, $\{x, y\}$ and $\{w, z\}$, deletes
these three edges, and replaces them with the three new edges 
$\{u, x\}$, $\{v, w\}$, and $\{y, z\}$.

\begin{lemma}  \label{lem:connecting_components}
There is a polynomial time algorithm that given a $\beta >0$ and a
simple undirected graph $G$, constructs a simple undirected graph
$G'$ such that $G$ is a set of connected components of $G'$, there
is at most one added connected component with more than two nodes,
and $G'$ is a degree-2 $\beta$  power-law graph.
\end{lemma}   

\noindent
\textbf{Proof:}~  
Let $H$ be the graph constructed from $G$ by the algorithm of
Proposition \ref{prop:power-law-embedding}.  If $H$ contains more
than one added connected component with more than two nodes, we can
repeatedly modify $H$ via a sequence of component-merger transformations
to reduce the number of these added connected components until there
is at most one such added connected component.  We let $G'$ be the
resulting graph.  \QED

Using Lemma \ref{lem:connecting_components}, 
for any fixed $\beta > 0$, all the results 
in Section~\ref{sec:general_results} hold, even if the underlying
graph of the given SIR system is required to be connected and a
degree-2 $\beta$  power-law graph.  For instance, we have the
following stronger version of Part~(1) of Theorem~\ref{thm:gen_hardness}.


\begin{theorem}\label{thm:hardness_degree_power_law}
For any $\beta > 0$ and any $t \geq 2$, the \tNewInfv{} problem is
\cnump-hard, even when the underlying graph of the given SIR system
is a connected degree-2 $\beta$  power-law graph.
\end{theorem}

\noindent
\textbf{Proof:}~  
Let $\cals{}$ be the SIR system constructed by the reduction used
in the proof of Part~(1) of Theorem~\ref{thm:gen_hardness}.  
Let $G_1$ be the underlying graph
of $\cals{}$.  Let $w_1$ and $w_2$ be two distinct nodes in node
set $V_2$, corresponding to clauses in the given \#M2SAT problem
instance.  Let $G_2$ be the graph obtained by adding the edge $\{w_1,
w_2\}$.  Note that $G_2$ consists of a single connected component,
and that the edge $\{w_1, w_2\}$ is part of a cycle.  Let $G_3$ be
the graph constructed from $G_2$ by the algorithm from Proposition
\ref{prop:power-law-embedding}.  Note that $G_3$ contains at most
one added connected component with more than two nodes.  If $G_3$
contains no added connected components with more than two nodes,
let $G_4$ be $G_3$.  Otherwise, let $G_4$ be obtained by 
performing either a cycle-merger or a cycle-tree-merger between the
nodes from $G_2$ and the added connected component, using edge
$\{w_1, w_2\}$ in the merger.

Let $\calsp{}$ be a SIR system with underlying graph $G_4$, the
same transmission probabilities as $\cals{}$ on the edges that were
also in $G_1$, and arbitrary nonzero transmission probabilities on
the new edges.  We extend the initial configuration by having all
the added nodes be in state \sstate.  It can be seen that the
reduction is still valid.  \QED
