\section{Extensions to Other Measures and Models}
\label{sec:extensions}

\subsection{Overview}
\label{sse:extensions_overview}

In the previous sections, we presented our results for 
forecasting certain
measures (e.g. number of new infections, total number of infections,
peak infection time) under the SIR model.
In this section, we point out that simple extensions of our constructions 
suffice to prove the computational intractability of short-term forecasting 
problems for many other epidemiologically
relevant measures defined in \cite{TC+2016} as well as for three
other epidemic models. 

\subsection{Extensions to Other Measures}
\label{sse:ext_measures}

Tabataba et al. \cite{TC+2016} present the definitions
of a number of epidemiologically relevant measures.
In Section~\ref{sec:general_results} we considered 
the forecasting problem for 
one of these measures, namely \textbf{peak time} (i.e., time which
the number of new infections reaches a maximum value).
Below, we reproduce the definitions other measures 
considered in \cite{TC+2016} and 
formulate the corresponding forecasting problems under the SIR model.
Next, we show that, in general, each of these problems is
computationally intractable.

\medskip
\noindent
\textbf{(a) Peak Value:}~ The \textbf{peak value} of an epidemic
is the largest number of new infections that occur at any time. 
Under the SIR model, the \textbf{Peak Value Forecasting} problem, denoted by
PVF, is the following: given an SIR system and integer $q$, compute
the probability that the peak value is at least $q$.

\medskip
\noindent
\textbf{(b) Take-Off Value and Time:}~ These measures 
capture the time at which there is a sharp rise in the 
number of new infections. 
To define this measure formally,
let $\sigma_t$ denote the number of new infections at time $t$.
The expression $\sigma_{t+1} - \sigma_t$ (which gives the approximate
slope of the epidemic curve) represents the \textbf{take-off-value} at time $t$.
A large positive take-off-value indicates a sharp increase 
in the number of infections.
The \textbf{take-off-time} refers to the first time at which
take-off-value exceeds a given threshold.
Under the SIR model, the \textbf{Take-Off Value Forecasting} problem, denoted by
TOVF, is the following: given an SIR system and integer $q$, compute
the probability that at some time during the course
of the epidemic, the take-off value is at least $q$.
Likewise, the \textbf{Take-Off Time Forecasting} problem, denoted by
TOTF, is the following: given an SIR system and integers $\tau$ and $q$, compute
the probability that the take-off value of at time $\tau$ is at least $q$.

\medskip
\noindent
\textbf{(c) Intensity Duration:}~ This quantity is defined as the maximum 
number of successive
time units over which the number of new infections is at or above a given threshold. 
Under the SIR model, the \textbf{Intensity Duration Forecasting} problem, denoted by
IDF, is the following: given an SIR system and integers $\tau$ and $q$, compute
the probability that the number of new infections remains at least $q$ for at least
$\tau$ successive time units.  

\medskip
\noindent
\textbf{(d) Speed of the Epidemic:}~ Let $h$ denote the first
time at which the number of new infections reaches a peak value and let  $\sigma_0$
and $\sigma_h$ denote the numbers of new infections at times 0 and $h$ respectively.
The \textbf{speed of the epidemic} is defined as the ratio $(\sigma_h - \sigma_0)/h$.
This quantity gives the slope of the line that joins the points $(0, \sigma_0)$ and
$(h, \sigma_h)$ on the curve that represents the number of new infections over time.
Under the SIR model, the \textbf{Epidemic Speed Forecasting} problem, 
denoted by ESF, is the following:
given an SIR system and a rational number $s$, compute the probability that the speed
of the epidemic is at least $s$.

\medskip
\noindent
\textbf{(e) Total Attack Rate:}~ The \textbf{total attack rate by time} $\tau$
is given by the ratio $n_{\tau}/n$, 
where $n$ and $n_{tau}$denote respectively the number of nodes 
in the SIR system and the total number of nodes infected by time $\tau$. 
Under the SIR model, the \textbf{Total Attack Rate Forecasting} problem, 
denoted by TARF, is the following:
given an SIR system, a rational number $\alpha$ and an integer $\tau$, 
compute the probability that the total attack rate of the epidemic 
by time $\tau$ is at least $\alpha$.

Having defined the forecasting problems for many measures considered in \cite{TC+2016},
we now show that, in general, each of these forecasting problems is 
computationally intractable.

\begin{theorem}\label{thm:hardness_new_measures}
The following forecasting problems are \cnump-hard:
(1) Peak Value Forecasting (PVF),
(2) Take-Off Value Forecasting (TOVF),
(3) Take-Off Time Forecasting (TOTF),
(4) Intensity Duration Forecasting (IDF),
(5) Epidemic Speed Forecasting (ESF) and
(6) Total Attack Rate Forecasting (TARF).
\end{theorem}

\medskip
\noindent
\textbf{Proof:}~ For each of these forecasting problems, our \cnump-hardness proof
is obtained by straightforward modifications to the constructions presented in
the proof of Theorem~\ref{thm:gen_hardness}.

\medskip
\noindent
\textbf{Part (1) -- Proof for PVF:}~ We modify 
the construction presented in the proof of hardness for the \tNewInfs{}
problem (Part~(1) of the proof of Theorem~\ref{thm:gen_hardness}) as follows.  
While node sets $V_0$ and $V_1$ remain the same as before, $V_2$ consists
of $mn$ nodes, with $n$ nodes corresponding to each clause of the \mtsat{}
instance.
This construction ensures that the peak number of new infections 
is at most $mn$ and that this number of new infections can occur
only at time $t = 2$. 
It can be verified that the probability that the peak value is $\geq mn$ is
equal to $N/2^n$, where $N$ is the number of satisfying assignments to the
given \mtsat{} instance.


\medskip
\noindent
\textbf{Part (2) -- Proof for TOVF:}~ We modify 
the construction presented in the proof of hardness for the \tNewInfs{}
problem (Part~(1) of the proof of Theorem~\ref{thm:gen_hardness}) as follows.  
While node sets $V_0$ and $V_1$ remain the same as before, $V_2$ consists
of $(n+1)m$ nodes, with $n+1$ nodes corresponding to each clause of the \mtsat{}
instance.
The take-off value $q$ is set to $(n+1)m-n$.
As in Part (1), it can be verified that the number of new infections
at any time is at most $(n+1)m$.
For any infection pattern that represents a satisfying assignment to the
given \mtsat{} instance, all $(n+1)m$ nodes in $V_2$ get infected 
at $t = 2$. 
Since the maximum number of nodes that can get infected at time $t = 1$ is $n$,
the take-off value is at least $(n+1)m-n$.
On the other hand,
for any infection pattern that represents an assignment that does not satisfy the
given \mtsat{} instance, at most $(n+1)(m-1)$ nodes in $V_2$ get infected 
at $t = 2$. 
Since at least one node in $V_1$ must get infected at time $t = 1$, 
the take-off value  is at most $(n+1)(m-1)-1$ = $(n+1)m - n -1$.
Thus, the probability that the take-off value is $\geq (n+1)m-n$ is
equal to $N/2^n$, where $N$ is the number of satisfying assignments to the
given \mtsat{} instance.

\medskip
\noindent
\textbf{Part (3) -- Proof for TOTF:}~ The proof is the same as the
one presented for TOVF above. 
The probability that take-off value at time $\tau = 1$ is at least $(n+1)m-n$ is
equal to $N/2^n$, where $N$ is the number of satisfying assignments to the
given \mtsat{} instance.
By adding a simple path of appropriate length to the node $s$, it can be
seen that the result holds for any $\tau \geq 1$. 

\medskip
\noindent
\textbf{Part (4) -- Proof for IDF:}~ We modify 
the construction presented in the proof for PVF (Part (1) above)
as follows.
We add another set $V_3$ of $mn$ nodes.
Node set $V_3$ is in one-to-one correspondence with $V_2$, and there is
an edge with transmission probability 1 between a node in $V_2$ and its
corresponding node in $V_3$.
Thus, the number of infections at $t = 3$ is at least as large as that at $t = 2$.
Therefore, the probability that the number of infections is at least $mn$ for
two successive time units is 
equal to $N/2^n$, where $N$ is the number of satisfying assignments to the
given \mtsat{} instance.
We note that this construction can be readily extended to increase the intensity
duration to any value $\tau \geq 2$.

\medskip
\noindent
\textbf{Part (5) -- Proof for ESF:}~ The construction is the same
as that for PVF (Part (1) above).
Recall that in that construction,
the peak number of new infections 
is at most $mn$ and that this number of new infections can occur
only at time $t = 2$. 
Thus, the probability that the speed of the epidemic is at least
$(mn-1)/2$ is equal to $N/2^n$, where $N$ is the number of 
satisfying assignments to the given \mtsat{} instance.


\medskip
\noindent
\textbf{Part (6) -- Proof for TARF:}~ The construction is the same
as that presented in the proof for TOVF (Part (2) above). 
In that construction, the total number of nodes = $1 + n + m(n+1)$ = $(m+1)(n+1)$. 
For any infection pattern that represents a satisfying assignment to the
given \mtsat{} instance, all $(n+1)m$ nodes in $V_2$ get infected 
at $t = 2$. 
Since there is one node (namely, $s$) that is infected at time 0,
at least $(n+1)m+1$ nodes get infected by time 2 if the infection
pattern corresponds to a satisfying assignment of the given \mtsat{}
instance; that is, the attack rate at time $t = 2$ is at least $[(n+1)m+1]/[(m+1)(n+1)]$.
On the other hand,
for any infection pattern that represents a non-satisfying assignment to the
given \mtsat{} instance, at most $(n+1)(m-1)$ nodes in $V_2$ get infected 
at $t = 2$. 
Thus, for this case, the total number of nodes that can get infected by
time 2 is at most $(n+1)(m-1) + n+1$ = $(n+1)m$; thus, the attack rate
at $t = 2$ is $[(n+1)m]/[(m+1)(n+1)]$.
Thus, the probability that the attack rate at $t = 2$ is $\geq [(n+1)m+2]/[(m+1)(n+1)]$ is
equal to $N/2^n$, where $N$ is the number of satisfying assignments to the
given \mtsat{} instance. 
As before, this result can also be extended to any value of $t \geq 2$. \QED

Our results for the forecasting problems associated with the 
various measures considered in this section are summarized in
Table~\ref{tab:new_measures_results}.

\begin{table}
\begin{center}
\begin{tabular}{|c|l|}\hline
\multicolumn{1}{|c|}{\textbf{Problem}} &
\multicolumn{1}{|c|}{\textbf{Result}} \\ \hline\hline
Peak Value Forecasting (PVF) & \cnump-hard \\ \hline
Take-Off Value Forecasting (TOVF) & \cnump-hard \\ \hline
Take-Off Time Forecasting (TOTF) & \cnump-hard for any $t \geq 1$ \\ \hline
Intensity Duration Forecasting (IDF) & \cnump-hard for any duration $\tau \geq 2$ \\ \hline
Epidemic Speed Forecasting (ESF) & \cnump-hard \\ \hline
Total Attack Rate Forecasting (TARF) & \cnump-hard for any $t \geq 2$ \\ \hline\hline
\end{tabular}
\end{center}
\caption{Results for Forecasting Problems for Various Epidemiological 
Measures from \cite{TC+2016}}
\label{tab:new_measures_results}
\end{table}

\subsection{Extensions to Other Epidemic Models}
\label{sse:ext_models}

We present extensions of our results to 
three other epidemic models, namely
SI, SIS and probabilistic threshold.
For the sake of brevity, we present our proofs
mainly for the \tNewInfs{} problem under these models.
Results for other forecasting problems defined 
in Section~\ref{sse:prob_formulation} can be established 
in a similar manner.

\subsubsection{Extensions to the SI Model}
\label{sss:si_extensions}

The SI model is a variant of the SIR model.
Recall that in the latter model, a node remains in state \istate{}
for a certain amount of time and then changes to the \rstate{} state.
In contrast, under the SI model \cite{Easley_Kleinberg-2010},
once a node changes to the state \istate, it remains in
that state forever; such a node can continue to infect
other nodes in future time steps.
The definition of the \tNewInfs{} problem under the SI model 
is the same as that under the SIR model: given an SI system 
(specified by an undirected graph, the transmission probability for
each edge and the initial configuration), a subset $S$ of nodes
and integers $t$ and $q$, compute the probability that there will 
be at least $q$ new infections among the nodes in $S$ at time $t$.
The following result shows that this problem remains computationally
intractable for any $t \geq 2$.

\begin{proposition}\label{pro:si_model_two_new_inf}
The \tNewInfs{} problem under the SI model is \cnump-hard 
even for any $t \geq 2$.
\end{proposition}

\medskip
\noindent
\textbf{Proof:}~ We first consider the case where $t = 2$.
The reduction from \mtsat{} to this problem is virtually identical to that
presented in the proof of Part~(1) of Theorem~\ref{thm:gen_hardness}.
The only difference is that we let $S = V_2$ and
$q = |V_2| = m$. 
The construction ensures that the nodes in $S$ cannot be in state \istate{} 
before $t = 2$.
The argument to show that each infection pattern which causes all
the nodes in $S$ to get infected at $t = 2$ corresponds to a
satisfying assignment of the \mtsat{} instance is the same
as that presented in the proof of Part~(1) of 
Theorem~\ref{thm:gen_hardness}.

The extension of the above proof for any $t \geq 3$ is identical to that 
presented in the proof of Part~(1) of Theorem~\ref{thm:gen_hardness}.
Given an integer $t \geq 3$, we construct a simple path
$\langle s_0, s_1, \ldots, s_{t-3}\rangle$
consisting of $t-2$ new nodes and add an edge from $s_{t-3}$ to $s$.
The transmission probability for all the new edges is set to 1.
At time 0, node $s_0$ is in state \istate{} and all
other nodes are in state \sstate.
The new nodes and edges ensure that
node $s$ gets infected at time $t-2$.
The requirement is to compute the probability that at least $m$
nodes in $V_2$ get infected at time $t$.
It can be verified that the probability of this event is the
same as the probability that an assignment chosen uniformly
randomly satisfies all the clauses of the \mtsat{} instance.
\QED

It can also be seen that when $t = 1$, the \tNewInfs{} problem
under the SI model can be solved efficiently using the dynamic
programming algorithm given in the proof of 
Theorem~\ref{th:one-step-forecasting} since the corresponding
configuration constraint is 1-symmetric 
(as explained in Part~(a) of Example~\ref{ex:r_symm}).

Using proofs similar to that for Proposition~\ref{pro:si_model_two_new_inf},
it can be seen that problems \tTotInfs{} and \tPeak{} also remain \cnump-hard
under the SI model for $t \geq 2$.
Likewise, the proof in \cite{SD-2012} can also be seen to imply
that \tVuls{} problem remains \cnump-hard when $t$ is part of the problem instance.

Results similar to those in Section \ref{sec:vul} for the \tTotVuls{}
problem can also be shown for the SI model. 
In this case, the
solution to the \tTotVuls{} problem for a node $v$ is defined as the
probability that $v$ gets infected by time $t$.  
We consider a ``time-expanded'' graph $G_{exp}$ defined in the following manner.
$G_{exp}$ is a directed graph with node sets 
$V^i=V$, for $i=0,\ldots, t$. 
We will refer to the ``copy'' of node $v$ in $V^i$ by $v^i$.
For any $i$, there are no edges within the set $V^i$.  
For each edge $\{u, v\}\in E$, we have the directed edges 
$(u^i, v^{i+1})$ and $(v^i, u^{i+1})$, for $0 \leq i <t$,
with the transmission probability 
$p(u^i, v^{i+1}) = p(v^i, u^{i+1}) = p(u,v)$.  
In addition, we have edges $(u^i, u^{i+1})$ for all
$u\in V$, $1 \leq i < t$, with $p(u^i, u^{i+1})=1$.  
We now consider the SIR
model on the instance $G_{exp}$ with the probabilities defined as
above, and the node $s^0$ being the source. 
This model is equivalent to the SI model on the graph $G$. 
This can be shown inductively by
using the fact that if node $v^i$ gets infected in $G_{exp}$, then
all nodes $v^j, j>i$ will get infected.  
Therefore, the probability
that a node $v^t$ is infected in the SIR model in $G_{exp}$ equals
the probability that $v$ is infected at time $t$ under the SI model
on $G$.

We now apply the algorithm from Section \ref{sec:vul} to the SIR
model on graph $G_{exp}$, except that we consider directed paths
from $s^0$ to node $v^t$, for $v\in S$. 
The algorithm and
analysis presented in the proof of 
Theorem \ref{thm:totvul_approx} extend to the directed
setting as well, giving us the following result.

\begin{proposition}\label{pro:si_model_ttotvul}
For any $\epsilon, \delta\in(0,1)$, an
$(\epsilon, \delta)$-approximation to the \tTotVuls{} problem 
for set $S$ under the SI model on $G$
can be computed in time 
$O(\frac{1}{\epsilon^2}\,m\,\Delta^{2t|S|}\log{(1/\delta)})$,
where $\Delta$ is the maximum node degree in $G$. \QED 
\end{proposition}

\subsubsection{Extensions to the SIS Model}
\label{sss:sis_extensions}

The SIS model \cite{Easley_Kleinberg-2010} is also a variant 
of the SIR model.
The only difference is that under the SIS model,
the state of a node changes from \istate{} to \sstate{}
after a certain time period.
For consistency with our results for the SIR model,
we will assume that each node in state \istate{} changes
to \sstate{} after one time unit.
A major difference between the SIS model and the SIR model
is that a node can get reinfected many times during the 
diffusion process. 
As a consequence, there are two possible ways to formulate the \tNewInfs{}
problem under the SIS model.
In the first formulation, the focus is on nodes which are state \sstate{} 
at $t-1$ and change to \istate{} at time $t$. 
(Such nodes may have been in state \istate{} during the period
0 to $t-2$.)
In the second formulation, the focus is on the nodes which
get infected for the \emph{first} time at $t$.
(Thus, such nodes are in state \sstate{} during the period 0 to $t-1$
and their state changes to \istate{} at $t$.)
We will present our result under the first formulation; it
can be seen that the result also holds under the second formulation.

Thus, our formulation of the \tNewInfs{} problem under the SIS
model is as follows:
given an SIS system (specified by an undirected graph, 
the transmission probability for
each edge and the initial configuration), a subset $S$ of nodes
and integers $t$ and $q$, compute the probability that 
at least $q$ nodes in the set $S$ are in state \istate{} 
at time $t$.
The following result shows that this problem remains computationally
intractable for any $t \geq 2$.

\begin{proposition}\label{pro:sis_model_two_new_inf}
The \tNewInfs{} problem under the SIS model is \cnump-hard 
for any $t \geq 2$.
\end{proposition}

\smallskip
\noindent
\textbf{Proof:}~ We will present the details for $t = 2$; 
the extension of the proof to any $t \geq 3$ is along the same lines
as mentioned in the proof of Proposition~\ref{pro:si_model_two_new_inf}.

For $t = 2$, the reduction from \mtsat{} 
presented in the proof of Part~(1) of Theorem~\ref{thm:gen_hardness}
can again be used along with the assignments 
$S = V_2$ and $q = |V_2| = m$. 
The construction ensures that the nodes in $S$ cannot get 
infected for the first time before $t = 2$. 
Thus, the previous argument
can be used to conclude that 
each infection pattern which causes all
the nodes in $S$ to get infected time at
at $t = 2$ corresponds to a
satisfying assignment of the \mtsat{} instance. 
\QED

As in the SI model, when $t = 1$, the \tNewInfs{} problem
under the SIS model can also be solved efficiently using the dynamic
programming algorithm given in the proof of 
Theorem~\ref{th:one-step-forecasting} since the corresponding
configuration constraint is 1-symmetric and no reinfections
are possible for $t = 1$.

Using proofs similar to that for Proposition~\ref{pro:sis_model_two_new_inf},
it can be seen that appropriate versions of problems \tTotInfs{} and \tPeak{} 
also remain \cnump-hard under the SI model for $t \geq 2$.
Likewise, the proof in \cite{SD-2012} can also be seen to imply
that \tVuls{} problem remains \cnump-hard when $t$ is part of the problem instance.

A result similar to Proposition~\ref{pro:si_model_ttotvul} for the
\tTotVuls{} problem under the SI model (Section \ref{sss:si_extensions}) 
can also be obtained for the SIS model.
Here, we consider the solution to the \tTotVuls{} problem for
a node $v$ to be the probability that $v$ gets infected 
\emph{at least once} by time $t$.
We use a time expanded graph $G_{exp}$ that is the same as in
Section \ref{sss:si_extensions}, except for the following:
we do not have the edges $(u^i, u^{i+1})$ for $u\in V, i <  t$. 
It can be shown by induction that 
the SIR system on the resulting graph $G_{exp}$, 
is equivalent to the SIS model on the graph $G$.
This gives us the same result as 
Proposition~\ref{pro:si_model_ttotvul} for the SIS model.


\subsubsection{Extensions to the Probabilistic Threshold Model}
\label{sss:pthresh_extensions}

In the probabilistic threshold (PT) model, each node has two possible states,
namely \sstate{} and \istate.
This model was considered in \cite{BH+2011} using state values 0 and
1 for \sstate{} and \istate{} respectively.
As in the SI model, once a node reaches the state \istate, it remains in that
state forever and may infect other nodes.
Unlike the SIR, SI and SIS models, the probability values in the PT
model are \emph{not} associated with the edges.
Instead, each node $v$ of the graph has a \textbf{threshold} $\tau_v$ 
and a probability $p_v$ with the following interpretation.
At time $t$, let node $v$ be in state \sstate.
\begin{enumerate}
\item If the number of neighbors of $v$ in state \istate{} is \emph{less than}
$\tau_v$, then the state of $v$ at $t+1$ is also \sstate.

\item If at least $\tau_v$ of $v$'s neighbors are in state \istate,
then the state of $v$ at time $t+1$ is \istate{} with probability $p_v$ and
\sstate{} with probability $1 - p_v$.
\end{enumerate}

The definition of the \tNewInfs{} problem under the PT model
is the same as that under the SIR model: given a PT system
(specified by an undirected graph, the threshold and probability values
for each node and the initial configuration), a subset $S$ of nodes
and integers $t$ and $q$, compute the probability that there will
be at least $q$ new infections among the nodes in $S$ at time $t$.
The following result shows that this problem remains computationally
intractable for any $t \geq 2$.

\begin{proposition}\label{pro:pt_model_two_new_inf}
The \tNewInfs{} problem under the PT model is \cnump-hard 
even for any $t \geq 2$.
\end{proposition}

\smallskip
\noindent
\textbf{Proof:}~ For $t = 2$, the proof is by a minor modification 
to the reduction from \mtsat{} 
presented in the proof of Part~(1) of Theorem~\ref{thm:gen_hardness}.
The graph $G(V,E)$ constructed from the \mtsat{} instance
is the same as before.
Further, we let $S = V_2$ and $q = |V_2| = m$. 
The probability and threshold values for the nodes are constructed as follows.
Since the initial state of node $s$ is \istate, its threshold and
probability value are chosen arbitrarily.
For each node $v \in V_1$, the threshold value is 1 and the
probability is $1/2$. 
For each node $v \in V_2$, the threshold and probability values
are both 1. 
Once again, 
the construction ensures that the nodes in $S$ cannot get 
infected before $t = 2$. 
Further, since the probability value for each node $v \in V_1$ is $1/2$,
each infection pattern has a probability of $1/2^n$.
With this observation, it can be seen that 
each infection pattern which causes all
the nodes in $S$ to get infected at $t = 2$ corresponds to a
satisfying assignment of the \mtsat{} instance. 

To extend the proof for any $t \geq 3$, 
we construct a simple path
$\langle s_0, s_1, \ldots, s_{t-3}\rangle$
consisting of $t-2$ new nodes and add an edge from $s_{t-3}$ to $s$.
At time 0, node $s_0$ is in state \istate{} and all
other nodes are in state \sstate.
The threshold and probability values for $s_0$ are chosen arbitrarily;
for all the other nodes in the path and for node $s$ itself,
the threshold and probability values are both chosen as 1.
The new nodes and edges ensure that
node $s$ gets infected at time $t-2$.
The requirement is to compute the probability that at least $m$
nodes in $V_2$ get infected at time $t$.
It can be seen that this probability is equal to $N/2^n$, where
$N$ is the number of satisfying assignments to
the \mtsat{} instance.  \QED

We now show that the above result is tight by giving a
dynamic programming algorithm for 
the \tNewInfs{} problem for $t = 1$ under the PT model.
This algorithm cannot be derived from the general dynamic programming
framework presented in Section~\ref{sss:constrained_config} since
the PT model does not use edge probabilities.

\begin{proposition}\label{pro:pt_t_1_dyn_prog}
The \OneNewInfs{} problem under the PT model can be solved in 
polynomial time.
\end{proposition}

\smallskip
\noindent
\textbf{Proof:}~ Let $S$ denote the subset of nodes and let $q$ be
the integer specified in the \OneNewInfs{} problem instance.  
Our goal is to compute the probability that at $t = 1$, at least $q$
nodes in $S$ are infected.
Let $S = \{v_1, v_2, \ldots, v_k\}$ so that $|S| = k$.

\medskip
For each node $v \in S$, let $\pi(v)$ denote the probability that
$v$ is infected at $t = 1$.
It is easy to compute $\pi(v)$:
if the number of neighbors of $v$
in state \istate{} at $t = 0$ is at least $\tau_v$, then $\pi(v)$ =
$p_v$;~ otherwise, $\pi(v) = 0$. 

\medskip
The dynamic programming table $P$ is a two-dimensional array with $k$ rows
and $k+1$ columns.
For $1 \leq i \leq k$ and $0 \leq j \leq k$,
entry $P[i,j]$ of this array will store the probability that \emph{exactly}
$j$ nodes of the set $\{v_1, \ldots, v_i\}$ are infected at $t = 1$. 

\medskip
\noindent
To begin with, all the entries of row 1 of $P$ can be computed as follows. 

\medskip
\noindent
\hspace*{0.25in}
\mbox{
\begin{tabular}{lcl}
$P[1, 0]$ &=& Probability that $v_1$ is not infected at $t = 1$ \\
          &=& $1 - \pi(v_1)$.  \\ [1ex]
$P[1, 1]$ &=& Probability that $v_1$ is infected at $t = 1$ \\
          &=& $\pi(v_1)$. \\ [1ex]
$P[1, j]$ &=& 0,~ $2 \leq j \leq k$.  
\end{tabular}
}

\medskip
\noindent
Now, assume that for some $i \geq 1$, all the entries of row $i$ of $P$ have been computed.
The entries of row $i+1$ of $P$ can be computed as follows.

\medskip
\noindent
For $j = 0$, we have

\medskip
\noindent
\hspace*{0.25in}
\mbox{
\begin{tabular}{lcl}
$P[i+1, 0]$ &=& $P[i, 0] \times $ Pr\{$v_{i+1}$ is not infected at $t =1$\} \\
            &=& $P[i,0] \times (1-\pi(v_{i+1})).$ \\ 
\end{tabular}
}

\medskip
\noindent
For $1 \leq j \leq i+1$, we have

\medskip
\noindent
\hspace*{0.25in}
\mbox{
\begin{tabular}{lcl}
$P[i+1, j]$ &=& $P[i, j] \times $ Pr\{$v_{i+1}$ is not infected at $t =1$\} $+$  \\
            & & $P[i, j-1] \times $ Pr\{$v_{i+1}$ is infected at $t =1$\}  \\ 
            &=& $P[i,j] \times (1-\pi(v_{i+1})) + P[i,j-1] \times \pi(v_{i+1}).$
\end{tabular}
}

\medskip
\noindent
Finally, note that $P[i+1, j] = 0$~ for~ $i+2 \leq j \leq k$.

\medskip
Once all the entries of array $P$ are available, the answer to the \OneNewInfs{}
problem is given by $\sum_{j=q}^{k} P[k,j]$.
It can be verified that the algorithm runs in polynomial time. \QED 

Using proofs similar to that for Proposition~\ref{pro:pt_model_two_new_inf},
it can be seen that appropriate versions of problems \tTotInfs{} and \tPeak{} 
also remain \cnump-hard under the SI model for $t \geq 2$.
However, for the \tVuls{} problem, a new hardness result can be established
under the PT model. 
In particular, as we show below, under the PT model,
the \tVuls{} problem is \cnump-hard for any \emph{fixed} value of $t \geq 3$. 
In contrast, under the SIR, SI and SIS models, although the \tVuls{} problem is
computationally intractable when $t$ is part 
of the problem instance \cite{SD-2012}, the complexity of the 
problem for \emph{fixed} values of $t \geq 3$ is open.

\begin{proposition}\label{pro:tvul_pt_model}
The \tVuls{} problem under the PT model is \cnump-hard 
for each fixed value of $t \geq 3$.
\end{proposition}

\noindent
\textbf{Proof:}~ We will first show that the \tVuls{} problem is
\cnump-hard for $t = 3$ and then point out how the proof can be
extended for any fixed value of $t \geq 4$.

The proof is by a minor modification 
to the reduction from \mtsat{} to \tNewInfs{} problem presented
in the proof of Proposition~\ref{pro:pt_model_two_new_inf} above.
We use the same graph $G(V,E)$ constructed in that proof 
along with another node $w$ which is adjacent to all the $m$ nodes in $V_2$.
The threshold for $w$ is $m$ and its probability value is 1.
The threshold and probability values for all the other nodes are the same as the
ones specified in the proof of Proposition~\ref{pro:pt_model_two_new_inf}.
The goal is to compute the vulnerability of node $w$ at $t = 3$
(i.e., the probability that $w$ gets infected at $t = 3$).

It can be seen from the construction that $w$ gets infected at $t = 3$
iff all the nodes in $V_2$ get infected at $t = 2$.
Since computing the probability of the latter event is \cnump-hard
(Proposition~\ref{pro:pt_model_two_new_inf}), it follows
that computing the vulnerability of $w$ at $t = 3$ is also
\cnump-hard.

To extend the proof to any fixed value of $t \geq 4$, say $r$, 
we add a simple path $\langle w_1, w_2, \ldots, w_{r-3}\rangle$
consisting of $r-3$ nodes to $w$.
For each node $w_j$ in the path, the threshold and probability values
are both chosen to be 1.
It can be seen that the vulnerability of node $w_{r-3}$ at $t = r$ is
equal to the vulnerability of node $w$ at $t = 3$.
\QED

We now observe that the result of Proposition~\ref{pro:tvul_pt_model}
is tight by pointing out that the \OneVuls{} and \TwoVuls{} problems
can be solved efficiently under the PT model.

\begin{proposition}\label{pro:pt_one_two_vul_poly}
The \OneVuls{} and \TwoVuls{} problems can be solved in polynomial
time for the PT model.
\end{proposition}

\noindent
\textbf{Proof:}~ The result for the \OneVuls{} problem is obvious 
since for any node $v$,
we need to find only the number of neighbors of $v$ 
in state \istate{} at $t = 0$.
If this number is at least $\tau_v$, then the vulnerability $\pi(v)$ of $v$
at $t = 1$ is $p_v$;~ otherwise, $\pi(v) = 0$. 

To solve the \TwoVuls{} problem for any node $v$, 
let $I_v$ denote the set of neighbors of $v$
which are in state \istate{} at $t = 0$.
Also, let $R_v$ denote the set of neighbors of $v$ such that 
each node $w \in R_v$ satisfies two conditions: 
$w$ is in state \sstate{} at $t = 0$ 
and $w$ is adjacent to 
at least one node in state \istate{} in the initial configuration.
%%Let $\tau_v$ denotes the threshold of node $v$.
We can compute the vulnerability of $v$ at $t = 2$ as follows.
If~ $|I_v| \geq \tau_v$,~ then the vulnerability of $v$ at $t = 2$ is 
given by  $(1-p_v) \times p_v$. 
Otherwise, we proceed as follows.
\begin{description}
\item{(a)} Compute the vulnerability $\pi(v)$ of $v$ at $t = 1$ as discussed above.
\item{(b)} Compute the the probability $\pi'$ that at least $\tau_v - |I_v|$ nodes in $R_v$ 
           are infected at $t = 1$.  (This can be done using the dynamic programming
           algorithm discussed in the proof of Proposition~\ref{pro:pt_t_1_dyn_prog}.)
\item{(c)} The vulnerability of $v$ at $t = 2$ is given by 
           $[1-\pi(v)] \times \pi' \times p_v$. 
\end{description}
It can be verified that the above algorithm runs in polynomial time.
\QED

Our results for forecasting problems under the SI, SIS and PT models
are summarized in Table~\ref{tab:results_for_models}.

\begin{table}
\begin{center}
\begin{tabular}{|c|p{4in}|}\hline
\multicolumn{1}{|c|}{\textbf{Model}} &
\multicolumn{1}{|c|}{\textbf{Results}} \\ \hline\hline
SI Model & {(a)~\tNewInfs, \tTotInfs{} and \tPeak{} are \cnump-hard for any $t \geq 2$.
            \newline
           (b)~ There is a randomized approximation scheme for \newline
           \tTotVuls{} when $t$ and $|S|$ are fixed.} \\ \hline
SIS Model & {(a)~\tNewInfs, \tTotInfs{} and \tPeak{} 
             are \cnump-hard for any $t \geq 2$.  \newline
           (b)~ There is a randomized approximation scheme for \newline
           \tTotVuls{} when $t$ and $|S|$ are fixed. 
           (The quantity being approximated is the probability that the
            given node gets infected at least once by time $t$.)} \\ \hline
PT Model & {(a)~\tNewInfs, \tTotInfs{} and \tPeak{} are 
            \cnump-hard for any $t \geq 2$.  \newline
           (b)~ \tVuls{} is \cnump-hard for any fixed $t \geq 3$ even
              when $|S| = 1$; however, the problem is
           efficiently solvable for any $S$ for $t = 1$ and $t = 2$.} \\ \hline\hline
\end{tabular}
\end{center}
\caption{Summary of Results for Forecasting Problems Under Various Models}
\label{tab:results_for_models}
\end{table}

