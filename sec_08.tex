\section{Conclusions and Future Work}
\label{sec:concl}

We formulated several short term forecasting problems under
the SIR model. 
Our results show that many of these problems are computationally
intractable even when the time horizon is as small as two.
These results supplement the reasons discussed in
\cite{Drake-2005,Drake-2006} for the difficulty of accurately predicting
several disease parameters. 
We also presented extensions of our results to realistic 
social networks.
Further, we demonstrated the tightness of our complexity results 
by showing that many of the problems can be solved efficiently
when the time horizon is one.
We also presented randomized approximation algorithms for 
some short term forecasting problems. 
In addition, we showed how our results can be extended to a
number of additional forecasting measures introduced in \cite{TC+2016}
and to three other common epidemic models. 

We close by pointing out some directions for future research.
One direction is to investigate whether there are useful classes
of social networks for which short term forecasting problems can be
solved efficiently. 
In a companion paper \cite{Rosenkrantz_etal_2016}, we have
shown that many forecasting problems under the SIR model can be solved efficiently
when the treewidth of the underlying graph is bounded.
A second direction is to study the complexity of the problem of 
determining the expected number of infections at time $t$, 
where $t$ is a \emph{fixed} value $\geq 3$. 
(We showed in Section~\ref{sec:poly_versions} that the problem can be 
solved efficiently for $t = 1$ and $t = 2$.) 
We also showed that the problem of computing the probability 
of reaching a configuration
that satisfies an $r$-symmetric constraint in one time step
can be solved efficiently.
It is of interest to investigate whether this result can be
extended to other forms of constraints or to special classes of
networks.
Another direction is to study forecasting problems where the input consists
of two or more networks and the goal is to determine which of the networks 
is more likely to cause a given number of infections at a specified time.
Finally, it is also of interest to study forecasting problems
under a model where the initial configuration is specified by a 
vector of probability values, with each value denoting the probability
that the corresponding node is in state \istate{}.
