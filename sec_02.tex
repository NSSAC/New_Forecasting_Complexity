\section{SIR Systems as Stochastic Dynamical Systems}
\label{sec:sir_dyn_system}

A general model for \emph{stochastic}
synchronous dynamical systems (SSyDSs) was introduced in \cite{BH+2011}.
In this section, we point out that SIR systems are a special form of SSyDSs.
Each SSyDS \cals{} has three components, namely a domain \cald{}
(i.e., a finite set of permissible state values),
an underlying graph $G(V,E)$ with $n$ nodes and 
a collection ${\cal F} = \{f_1, f_2, \ldots, f_n\}$ of functions,
where $f_i$ is the \textbf{local transition function}
associated with node $v_i \in V$,
$1 \leq i \leq n$.
In the context of SIR systems, the domain \cald{} is
is the set \{\sstate, \istate, \rstate\}.
Some additional terminology is needed to specify
the local transition functions for SIR systems. 
For any node $v \in V$, the \textbf{neighborhood} of $v$, denoted
by $N_v$, contains each node $u$ such that the edge $\{u,v\}$ is in $E$. 
Each edge $e = \{u,v\} \in E$ is associated with a 
\textbf{transmission probability} $p_e$. 
A {\bf configuration} of an SIR system at time $t$ is an $n$-vector   
$(b_1^t, b_2^t, \ldots, b_n^t)$,  
where $b_i^t \in \cald$ is the value of the state of 
node $v_i$ at time $t$, $1 \leq i \leq n$.  

A single transition of an SIR system from one configuration to another is   
obtained by updating the state of each node $v_i$ \emph{synchronously}
using the corresponding local transition function $f_i$ whose 
specification is as follows.
\begin{description}
\item{(a)} If the state of $v_i$ at time $t$ is \rstate, then
the state of $v_i$ at time $t+1$ is also \rstate.
(Thus, once a node reaches the state \rstate, it remains
in that state forever.)

\item{(b)} If the state of $v_i$ at time $t$ is \istate, then
the state of $v_i$ at time $t+1$ is \rstate.
(Thus, each node remains in state \istate{} for exactly one 
time unit.)

\item{(c)} If the state of $v_i$ at time $t$ is \sstate, then 
the state of $v_i$ at time $t+1$ is determined by the following 
stochastic process.
Let $X_i(t) \subseteq N_{v_i}$ denote the set of neighbors of 
$v_i$ whose state is \istate{} at time $t$, and
let $\pi(i,t)$ be defined as follows. 
\begin{center}
\begin{tabular}{lcll} 
$\pi(i,t)$  & = & 0 & \textrm{if}~ $X_i(t) ~=~ \emptyset$ \\ 
            & = & $1 - \displaystyle{\prod_{u \in X_i(t)} 
                         (1 - p_{\{u,v_i\}}})$ &
                \textrm{otherwise.} 
\end{tabular}
\end{center}
The state of $v_i$ at time $t+1$ is \istate{} 
with probability $\pi(i,t)$ and \sstate{} with probability $1 - \pi(i,t)$.
\end{description}

\noindent
\textbf{Explanation for the Equation for $\pi(i,t)$:}~
If $X_i(t) = \emptyset$, none of $v_i$'s neighbors 
can infect $v_i$ at time $t$. 
Thus, $v_i$ remains in state \sstate{} at time $t+1$.
Otherwise, at time $t$, each node $u \in X_i(t)$ 
tries to infect $v_i$, \emph{independently} of the other nodes in $X_i(t)$,
with probability $p_{\{u,v_i\}}$, the transmission probability of
the edge $\{u,v_i\}$.
The expression for $\pi(i,t)$ shown above
gives the probability that at least one of 
these attempts is successful.

\medskip
Initially (i.e., at $t = 0$), one or more nodes are in state \istate{}
and the other nodes are in state \sstate.
Starting from the given initial configuration $\calc_0$, the system
goes through a sequence of configurations over time.
If the configuration at time $t$ is $\calc_t$, the next configuration
$\calc_{t+1}$ is called a \textbf{successor} of $\calc_t$.
Since the system is stochastic, there may be more than one successor
for a given configuration.
For a general SSyDS, one of the successors of a configuration \calc{}
may be \calc{} itself.
Such a configuration is called a \textbf{pseudo fixed point}.
In addition, an SSyDS may also have a configuration \calc{} such that
the only successor of \calc{} is \calc{} itself; such a configuration
is called a \textbf{true fixed point}.
As we observe in Section~\ref{sse:prelim}, for any SIR system,
every pseudo fixed point is a true fixed point. 
%% Any configuration \calc{} which is its own successor is
%% called a \textbf{fixed point}. 

\medskip
We say that an SIR system is \textbf{uniform} if all the edges in
the underlying graph have the same transmission probability $p$.
Also, we say that a node $v$ \textbf{gets infected} 
during an execution of an SIR system 
if the state of $v$ changes to \istate{} at
some time during the execution.
The \textbf{vulnerability} of a node $v$
is the probability that $v$ gets infected during 
a run of the SIR system \cite{EK+2005}.
%%a run of the SIR system \cite{EK+2005}.

\begin{figure}[tbh]
\rule{\textwidth}{0.01in}
\begin{minipage}[t]{0.5\textwidth}
%%\vspace*{0.2in}
\medskip
%%\mbox{
\begin{center}
\input{sir_example.pdf_t}
\end{center}
%%}
\medskip
\end{minipage}
%% \hspace*{0.1in}
\hfill
\begin{minipage}[t]{0.5\textwidth}
\vspace*{0.3in}
\begin{center}
\begin{tabular}{|c|l|}\hline
\textbf{Time} & \textbf{Configuration} \\ \hline\hline
0 & $(\istate, \sstate, \sstate, \sstate, \sstate, \sstate, \sstate)$ \\ \hline 
1 & $(\rstate, \istate, \istate, \sstate, \sstate, \sstate, \sstate)$ \\ \hline 
2 & $(\rstate, \rstate, \rstate, \sstate, \istate, \sstate, \istate)$ \\ \hline 
3 & $(\rstate, \rstate, \rstate, \istate, \rstate, \sstate, \rstate)$ \\ \hline 
4 & $(\rstate, \rstate, \rstate, \rstate, \rstate, \sstate, \rstate)$ \\ \hline 
\end{tabular}
\end{center}
\medskip
\end{minipage}
\caption{The graph of an SIR system and one possible sequence of its
configurations, leading to a fixed point at time $t = 4$. 
(Each configuration is a 7-tuple $(s_0, s_1, s_2, s_3, s_4, s_5, s_6)$,
where $s_i \in$ \{\sstate, \istate, \rstate\} is the
state of node $v_i$, $0 \leq i \leq 6$.)
}
\label{fig:sir_example}
\smallskip
\rule{\textwidth}{0.01in}
\end{figure}

\begin{example} \label{ex:sir_example}
%%\noindent
%%\textbf{Example 2.1:}~
The graph of an SIR system consisting of seven 
nodes is shown in Figure~\ref{fig:sir_example}.
The nodes are labeled $v_0$ through $v_6$ and the transmission
probability of each edge is also shown.
Suppose at $t = 0$, node $v_0$ is in state \istate{} and all
other nodes are in state \sstate. 
Starting from this initial configuration, one possible sequence of 
configurations that the system may
go through before reaching a fixed point at time $t = 4$ is shown
in the table in the figure.

As shown in that table, at $t = 1$, nodes $v_1$ and $v_2$ get infected,
$v_0$ changes to state \rstate{} and other nodes remain in state \sstate.
Further, at $t = 2$, nodes $v_4$ and $v_6$ get infected,
$v_1$ and $v_2$ change to state \rstate{} and nodes $v_3$ and $v_5$
remain in state \sstate.
Given the configuration at time $t = 1$, the probability that $v_4$
gets infected at time $t = 2$ can be computed as follows.
The infected neighbors of $v_4$ are $v_1$ and $v_2$, and the
transmission probabilities of the edges $\{v_1, v_4\}$ and $\{v_2, v_4\}$
are $3/4$ and $1/2$ respectively.
Thus, the probability that $v_4$ gets infected at time $t = 2$ is
given by $1 - (1-3/4)(1-1/2)$ = $1 - 1/8$ = $7/8$.
In a similar manner, the probability that node $v_6$ gets infected
at $t = 2$ is $1 - (1-1/2)$ = $1/2$.

We note that at $t = 3$, node $v_6$ (which was infected at $t = 2$)
causes $v_3$ to get infected.
At $t = 4$, the system reaches a fixed point with node $v_5$ in
state \sstate{} and all other nodes in state \rstate. \hfill $\Box$
\end{example}

\medskip
Following \cite{BH+2011},
the \textbf{generalized phase space} ${\cal P}_{\cal S}$
of an SIR system ${\cal S}$ is a directed graph defined as follows.
There is a node in ${\cal P}_{\cal S}$ for
each configuration of \cals.
There is a directed edge
from the node representing configuration $\calc_1$ to that representing
configuration $\calc_2$ if the probability that the system will reach
$\calc_2$ from $\calc_1$ in one step is (strictly) greater than zero.
In ${\cal P}_{\cal S}$, the directed edge from $\calc_1$ to $\calc_2$
is associated with a positive number $p$, $0 < p \leq 1$,
representing the probability of the corresponding 1-step transition.
Thus, the generalized phase
space represents the \textbf{Markov chain} \cite{MU-2005}
for the SIR system.
For an SIR system with $n$ nodes over the 
domain \{\sstate, \istate, \rstate\},
the number of possible configurations is $3^n$.
Thus, the size of the Markov chain of an SIR system
is \emph{exponential} in the size of the system's representation.
