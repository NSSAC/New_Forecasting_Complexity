\begin{center}
\Large{\textbf{Appendix}}
\end{center}

\noindent
\textbf{Statement of Theorem~\ref{thm:hardness_for_whole_set}:}~
\begin{description}
\item{(1)}
\emph{
For any $t \geq 2$,~ the problems \tNewInfv,~ \tTotInfv{} ~and \\
\tTotVulv{} ~are~ \cnump-hard.
}

\item{(2)}
\emph{
Unless \textbf{P} = \cnp,
for each of the problems \tNewInfv, \tTotInfv{} and
\tTotVulv{} and for any $t \geq 2$,
there is a constant $\epsilon > 0$
such that if $p^*$ is the actual solution value,
the quantity $\log{(2^n\,p^*)}$ cannot be approximated to within the factor
$O(n^{\epsilon})$ in polynomial time, where $n$
is the maximum number of nodes that can get infected at $t-1$.
}
\end{description}

\medskip

\noindent
\textbf{Proof of Part (1):}~ 

\medskip
\noindent
\textbf{Proof for} \tNewInfv:~ 
We first consider the case where $t = 2$.
The reduction from \mtsat{} for this problem is identical to the one for
\tNewInfs{} for $t = 2$ given in the proof of 
Part~(1) Theorem~\ref{thm:gen_hardness}.
In that reduction, we note that every node that gets infected at $t = 2$
is a member of the node $V_2$ which contains $m$ nodes.
Therefore, the probability that at least $m$ nodes of the entire
node set $V$ get infected at $t = 2$ is equal to the probability that
all the $m$ nodes in $V_2$ get infected at time $t = 2$.
The latter probability is equal to $N/2^n$, where $n$ and $N$ denote 
respectively the number of variables in and the number of satisfying
assignments of the given \mtsat{} instance.
The result for \tNewInfv{} for $t = 2$ follows.
The construction for $t \geq 3$ is also identical to that presented
in the proof of Theorem~\ref{thm:gen_hardness}.  

\medskip
\noindent
\textbf{Proof for} \tTotInfv:~
Again, we first consider the case where $t = 2$.
We modify the construction of graph $G(V,E)$ 
and other components of the problem instance from the construction
used in the proof of Theorem~\ref{thm:gen_hardness} as follows.

\begin{description}
\item{(a)} Node sets $V_0$ and $V_1$ are unchanged.
Node set $V_2$ is modified to contain $mn$ nodes; further, 
$V_2$ is partitioned into $m$ subsets $W_1$, $W_2$, \ldots $W_m$,
where, for $1 \leq j \leq m$, subset $W_j = \{w_j^1, w_j^2, \ldots, w_j^n\}$ 
corresponds to clause $C_j$ and contains $n$ nodes.
Thus, $|V| = mn+n+1$.

\item{(b)} Edge set $E_1$ is unchanged. Edge set $E_2$ is modified as follows.
For each clause $C_j = (x_a \vee x_b)$, $1 \leq j \leq m$,
$E_2$ has the following $2n$ edges: 
$\{v_a, w_j^1\}$, $\{v_a, w_j^2\}$, $\ldots$, $\{v_a, w_j^n\}$, 
$\{v_b, w_j^1\}$, $\{v_b, w_j^2\}$, $\ldots$, $\{v_b, w_j^n\}$.
Thus, $|E| = n+2mn$.

\item{(c)} The value of $q$ in specifying event \cale{} is 
given by $q = mn + 2$.
\end{description}

\noindent
The construction of the transmission probabilities and initial
configuration are unchanged.
The event \cale{} whose probability is to be computed 
is that the number of nodes of $V$ in state \istate{} or 
state \rstate{} at $t = 2$ is at least $mn + 2$.  

To prove that the probability of \cale{} is $N/2^n$,
we establish the following: (i) each infection pattern 
corresponding to a satisfying
assignment for the \mtsat{} instance causes 
all $mn$ nodes in $V_2$
to get infected at time $t = 2$, thereby making event \cale{} occur, 
and (ii) each infection pattern corresponding to a non-satisfying assignment 
causes at most $(m-1)n$  nodes in $V_2$
to get infected at time $t = 2$, 
thereby preventing event \cale{} ~from occurring.
The proof is again in two parts.

\noindent
\underline{Part 2.1:}~
Consider any infection pattern that corresponds to a
satisfying assignment.
Since the assignment satisfies all clauses, at time $t = 2$,
each node in $V_2$ has at least one edge to a node 
in state \istate{} in $V_1$.
Moreover, at least one node in $V_1$ is in state \istate{}  at $t = 1$. 
Thus, including node $s$ which was infected at time 0,
at least $mn+2$ nodes are infected by time $t = 2$.  

\noindent
\underline{Part 2.2:}~
Consider any infection pattern that corresponds to a
non-satisfying assignment.
Then, there is at least one clause $c_j$ that is not satisfied.
Each node in set $W_j$ has two neighbors, say $v_a$ and $v_b$,
in $V_1$, and both $v_a$ and $v_b$ are in state \sstate{} at the 
end of time step $t = 1$.
Therefore, the nodes in $W_j $ cannot be infected at $t = 2$.
Thus, even if every node in $V_1$ gets infected at time $t = 2$,
at most $1 + n + (m-1)n$ nodes are infected by  time $t = 2$,
that is, the total number of nodes infected  by  time $t = 2$ 
is at most $mn+1$.

This completes the proof of \cnump-hardness of \TwoTotInfv.
The \cnump-hardness of\\ \tTotInfv{} for set $V$ for any $t \geq 3$ can be 
proven in a manner similar to that used in the proof of 
Theorem~\ref{thm:gen_hardness}.


\medskip
\noindent
\textbf{Proof for} \tTotVulv:~
Again, we first consider the case where $t = 2$.
We remind the reader that \tTotVulv{} problem at $t = 2$
requires the computation of the probability that \emph{all} the nodes
in $V$ get infected \emph{by} time $t = 2$.
We modify the construction of graph $G(V,E)$ 
and other components of the problem instance from the construction
used in the proof of Theorem~\ref{thm:gen_hardness} as follows.

\begin{description}
\item{(a)} Node sets $V_0$, $V_1$ and $V_2$ are unchanged.
Thus, $|V| = m+n+1$.

\item{(b)} Edge sets $E_1$ and $E_2$ are unchanged. 
An additional set of edges, denoted by $E_3$, includes an edge
between every pair of nodes in $V_1$.
(The edges in $E_3$ ensure that the nodes of $V_1$ form a clique.)
Hence, $|E| = n(n+1)/2+2m$.

\item{(c)} The transmission probabilities of the edges in $E_1 \cup E_2$
remain the same as before.
The transmission probability of each edge in $E_3$ is set to 1.
\end{description}

\noindent
Consider any satisfying assignment for the given \mtsat{} instance.
Such an assignment must infect at least one node in $V_2$ at $t = 1$ and
causes all the nodes in $V_2$ to get infected at $t = 2$.
Since the nodes in $V_1$ form a clique and the transmission probability
of each edge in that clique is 1, any node of $V_1$ which is in state \sstate{}
at $t = 1$ gets infected at $t = 2$.
In other words, every satisfying assignment to the \mtsat{} instance causes
all the nodes in $V$ to be infected \emph{by} time $t = 2$.
Any assignment that does not satisfy the given \mtsat{} instance leaves
at least one node of $V_2$ to remain in state \sstate{} at $t = 2$.
Therefore, the probability that all nodes of $V$ get infected by $t = 2$
is equal to the probability that an assignment chosen uniformly randomly
satisfies the given \mtsat{} instance.
It follows that the \tTotVulv{} problem is \cnump-hard for $t = 2$.
The \cnump-hardness of \tTotVulv{} for any $t \geq 3$ can be 
proven in a manner similar to that used in the proof of 
Theorem~\ref{thm:gen_hardness}.

\medskip

\noindent
\textbf{Proof of Part (2):}~ The reductions used in proof of Part~(1) above show
that for each of the three problems, the solution value $p^*$ is the
probability that the an assignment chosen uniformly randomly from the set
of all possible assignments to the \mtsat{} instance.
Also, in each case, $n$ is the maximum number of nodes that can get
infected at time $t-1$.
Thus, the result of Part~(2) follows using the same argument as the one
used to prove Part~(2) of Theorem~\ref{thm:gen_hardness}.
\QED
