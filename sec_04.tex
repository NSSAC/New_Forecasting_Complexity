%% \section{Computational Intractability Results for Forecasting 
%% Problems Over a Collection of Configurations}
\section{Computational Intractability Results for 
General Networks}
\label{sec:general_results}

%\newcommand{\cali}{\mbox{${\cal I}$}}  

%%\newcommand{\ThrTotInf}{\mbox{$3$}-\textsc{TotInf}}

\subsection{Overview}
\label{sse:results_overview}

In this section, we show that the forecasting problems 
defined in Section~\ref{sse:prob_formulation} are computationally
intractable in general even for small values of $t$.
We start with a result (Theorem~\ref{thm:gen_hardness}) 
to convey how a known computationally intractable problem (namely, \mtsat)
can be reduced to forecasting problems under the SIR model.
%% We also demonstrate that this basic result is tight by showing that 
%% the version of forecasting problems which are \cnump-hard{}
%% for $t = 2$ are efficiently solvable for $t = 1$. 
In Section~\ref{sec:realistic},
we show how these hardness results can be extended 
to restricted classes of graphs 
(such as graphs with low diameter or high average 
clustering coefficient and power law graphs).

\subsection{Results for General Graphs}
\label{sse:gen__results}

\begin{theorem}\label{thm:gen_hardness} \leavevmode
\begin{description}
\item{(1)} For any $t \geq 2$,~ the problems~ \tNewInfs,~ \tTotInfs,~ \tVuls{} 
and~ \tTotVuls{} are~ \cnump-hard. 

\item{(2)}
Unless \textbf{P} = \cnp, 
for each of the problems \tNewInfs, \tTotInfs, \tVuls{} and
\tTotVuls{} and for any $t \geq 2$,
there is a constant $\epsilon > 0$
such that if $p^*$ is the actual solution value,
the quantity $\log{(2^n\,p^*)}$ cannot be approximated to within the factor
$O(n^{\epsilon})$ in polynomial time, where $n$
is the maximum number of nodes that can get infected at time $t-1$. 

\item{(3)} Problem~ \tPeak{} is~ \cnump-hard ~for any $t \geq 1$.
\end{description}
\end{theorem}


%% \medskip
\noindent
\textbf{Proof:}~ 

\noindent
\textbf{Proof of (1):} We first provide a single reduction that establishes the
\cnump-hardness of all the four problems for $t = 2$.
We then show how the reduction can be modified to obtain the results 
for any $t \geq 3$.

\medskip
We begin by establishing the hardness results for $t = 2$. 
The reduction is from the \mtsat{} problem defined in 
Section~\ref{sss:boolean_sat}.
Let the given instance of \mtsat{} consist of the
set $X = \{x_1, x_2, \ldots, x_n\}$ of $n$ Boolean variables and
set $C = \{C_1, C_2, \ldots, C_m\}$ of $m$ clauses, 
where each clause $C_j = (x_a \vee x_b)$ for some variables $x_a$ and
$x_b$ $\in X$.
The graph $G(V,E)$ and the other components of the \TwoNewInfs{} 
problem instance are constructed as follows.
\begin{description}
\item{(a)} The node set $V$ consists of three pairwise disjoint subsets
$V_0$, $V_1$ and $V_2$.
Node set $V_0 = \{s\}$ contains just one node.
Node set $V_1 = \{v_1, v_2, \ldots, v_n\}$ contains $n$ nodes, 
with node $v_i$ corresponding to Boolean variable $x_i$, $1 \leq i \leq n$. 
We refer to these nodes as {\bf variable nodes}.
Node set $V_2 = \{w_1, w_2, \ldots, w_m\}$ contains $m$ nodes, 
with node $w_j$ corresponding to clause $C_j$, $1 \leq j \leq m$. 
We refer to these nodes as {\bf clause nodes}.
Thus, $|V| ~=~ m+n+1$.
%% Node set $V_2$ is partitioned into $m$ subsets $W_1$, $W_2$, \ldots $W_m$,
%% where subset $W_j = \{w_j^1, w_j^2, \ldots, w_j^{n+1}\}$ 
%% corresponds to clause $C_j$ and contains $n+1$ nodes, $1 \leq j \leq m$.
%% (Thus, $V$ has a total of $(m+1)(n+1)$ nodes.)

\item{(b)} The edge set $E$ consists of two disjoint sets $E_1$ and $E_2$.
Here, $E_1 = \{\{s, v_i\}: 1 \leq i \leq n\}$.
For clause $C_j = (x_a \vee x_b)$ of the \mtsat{} instance,
$E_2$ has the two edges $\{v_a, w_j\}$ and
$\{v_b, w_j\}$, $1 \leq j \leq m$.
%% $\{v_b, w^r_j\}$, $1 \leq r \leq n+1$ and $1 \leq j \leq m$.
Thus, $|E| ~=~ n+2m$. 

\item{(c)} 
%% The transmission probabilities on the edges are
%% chosen as follows.
For each edge in $E_1$, the transmission probability is set to $1/2$;~ 
for each edge in $E_2$, the transmission probability is set to $1$. 

\item{(d)} Initially, node $s$ is in state \istate{} and all
other nodes are in state \sstate.

\item{(e)} For all the four problems, $V_2$ serves 
as the set $S$. 
For the \TwoNewInfs{} and \\
\TwoTotInfs{} problems, the quantity $q$ in specifying 
event \cale{} is given by $q = |V_2| = m$.
\end{description}

\begin{figure}[tbh]
\rule{\textwidth}{0.01in}
%%\vspace*{0.2in}
\smallskip
\noindent
Boolean variable set $X = \{x_1, x_2, x_3\}$.

\smallskip
Clause set $C = \{(x_1 \vee x_2),~ (x_1 \vee x_3),~ (x_2 \vee x_3)\}$.

\medskip
%%\mbox{
\begin{center}
\input{tnew_inf_reduction.pdf_t}
\end{center}
%%}
\caption{An example for the reduction presented in
the proof of Part~(1) of Theorem~\ref{thm:gen_hardness}.
Parts (a) and (b) of the figure show the resulting
graphs of the SIR system for $t = 2$ and $t = 4$ respectively.
}
\label{fig:tnewinf_proof}
\smallskip
\rule{\textwidth}{0.01in}
\end{figure}

This completes the construction and 
it can be seen the construction can be carried out in polynomial time.
For problem \TwoNewInfs{} 
the goal is to compute the probability of the following event \cale:
the number of new infections at $t = 2$ is at least $m = |S|$.  
For problem \TwoTotInfs{} 
the goal is to compute the probability of the following event \cale':
the total number of infections by $t = 2$ is at least $m = |S|$.  
Since our reduction sets $m = |S|$, these two problems also correspond to 
\TwoVuls{} and \TwoTotVuls{} respectively.

\medskip
An example of this construction is shown in
Figure~\ref{fig:tnewinf_proof}(a).
We first prove the correctness of the reduction for the \TwoNewInfs{}
problem.
The proof for \TwoTotInfs{} follows from this result
via simple observations.

\medskip
\noindent
\textbf{Correctness of the reduction for}~ \TwoNewInfs:~
We start with the following claim.

\medskip
\noindent
\textbf{Claim 1:}~ Let \cale{} denote the event that at least 
$m$ nodes of $V_2$ get infected at $t = 2$.
Further, let $N$ denote the number of satisfying assignments 
to the \mtsat{} instance.
The probability that the event \cale{} is equal to $N/2^n$.

\noindent
\textbf{Proof of Claim 1:}~ By our construction, only nodes in $V_1$
can get infected at time $t = 1$.
Further, since each node stays in state \istate{} for only one time
unit, only nodes in $V_2$ may get infected at time $t = 2$.
Since \cale{} requires at least $m$ new infections at time $t = 2$
and $V_2$ contains exactly $m$ nodes, event \cale{} is the same
as the event that all nodes in $V_2$ get infected at $t = 2$.

Call any subset of $V_1$ an \textbf{infection pattern}.
Since $V_1$ is in one-to-one correspondence with the 
Boolean variable set $X$, there is a
simple one-to-one correspondence between
the set of assignments to the variables in $X$
and the set of infection patterns of $V_1$: 
each infection pattern $P \subseteq V_1$ corresponds to the truth
assignment that sets the variables corresponding to the nodes in $P$
to \texttt{True} and the remaining variables to \texttt{False}. 
Since the transmission probability of each edge in $E_1$ is $1/2$,
the probability of each infection pattern is $1/2^n$. 

We now argue that each infection pattern corresponding to a satisfying
assignment for the \mtsat{} instance causes all nodes in $V_2$
to get infected at time $t = 2$, and that
each infection pattern corresponding to a non-satisfying
assignment for the \mtsat{} instance causes a \emph{proper} subset of 
$V_2$ to get infected at time $t = 2$.
The proof is in two parts.

\noindent
\underline{Part 1.1:}~
Consider any infection pattern that corresponds to a
satisfying assignment.
Since the assignment satisfies all clauses, at time $t = 2$,
each node $w_j \in V_2$ has at least one edge to a node 
in state \istate{} in $V_1$.
Further, since all the transition probabilities for the edges in $E_2$ are 1,
each node $w_j \in V_2$ gets infected at $t = 2$. 

\noindent
\underline{Part 1.2:}~
Consider any infection pattern that corresponds to a
non-satisfying assignment.
Then, there is at least one clause $C_j$ that is not satisfied.
The corresponding node $w_j$ has two neighbors, say $v_a$ and $v_b$,
in $V_1$, and both $v_a$ and $v_b$ are in state \sstate{} at the 
end of time step $t = 1$.
Therefore, $w_j \in V_2$ cannot be infected at time $t = 2$.
In other words, only a proper subset of $V_2$ get infected at $t = 2$.

Therefore, the probability of the event \cale{} is the same as
the probability that an assignment chosen uniformly randomly from
the set of all assignments satisfies all the clauses of the given
\mtsat{} instance.
Since the latter probability is $N/2^n$, Claim~1 follows. 

In view of Claim~1, if the probability of event \cale{} can be computed 
efficiently, then $N$, the number of satisfying assignments
of the given \mtsat{} instance can also be computed efficiently. 
This establishes the \cnump-hardness of \TwoNewInfs.


\medskip
\noindent
\textbf{Correctness of the reduction for}~ \TwoTotInfs:~
First consider the case where $t = 2$.
We note that in graph $G$, the earliest time at which any node in 
$V_2$ can get infected is $t = 2$.
Therefore, the probability that all $m$ nodes in $V_2$ get infected
\emph{by} time $t = 2$ is equal to the probability that they get infected
at time $t = 2$.
This shows the \cnump-hardness of \tTotInfs{} for $t = 2$.
The construction for any $t \geq 3$ is identical to that presented for
the \tNewInfs{} problem above.

\medskip
\noindent
\textbf{Correctness of the reduction for}~ \TwoVuls:~
Since set $S = V_2$, the solution value for \tVuls{} 
is the same as that for \tNewInfs.
Thus, the \cnump-hardness of \tVuls{} for any $t \geq 2$ 
follows from the above result for \tNewInfs.

\medskip
\noindent
\textbf{Correctness of the reduction for}~ \TwoTotVuls:~
Since set $S = V_2$, the solution value for \tTotVuls{} 
is the same as that for \tTotInfs.
Thus, the \cnump-hardness of \tTotVuls{} for any $t \geq 2$ 
follows from the above result for \tTotInfs.

\medskip
\noindent
\textbf{Extension of the results for any}~ $t \geq 3$:~
The \cnump-hardness of \tNewInfs{} for any $t \geq 3$ can be 
proven through a simple modification to the above construction.
The graph $G(V,E)$ constructed above is
augmented as follows.
Given an integer $t \geq 3$, we construct a simple path 
$\langle s_0, s_1, \ldots, s_{t-3}\rangle$
consisting of $t-2$ new nodes and add an edge from $s_{t-3}$ to $s$.
The transmission probability for all the new edges is set to 1.
At time 0, node $s_0$ is in state \istate{} and all 
other nodes are in state \sstate. 
(An example of this construction for $t = 4$ is shown in
Figure~\ref{fig:tnewinf_proof}(b).)
The new nodes and edges added to $G$ ensure that 
node $s$ gets infected at time $t-2$.
The requirement is to compute the probability that at least $m$
nodes in $V_2$ get infected at time $t$.  
It can be verified that the probability of this event is the
same as the probability that an assignment chosen uniformly
randomly satisfies all the clauses of the \mtsat{} instance.
It can be seen that the construction also establishes
the \cnump-hardness of \tTotInfs{} for any $t \geq 3$.
Since the instances resulting from these constructions 
actually correspond to \TwoVuls{} and \TwoTotVuls,
we also obtain \cnump-hardness of those two problems 
for any $t \geq 3$.

\medskip
This completes the proof of Part~(1).

\medskip
\noindent
\textbf{Proof of Part (2):}~ 
We use the same reduction from \mtsat{} given in Part~(1).
For any of the four problems, let $p^*$ denote the actual solution
value.
As shown in the proof of Part~(1), $p^* = N/2^n$, where $n$ and $N$
denote respectively the number of variables in and the number of satisfying
assignments of the given \mtsat{} instance.
Thus, $N = 2^n\,p^*$.
By Theorem~\ref{thm:m2sat_zuckerman}, there is an $\epsilon > 0$ such that
$N = 2^n\,p^*$ cannot be approximated to within the factor 
$O(n^{\epsilon})$ in polynomial time, unless \textbf{P} = \cnp.
In our transformation, $n = |V_1|$, the maximum number of nodes that 
can be infected at time $t-1$.
The last two statements together imply the result of Part~(2). 

\medskip
\noindent
\textbf{Proof of Part (3):}~ We first show that \OnePeak{} is
\cnump-hard. 
This is done by modifying the construction presented in
the proof of Part~(1) above as follows.

\begin{description}
\item{(a)} Node set $V_0$ is unchanged.
Node set $V_1$ is partitioned into two sets $V_{11}$ and $V_{12}$,
where $V_{11} = \{y_1, y_2, \ldots, y_r\}$
consists of $r = m(n+2)-n-1$ nodes and $V_{12} = \{v_1, v_2, \ldots v_n\}$
consists of $n$ nodes. 
($V_{12}$ is in one-to-one correspondence
with the set $X$ of Boolean variables of the~ \mtsat{} instance.) 
Also, $V_2$ is partitioned into $m$ subsets $W_1$, $W_2$, \ldots $W_m$,
where subset $W_j = \{w_j^1, w_j^2, \ldots, w_j^{n+2}\}$ 
corresponds to clause $C_j$ and contains $n+2$ nodes, $1 \leq j \leq m$.
Thus,  $|V| = 2m(n+2)$. 

\item{(b)} The edge set $E$ consists of three disjoint sets 
$E_1$, $E_2$ and $E_3$.
Here, $E_1 = \{\{s, v_i\}: 1 \leq i \leq n\}$ and
$E_2 = \{\{s, y_i\}: 1 \leq i \leq r\}$.
For each clause $C_j = (x_a \vee x_b)$ of the~ \mtsat{} instance
($1 \leq j \leq m$),
$E_3$ has the $2(n+2)$ edges $\{v_a, w_j^{\ell}\}$ and
$\{v_b, w_j^{\ell}\}$, $1 \leq \ell \leq n+2$. 
Thus, $|E| ~=~ 3m(n+2)-1$. 

\item{(c)} 
For each edge in $E_1$, the transmission probability is set to $1/2$; 
for each edge in $E_2 \cup E_3$, 
the transmission probability is set to $1$. 

\item{(d)} Initially, node $s$ is in state \istate{} and all
other nodes are in state \sstate.

\item{(e)} The event \cale{} ~whose probability is to be
computed is the following: the peak infection occurs at $t = 1$. 
\end{description}
In the following, we refer to any subset of $V_{12}$
as an \textbf{infection pattern}.
To specify an equation for the probability of event \cale,
we need the following claim. 

\medskip
\noindent
\textbf{Claim 2:}~ In the instance of the \OnePeak{} problem
constructed above, the peak occurs at either $t = 1$ or $t = 2$. 

\noindent
\textbf{Proof of Claim 2:}~ We first observe that at $t = 0$, only
one node (namely, $s$) is infected.
Further, at $t = 1$, all
the $r = m(n+2)-n-1$ nodes in $V_{11}$ get infected.
Using our assumption that $m \geq 2$ and $n \geq 2$,
it can be seen that $r \geq 5$. 
Thus, the peak cannot occur at $t = 0$.
To show that a peak cannot occur at values of $t > 0$,
except possibly for $t = 1$ or $t = 2$, 
we have two cases to consider. 

\noindent
\underline{Case 2.1:}~ Integer $t$ is \emph{odd}. 

Since we are excluding $t = 1$, $t$ is odd and $t \geq 3$.
For any such value of $t$, it can be seen from our construction
that new infections may occur only among the nodes in $V_{12}$. 
Note that $|V_{12}| = n$ and $r = m(n+2)-n-1 > n$ (since $m \geq 2$). 
Now, since at least $r > n$ new infections occur at $t = 1$, 
a peak cannot occur at any odd value of $t \geq 3$.

\noindent
\underline{Case 2.2:}~ Integer $t$ is \emph{even}. 

Since we are excluding $t = 0$ and $t = 2$, 
$t$ is even and $t \geq 4$.
For any such value of $t$, it can be seen from our construction
that new infections may occur only among the nodes in $V_2$. 
To show that the number of new infections in $V_2$ cannot 
be a peak for even values of $t \geq 4$, 
we consider the following two subcases.    

\noindent
\underline{Subcase 2.2.1:}~ The infection pattern at $t = 1$ 
is the empty set.

In this case, at $t = 2$, no new infections occur; 
that is, the system reaches a fixed point at $t = 2$. 
In other words, the number of new infections at all even values 
of $t \geq 2$ is 0.
Consequently, a peak cannot occur at even values of $t \geq 4$.  

\noindent
\underline{Subcase 2.2.2:}~ The infection pattern at $t = 1$ 
is non-empty.

Suppose node $v_i \in V_{12}$ is infected at $t = 1$. 
The Boolean variable $x_i$ corresponding to $v_i$ occurs
in at least one clause $C_j$ of the \mtsat{} instance.
In other words, for this case, the number of new infections at $t = 2$ is
at least $|W_j| = n+2$.
Therefore, the number of new infections that can occur 
for any even value of $t \geq 4$ is at most 
$|V_2| - (n+2)$ = $(m-1)(n+2)$ 
which can be seen to be less than $r = m(n+2)-n-1$.
Since at least $r$ new infections occur at $t = 1$, a peak cannot
occur at even values of $t \geq 4$.
This completes the proof of Case~3.2 and also that of Claim~3.

\medskip
\noindent
\textbf{Claim 3:}~
Let $N$ denote the number of satisfying assignments 
to the~ \mtsat{} instance.
The probability of the event \cale{} is equal to $1-(N/2^n)$.

\noindent
\textbf{Proof of Claim 3:}~ 
In view of Claim 3, it is enough to show that the probability that 
the peak occurs at $t = 2$ is $N/2^n$.
To do this, we show the following: (i) every infection pattern corresponding to
a satisfying assignment to the~ \mtsat{} instance causes
the peak infection to occur at time $t = 2$ and (ii)
every infection pattern corresponding to
a non-satisfying assignment causes the peak infection to
occur at time $t = 1$. 

\noindent
\underline{Part 3.1:}~ Consider any infection pattern corresponding
to an assignment that satisfies all the clauses.
The largest number of new infections at $t = 1$ is bounded by $|V_1|$
= $|V_{11}| + |V_{12}|$ = $m(n+2)-1$.
Now consider the number of nodes infected at time $t = 2$.
Since the infection pattern satisfies all the clauses, at $t = 2$
all the $m(n+2)$ nodes in $V_2$ get infected.
In other words, the peak occurs at $t = 2$. 

\medskip
\noindent
\underline{Part 3.2:}~ Consider any infection pattern corresponding
to a non-satisfying assignment.
Since $s$ is in state \istate{} at time $t = 0$ and the
transmission probability of each edge in $E_1$ is 1,
all the $r = m(n+2)-n-1$ nodes in $V_{11}$ get infected at $t = 1$.
In other words, the number of nodes infected at $t = 1$ is
at least $m(n+2)-n-1$.
Since the infection pattern corresponds to a non-satisfying
assignment, at most $(m-1)(n+2)$ nodes are infected at $t = 2$.
It is easy to verify that the number of new infections at $t = 2$ is
less than that at $t = 1$.
In other words, the peak occurs at $t = 1$.

Therefore, the probability of the event \cale{} is the same as
the probability that an assignment chosen uniformly randomly from
the set of all assignments does \emph{not} satisfy 
all the clauses of the given \mtsat{} instance.
Since the latter probability is $1-(N/2^n)$, Claim~4 follows. 

The \cnump-hardness of \OnePeak{} ~follows from Claim~3.

The \cnump-hardness of \tPeak{} for any $t \geq 2$ can be 
established in a manner similar to that of Part~(1). 
(The only difference is the path $\langle s_0, s_1, \ldots, s_{t-2}\rangle$
that augments the graph contains $t-1$ nodes.)
\QED

We close this section with two remarks.

\medskip

\noindent
\textbf{Remark 1:}~
The formal result stated in Part~(2) of 
Theorem~\ref{thm:gen_hardness} shows the difficulty of 
approximating the quantity $\log(2^n\,p^*)$, where
where $P^*$ is the solution value for the appropriate forecasting
problem and $n$ is the maximum number of nodes that 
can get infected at time $t-1$.
Given the underlying graph, we note that it is easy to compute the
value of $n$; it is simply the number of nodes which are at a
distance of at most $t-1$ from
any of the nodes which are in state \istate{} at $t = 0$.
Therefore, the result of Part~(2) indeed shows the difficulty 
of approximating the value of $p^*$. 

\medskip

\noindent
\textbf{Remark 2:}~ Recall that problems \tNewInfv, \tTotInfv,
\tVulv{} and \tTotVulv{} consider all nodes of the
underlying graph.
Of these four problems, we note that \tVulv{} is easy to
solve in view of the following observations.

\begin{description}
\item{(a)}
If none of the nodes of the SIR system are in state \istate{} at $t = 0$, 
then for any $t \geq 1$, no infections can occur.
Thus, the answer to the \tVulv{} problem is zero in this case
for all $t \geq 0$.

\item{(b)} If all the nodes of the SIR system are in state \istate{} at $t = 0$, 
then the answer to the problem is 1 for $t = 0$ and 
0 for for any $t \geq 1$ (since all the nodes change to 
state \rstate{} at $t = 1$).

\item{(c)}
If the number of nodes in state \istate{} at $t = 0$ is at least 1 but
\emph{less than} $|V|$, then the nodes in $V$ cannot
get simultaneously infected at any time $t \geq 0$; thus, the solution
value in this case is 0 for all $t \geq 0$.
\end{description}
The following result shows that the other three problems, namely
\tNewInfv,\\ \tTotInfv{} and \tTotVulv{} remain 
computationally intractable.

\begin{theorem} \label{thm:hardness_for_whole_set} \leavevmode
\begin{description}
\item{(1)}
For any $t \geq 2$, the problems \tNewInfv, \tTotInfv{} ~and \\
\tTotVulv{} are \cnump-hard. 
\item{(2)}
Unless \textbf{P} = \cnp,
for each of the problems \tNewInfv, \tTotInfv{} and
\tTotVulv{} and for any $t \geq 2$,
there is a constant $\epsilon > 0$
such that if $p^*$ is the actual solution value,
the quantity $\log{(2^n\,p^*)}$ cannot be approximated to within the factor
$O(n^{\epsilon})$ in polynomial time, where $n$
is the maximum number of nodes that can get infected at time $t-1$.
\end{description}
\end{theorem}

This theorem can be proved through appropriate modifications to
the construction given in the proof of 
Theorem~\ref{thm:gen_hardness}.
The details appear in the appendix.
