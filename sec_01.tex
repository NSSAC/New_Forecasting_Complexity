\section{Introduction} \label{sec:intro}

\epigraph{``Prediction is very difficult, especially if it's about the future".}
{--- \textup{Niels Bohr~ (1885--1962)}}  

\medskip
As large unexpected disease outbreaks are likely to have 
devastating economic consequences, there is increasing interest in
the development of systems that can provide early warnings regarding
epidemics.
This is borne out by the large number of epidemic forecasting challenges 
issued by various agencies; examples include 
the ``CHIKV Challenge'' by 
DARPA\footnote{\url{http://www.darpa.mil/news-events/2015-05-27}},
``Predict the Influenza Season Challenge'' by 
CDC\footnote{\url{https://www.federalregister.gov/documents/2013/11/25/2013-28198/announcement-of-requirements-and-registration-for-the-predict-the-influenza-season-challenge}},
the Dengue forecasting challenge by 
NOAA\footnote{\url{http://dengueforecasting.noaa.gov/}},
the NSF/NIH Ebola forecasting challenge and 
the IARPA Flu challenge \cite{Muthaiah_etal_2016}.
%%% Commented out since the URLs couldn't be found. -- Ravi
%% \textcolor{red}{URLs for these?}. 
%%%
There has also been a burst of activity on attempting to forecast different kinds of
phenomena such as epidemic outbreaks
\cite{nsoesie2013systematic,Nishiura:2011,Ohkusa:2011,Hall:2007,
Vespignani-BMC-2012, chakraborty:sdm14, Scarpino-Petri-2017},
cascades in social media and civil unrest
(e.g., \cite{Martin_etal_2016, Krishnan_etal_2016, Korkmaz_etal_2016,
Ramakrishnan:2014:BNE:2623330.2623373}).
The difficulty of forecasting geophysical phenomena 
such as earthquakes has also been noted in several papers
(e.g., \cite{Geller-1997,Geller-etal-1997}).
Many articles in a recent issue of \emph{Science} 
(Volume: 355, Issue: 6324, February 2017) point out
the difficulty of accurately forecasting the behaviors of
complex social systems \cite{Athey-2017,Bohannon-2017,Cederman-etal-2017,
Clauset-etal-2017,Hofman-etal-2017,Jasny-etal-2017,Kennedy-etal-2017,
Subrahmanian-etal-2017,Tetlock-etal-2017}.

\medskip
Despite a lot of work, epidemic forecasting remains poorly understood.  
A case in point is Google Flu Trends (GFT) 
\cite{ginsberg:nature08:flu-search-engine}. 
In its initial years, GFT
produced very good forecasts of flu incidence rates,
just based on search query results. 
However, the forecast accuracy decreased over time
(e.g., overestimates of the A/H3N2 epidemic
\cite{lazer:science14, olson:ploscb13}).
Since the spread of flu-like diseases is a complex stochastic process 
which depends on many time-varying factors \cite{Drake-2005, Drake-2006},
one cannot expect accurate long-term predictions
from a model such as GFT that relies on just one form of data.
It is pertinent to note that the data generated by GFT is being used at several 
institutions in developing new models for disease dynamics;
however, several fundamental questions regarding 
what can be forecast remain open.

\medskip
Many researchers have observed the need for carrying out
a systematic study of the various issues associated with forecasting
contagion dynamics in networked systems.
Some of these observations are summarized below.

\begin{itemize}
\item
Drake \cite{Drake-2005, Drake-2006} observes that
while systems may provide good forecasts 
of some epidemic  measures (e.g., timing),
there are fundamental limits on the effectiveness
of such systems in forecasting other measures (e.g., final epidemic size).
Several important reasons (such as the nature of stochastic 
disease propagation models and high sensitivity of disease parameters
to changes in the environment)
are articulated in \cite{Drake-2005,Drake-2006}
to explain such limits. 

\item
Cheng, Adamic, Dow, Kleinberg and Leskovec \cite{Cheng_etal_2016}
address the question of predicting whether a cascade will continue
to grow in a social network.  
In particular, they focus on identifying the features of a cascade
that can help in predicting the future course of the cascade.
They remark that ``a robust way to formulate the problem of cascade
prediction remains an open problem".

\item
Martin, Hofman, Sharma, Anderson and Watts \cite{Martin_etal_2016}
examine the limits of predictability
in complex social systems; they observe that even a small degree of
uncertainty can limit predictability.
Further, they raise the question of whether a phenomenon that
one is trying to predict is itself inherently unpredictable or 
the available data and techniques are inadequate to develop a
reliable prediction.
In another article, Hofman, Sharma and Watts \cite{Hofman-etal-2017}
mention that ``theoretical limits to the predictive accuracy of
complex systems must be better characterized".

\item 
Lazer, Kennedy, King and Vespignani \cite{lazer:science14} use 
Google Flu Trends as an example to caution against the use of social media 
and search information as a substitute for traditional data collection
and analysis methods to predict epidemic measures.
They highlight the need for systematically ``studying the evolution of the
socio-technical systems that are embedded in our society".

\item
Shaman et al. \cite{Shaman-etal-2012,Shaman-etal-2013}
use a differential equation based model along with new
data analysis techniques to predict the timing of the peak number of
infections for influenza outbreaks.
They demonstrate the usefulness of their framework in
predicting the peak timing for the 2012--2013 season.
They suggest that their methods can be made more robust 
when additional data is available. 

\item 
Biggerstaff et al. \cite{Biggerstaff-etal-2016} discuss results from 
the Centers for Disease Control and Prevention's (CDC's) ``Predict the
2013--2014 Influenza Season Challenge". 
They conclude that ``forecasting has become technically feasible, but
further efforts are needed to improve the accuracy so that policy makers can
reliably use those predictions".

\item
May \cite{May-2006} points out that care must be exercised in
drawing conclusions about the dynamics of ecological systems when
there is uncertainty in the structure and parameters of the 
underlying network model. 
\end{itemize}

Beckage, Gross and Kauffman \cite{beckage:ecosphere11} 
point to another another challenge
for forecasting, which they refer to as ``computational irreducibility''---this 
is the property of systems where
the dynamics cannot be predicted without observing the evolution of the system.
However, this can be difficult if determining the properties associated with
the evolution of a system is itself computationally demanding---this is referred to as 
\emph{computational intractability}
\cite{GJ-1979}, and has been a cornerstone of modern theoretical computer science.
Indeed, many researchers (e.g., Buss, Papadimitriou and Tsitsiklis \cite{Buss-etal-1991},
Moore \cite{Moore-1990}, Wolfram \cite{Wolfram-1985, Wolfram-1986})
have observed that computational intractability results for discrete
dynamical systems provide an indication of the unpredictability 
(or ``chaotic" behavior) of such systems.
The theory of computational intractability has helped 
to classify computational problems into a number of 
classes, such as \textbf{P}, \cnp, \cnump, etc. 
(see e.g., \cite{GJ-1979} for an introduction to this topic).
Informally, problems that are \cnp-hard or \cnump-hard are unlikely to 
have efficient algorithms, that is, algorithms
which run in time that is a polynomial function of the input size.
Computationally intractable problems arise in various fields 
(e.g., Mathematics, Physics, Biology, Social Science, Computer Science, 
Operations Research).  
In the context of forecasting,
some computational intractability results that arise in 
testing weather forecasts are presented in \cite{Fortnow-etal-2009};
however, the model used in their work is different our network-based
epidemic model. 

\medskip
In this paper, we show that many fundamental problems related to 
short-term predictions of epidemic properties
are computationally intractable  
even when all the model parameters are known and 
assumed to be insensitive to changes in the environment.
Using the observations in \cite{Buss-etal-1991,Wolfram-1985,Wolfram-1986,Moore-1990},
these results are indicative of the unpredictability of epidemic dynamics
in networked systems.
Thus, our results bring out a fundamental difficulty of 
predicting disease parameters. 
%%A detailed discussion on the implications of our results is given later.
%%\cite{Drake-2005,Drake-2006,beckage:ecosphere11} 
%% even when all the model
%%parameters are known and assumed to be insensitive to changes
%%in the environment.

Many computational models for the study of
epidemics have been proposed in the literature (see e.g., 
\cite{Brauer-2008,marathe:cacm13,Britton-2009,kermack+m:SIR,epi-Bailey,
epi-Hethcote}).
%% These models are useful to public health officials in both 
%% predicting the spread of an epidemic and in developing measures
%% to contain the epidemic.
Our results hold under the well known
Susceptible-Infected-Recovered (or SIR) model 
initially proposed in \cite{kermack+m:SIR}, 
as well as under a broad class of 
related models, such as SI, SIS and probabilistic threshold 
(complex contagion) models. 
We note that our results hold for networked populations, and not
in homogeneous mixing of populations, where the SIR and similar
models were originally studied.
%%A precise description of the SIR model on networks is
%%presented in Section~\ref{sec:sir_dyn_system}.
%%As observed by a number of researchers 
%%(e.g. \cite{marathe:cacm13,SD-2012,Bansal_etal_2007,eubank:nature04,
%%newman:spread02,Cauchemez15022011}), contact patterns arising
%%in practice do not satisfy such homogeneity assumptions.
%%%%%
%%%%% Up to this, the material is the same as that in sec_00.tex.
%%%%%
A precise description of the SIR model on networks is
presented in Section~\ref{sec:sir_dyn_system}.
As observed by a number of researchers 
(e.g. \cite{marathe:cacm13,SD-2012,Bansal_etal_2007,eubank:nature04,
newman:spread02,Cauchemez15022011}), contact patterns arising
in practice do not satisfy such homogeneity assumptions.

\medskip

\noindent
\textbf{Our contributions.}
In this paper, we introduce a number of short-term forecasting
problems under the SIR model. 
We show that several such forecasting problems are computationally
intractable even when the specified time value $t$ is as small as 2.
Our results also hold for realistic social networks
(e.g. power-law networks, networks with low diameter).
We demonstrate that our complexity results are tight by 
showing that the problems
are efficiently solvable for appropriate lower values of $t$.
Further, we provide efficient randomized approximation 
algorithms for some forecasting problems.
We also show how our results can be extended to the 
forecasting measures introduced in \cite{TC+2016} and to
three other epidemic models.  
The SIR model has been studied extensively. Many results
regarding the conditions under which the disease will spread rapidly and
techniques to control the spread have been reported (e.g. 
see \cite{epi-Hethcote,Britton-2009,Brauer-2008,marathe:cacm13} and the
references cited therein).
To our knowledge, only a few references have addressed 
computational complexity issues for problems under the SIR model.
We will provide a summary of these results in Section~\ref{sse:related_work}.

\medskip

\noindent
\textbf{Implications.}

\medskip\smallskip

\noindent
\textbf{1.~Unpredictability of epidemic dynamics sets in at a very early time.} \ 
Our results show that from a computational complexity point of view, 
epidemic dynamics becomes unpredictable even when the 
time horizon is as small as 2.
This is in contrast to previous complexity results on the unpredictability
of dynamical systems which needed an exponential number of time steps for
deterministic systems (e.g., PSPACE-hardness of reachability problems
for dynamical systems shown in \cite{BH+06}) and a polynomial number of time
steps for probabilistic systems (e.g., PSPACE-hardness of simulating 
quadratic dynamical systems shown in \cite{Arora-etal-1994}).

\medskip
\noindent
\textbf{2.~Unpredictability holds for both macroscopic and individual properties.} \ 
Our computational intractability results hold for predicting 
macroscopic properties (e.g., finding the probability that there will be at least
$q$ new infections at any time $t \geq 2$) as well as individual properties
(e.g., finding the probability that a specific node is infected at time $t = 3$
under the probabilistic threshold model) over a short time horizon.

\medskip
\noindent
\textbf{3.~Unpredictability holds even when prior knowledge regarding
the system behavior is available.} \ 
Most of our computational intractability results hold for any time value $t \geq 2$.
An examination of the proofs of these results shows that
from a worst-case stand point, the unpredictability results do not change
even if the behavior of the system is known for most previous time steps. 

\medskip
\noindent
\textbf{4.~The unpredictability results are pervasive.} \
Our unpredictability results hold for a variety of problems 
and contagion propagation models. 
The problem variants include predicting the number of new infections
at a certain time, the total number of infections up to a certain time,
the time and size of the peak number of new infections, etc.
The contagion models studied include SIR, SI, SIS and probabilistic
threshold. 
Furthermore, the complexity results hold for
several classes of networks such as power-law networks and small world
networks (e.g., complete graphs whose diameter is 1).

\medskip
\noindent
\textbf{5.~One random parameter is adequate for unpredictability.} \ 
Our complexity results rely on exactly one source of randomness, namely
the transmission probability. 
The results hold even when the remaining model parameters 
(e.g., the network structure, 
incubation time, infectious period, initially infected node)
are deterministic and do not vary over time.  

\medskip
\noindent
\textbf{6.~Provable predictability results for certain special cases.} \
For certain special forecasting problems, our results provide 
provably good approximations when the time horizon is fixed.
Such special cases include computing the expected number of infections,  
the probability of a node being infected by a certain time and the
probability that the total number of infections exceeds a specified 
count by a certain time.  
%%The last results are
%%interesting and are based on using an importance sampling technique.

\smallskip
In summary, our results show that one cannot expect to find
provably correct and efficient algorithms for predicting 
epidemic dynamics on networks.  
The results do not rule out heuristics that work well in practice;
they suggest that good algorithms that work across a range of inputs 
should exploit special properties of problem instances.

\iffalse
%%%%%%%%%%%%%%%%%%%%%%%%%%%%%%%%%%%%%%%%%%%%%%%%%%%%%%%
Our results indicate that, in general, the basic forecasting 
problems that arise in the context of contagion dynamics cannot 
be solved efficiently even for a time horizon as small as 2,
unless standard assumptions of complexity
theory turn out to be false.
Further, our results show that one cannot even hope for efficient
algorithms that produce provably approximate estimates of the 
actual probability values. 
In addition, the fact that our computational intractability results hold
realistic social networks (e.g. small world networks, power law networks)
points out that, in general, algorithms that
produce exact (or provably approximate) forecasts are unlikely to
run in a reasonable amount of time on large 
social networks that arise in practice.

\medskip
Other researchers 
(e.g. \cite{Drake-2005,Drake-2006, Martin_etal_2016,May-2006,beckage:ecosphere11})
have attributed the difficulty of making accurate 
forecasts to the uncertainty in the values of model parameters.
While parameter uncertainty is a source of difficulty,
our results point out that 
there is a fundamental computational limitation on
accurate (or approximate) forecasting, even when the 
model parameters are \emph{known exactly}. 
Thus, our results supplement the reasons discussed in 
\cite{Drake-2005,Drake-2006, Martin_etal_2016,May-2006,beckage:ecosphere11}
by bringing out another intrinsic difficulty 
of predicting parameters of contagion dynamics even when all the model
parameters are assumed to be insensitive to changes
in the environment.

\medskip
Our randomized approximation schemes show that for forecasting
problems involving small subpopulations and a small time horizon,
one can indeed obtain provably good approximate solutions efficiently. 
These results suggest that restricting to a suitable 
subpopulation is a useful step in obtaining reliable and nearly optimal
forecasts related to contagion dynamics.
%%%%%%%%%%%%%%%%%%%%%%%%%%%%%%%%%%%%%%%%%%%%%%%%%%%%%%%
\fi

\medskip

\noindent
\textbf{Organization.}~
The remainder of this paper is organized as follows.
In Section~\ref{sec:sir_dyn_system}, we provide precise definitions
of the probabilistic process underlying the SIR model using the
terminology of discrete probabilistic dynamical systems \cite{BH+2011}.
In Section~\ref{sec:prob_form}, we use this model
to develop rigorous formulations of the various short-term 
forecasting problems, summarize our contributions and discuss related work. 
In Section~\ref{sec:general_results}, we present complexity
results for various forecasting problems for general graphs.
Section~\ref{sec:realistic} extends these results to realistic social
networks (e.g. networks with small diameter and large average clustering
coefficient and networks with power-law degree distributions).
Section~\ref{sec:poly_versions} complements our hardness results by
presenting some efficiently solvable and efficiently approximable 
versions of forecasting problems.
Section~\ref{sec:extensions} shows how our results can be extended to 
other forecasting measures introduced in \cite{TC+2016} and to three
other epidemic models.
Conclusions and suggestions for future work are provided 
in Section~\ref{sec:concl}.
