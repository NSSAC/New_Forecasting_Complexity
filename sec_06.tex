\section{Efficiently Solvable or Approximable Versions of 
Forecasting Problems}
\label{sec:poly_versions}

\subsection{Overview}

This section presents efficient algorithms and approximations for some
short-term forecasting problems.
Results presented in this section include the following.

\begin{description}
\item{(a)}
Section~\ref{sse:poly} presents a polynomial
algorithm for a general version of the one step forecasting 
problem.
Efficient algorithms for \OneNewInfs, \OneTotInfs{}, \OneVuls{} and \OneTotVuls{}
for any set $S$ follow directly from this general result.
\item{(b)}
Section~\ref{sss:two_step_poly} presents efficient algorithms for 
\TwoNewInfs, \TwoTotInfs{}, \TwoVuls{} and \TwoTotVuls{} 
when the set $S$ is of size $O(\log{n})$, where
$n$ is the number of nodes in the underlying graph.
The efficient solvability of these problems in
conjunction with linearity of expectation \cite{MU-2005}
implies that the expected number
of newly infected nodes for 
$t = 1$ and $t = 2$ can also be computed efficiently.
\item{(c)}
Section~\ref{sse:fpras_expected_inf} presents efficient 
randomized approximation
algorithms for the \tTotVuls{} and \tTotInfs{} problems, 
as well as  for the problem of computing
higher order moments of the total number of infections, by
time $t$, when $t$ is \emph{fixed}.
\end{description}

\subsection{Efficient Algorithms for Restricted Versions of
Forecasting Problems}
\label{sse:poly}
 
\subsubsection{Constrained Configuration Problems: Definitions}
\label{sss:constrained_config}

The notion of an $r$-\emph{symmetric} function with Boolean
inputs was defined in \cite{BH+2007}.
We now extend this definition to allow inputs
from the set \bset.
%% \{\sstate, \istate, \rstate\}.
Throughout this section, we use $\mathcal{B}$ to denote the
set $\{0,1\}$ of Boolean values.

\begin{definition} \label{def:r_symm}
Consider an $n$-input predicate $\lambda: \bset^{n} \longrightarrow \mathcal{B}$.
Predicate $\lambda$ is \textbf{symmetric} if its value  
depends only on how many of the inputs have values \sstate,
\istate{} and \rstate{} respectively.
(In particular, the value of $\lambda$ does \emph{not}
depend on the order in which the input values are specified.)   

Predicate $\lambda: \bset^{n} \longrightarrow \mathcal{B}$
is \textbf{$r$-symmetric} if the set of $n$ inputs to $\lambda$ can be 
partitioned into at most $r$ classes  
such that the value of $\lambda$ depends only on how many of the   
inputs in each of the $r$ classes have the values
\sstate, \istate{} ~and~ \rstate.
\end{definition}
We note that a symmetric predicate is 1-symmetric.

A \textbf{configuration constraint} $\lambda$ on a SIR system \cals{} is a
constraint, that is, a predicate, on a configuration of \cals{}.  
If the underlying graph $G(V,E)$ contains $n$ nodes,
then $\lambda$ is a predicate on the $n$ node values in the configuration. 
An \textbf{$r$-symmetric configuration constraint} 
$\lambda$ is a configuration constraint that is $r$-symmetric.
Each class of inputs
$\nu$ to an $r$-symmetric configuration constraint $\lambda$ is a
subset of nodes of $G$.  
We now present some examples of $r$-symmetric predicates.

\begin{example}\label{ex:r_symm}
Let \cals{} be an SIR system with $n$ nodes.
\begin{description}
\item{(a)} 
Consider the configuration constraint $\lambda_{\textsc{NewInf},S}$,
which, for a specified integer value $q$,
is true for configuration \calc{} of system \cals{} ~iff~
there are at least $q$ members of node set $S$ whose state in \calc{} is $\istate$.
This configuration constraint is 2-symmetric,
since its inputs can be partitioned into two classes: one class consisting of 
the node set $S$ and the other class consisting of all the remaining nodes.
(If \calc{} represents the configuration of \cals{} at time $t$,
then $\lambda_{\textsc{NewInf},S}(\calc)$ is true iff there are at least 
$q$ new infections among the nodes in $S$ at time $t$.)

\item{(b)} 
Consider the configuration constraint $\lambda_{\textsc{TotInf},S}$,
which, for a specified integer value $q$,
is true for configuration \calc{} of system \cals{} ~iff~
there are at least $q$ members of node set $S$ 
whose state in \calc{} is either \istate{} or \rstate{}.
This configuration constraint is also 2-symmetric.
(If \calc{} represents the configuration of \cals{} at time $t$,
then $\lambda_{\textsc{TotInf}}(\calc)$ 
is true iff there are at least $q$ infections among the nodes in $S$ by time $t$)
%, assuming 
%that no node of \cals{} was in state \rstate{} at time 0.)

\item{(c)} 
Consider the configuration constraint $\lambda_{\textsc{Vul},S}$,
which is true for configuration \calc{} of \cals{} iff
all members of node set $S$ are in state \istate{} in \calc{}.
This configuration constraint is also 2-symmetric.

\item{(d)} 
Consider the configuration constraint $\lambda_{\textsc{TotVul},S}$,
which is true for configuration \calc{} of \cals{} iff 
all members of node set $S$ are in either state \istate{} or state \rstate{}.
This configuration constraint is also 2-symmetric.

\item{(e)} 
Suppose the nodes of~ \cals{} are partitioned into
two subsets $V_1$ and $V_2$, where $V_1$ and $V_2$ are respectively the sets of nodes
corresponding to children (say, age $< 18$) and 
adults (i.e., age $\geq 18$) in the population.
For a given configuration \calc,
let $N_i$ be the the number of nodes in $V_i$ 
that are in state \istate{} in \calc, $i = 1, 2$.
Consider the configuration constraint $\lambda_{\textsc{ChildAdult}}$,
which, for a specified integer value $q$,
is true for configuration \calc{} of \cals{} iff
$|N_1 - N_2| \leq q$.
(If \calc{} represents the configuration of \cals{} at time $t$,
then $\lambda_{\textsc{ChildAdult}}(\calc)$ is true iff the difference between number of infected children
and the number of infected adults at time $t$ is at most $q$.) 
Constraint  $\lambda_{\textsc{ChildAdult}}$ is 2-symmetric,
since its inputs can be partitioned into two classes,
namely $V_1$ and $V_2$.   
\hfill $\Box$
\end{description}
\end{example}

An $n$-input symmetric configuration constraint $\lambda$ can be specified using
a table with $(n+1)^2$ rows, with the $i^{\rm{th}}$ 
row specifying the value of $\lambda$ for 
the pair $(x_i, y_i)$, where $x_i$ and $y_i$ are
the number of inputs with values \sstate{} and \istate{}
respectively and $0 \leq x_i, y_i \leq n$.
(Since the configuration constraint has $n$ inputs, the number of inputs
in state \rstate{} is equal to $n-x_i-y_i$.) 
In general, an $r$-symmetric configuration constraint with $n$ inputs
can be specified using a table with at most $(n+1)^{2r}$ rows,
where each row represents the number of inputs 
having values \sstate{} and \istate{}
in each of the $r$ classes of $\lambda$.

In the remainder of this section, when we use $r$-symmetric configuration constraints,
we assume that $r$ is a \emph{fixed} integer which is independent
of the problem instance.   
We also assume that each $r$-symmetric configuration constraint is specified
through a table as described above, or that
the table can be constructed in polynomial
time from the given system \cals{}.

\newcommand{\tcpp}{\mbox{$t$}-\textsc{Cpp}}

\begin{definition} \label{def:t_config}
The \textbf{$t$-configuration probability problem} (\tcpp)
for an SIR system is defined as follows:
given an SIR system \cals{}, an initial configuration \cali{}, a non-negative
integer $t$, and a configuration constraint $\lambda$, compute
the probability that  after exactly $t$ steps, the given system
\cals{}, starting in configuration \cali{}, reaches a configuration
satisfying constraint $\lambda$.
\end{definition}

The {\tNewInfs}, {\tTotInfs}, {\tVuls} and {\tTotVuls} problems are all 2-symmetric \tcpp{}s,
since they can be expressed using configuration constraints  
$\lambda_{\textsc{NewInf},S}$, $\lambda_{\textsc{TotInf},S}$, $\lambda_{\textsc{Vul},S}$
and $\lambda_{\textsc{TotVul},S}$, as discussed in Example~\ref{ex:r_symm}.
%% A more complex 2-symmetric configuration constraint is that the fraction
%% of children who are infected by time $t$ is more than twice the
%% fraction of adults who are infected.



\subsubsection{One Step Constrained Configuration Problems}
\label{sss:one_step_poly}
  
We now provide some additional definitions that will enable
us to develop an efficient algorithm for a general class
of constrained configuration problems that model one step
forecasting problems.
(Recall from our results in 
Sections~\ref{sec:general_results} and \ref{sec:realistic}
that, in general, forecasting problems involving two or more time steps
are computationally intractable.)

Consider a given SIR system $\cals{}$, with node set $V$ and 
initial configuration $\cali{}$. 
Note that all nodes that are in state $\istate$ or $\rstate$ in
configuration $\cali{}$ are in state $\rstate$ in the configuration
reached after one step.  A node that is in state $\sstate$  in
$\cali{}$ might be in state $\sstate$ or state $\istate$ in the
configuration reached after one step.  
Given initial configuration $\cali{}$,
we define a configuration $\calc{}$ of system $\cals{}$
to be $\cali{}$--\textbf{valid}
if every node that is in state $\istate$ in  $\calc{}$
is in state $\sstate$ in  $\cali{}$.
Note that every one-step configuration reached from $\cali{}$ is $\cali{}$--valid.

\newcommand{\nus}{\mbox{$\nu_{\mathbb{S}}$}}
\newcommand{\nui}{\mbox{$\nu_{\mathbb{I}}$}}
\newcommand{\nur}{\mbox{$\nu_{\mathbb{R}}$}}

Consider an $r$-symmetric configuration constraint $\lambda$.  
Let $\kappa_{\lambda}$ denote the set of at most $r$ classes of inputs
to constraint $\lambda$.  
For each class $\nu  \in \kappa_{\lambda}$,
let \nus{},~ \nui{} and \nur{} 
denote, respectively, the members of $\nu$ that are in state 
\sstate, \istate{} and \rstate{} in $\cali$. 

For an $r$-symmetric configuration constraint $\lambda$
and an initial configuration $\cali{}$,
a $(\lambda,\cali{})$--\textbf{signature} is a function 
$g : \kappa_{\lambda} \longrightarrow \mathbb{N}$~, 
such that for each class $\nu$  in
$\kappa_{\lambda}$, the value of $g(\nu)$ does not exceed $|\nus{}|$.
We let $\Gamma_{\mathcal{I}}^{\lambda}$ denote the
set of all possible $(\lambda,\mathcal{I})$--signatures.  
Note that when
$r$ is a constant, $|\Gamma_{\mathcal{I}}^{\lambda}|$ is bounded by
$(n+1)^r$, a polynomial function of the number of nodes in $G$.

\medskip
For a given $r$-symmetric configuration constraint $\lambda$
and initial configuration $\cali{}$,
we define a function,
denoted by $sig_\lambda$, 
which maps each $\cali{}$--valid configuration $\calc{}$
into a $(\lambda,\cali{})$--signature, as follows.
For a given $\cali{}$--valid configuration $\calc{}$,
$sig_\lambda(\calc)$ is the $(\lambda,\cali{})$--signature
such that for each $\nu  \in \kappa_{\lambda}$,
$sig_\lambda(\calc)(\nu)$
equals the number of nodes in $\nu$
that have value $\istate$ in $\calc$.
We refer to $sig_\lambda(\calc)$ as 
the $(\lambda,\cali{})$--signature \textbf{associated} 
with $\calc$.
%The concept of an associated $(\lambda,Y,\cali{})$--signature will be used in our 
%algorithm for the 1-configuration probability problem discussed below. 

Let $\calc$ be a configuration reached from configuration ${\cali}$ after {\bf one} step.
For a class  $\nu  \in \kappa_{\lambda}$,
let $x$ denote the value of $sig_\lambda(\calc)(\nu)$.
Then $\calc_{\nu}$ has $x$ nodes in state \istate{},
$|\nus{}| - x$ nodes in state $\sstate$,
and $|\nu| - |\nus|$ nodes in state $\rstate$.
Since this holds for every class in  $\kappa_{\lambda}$,
the value of predicate $\lambda$ on $C$ is determined by 
the associated $(\lambda,\cali{})$--signature $sig_\lambda(\calc)$.
We refer to $\widehat{\lambda}_{\mathcal{I}}$ as the predicate on 
$(\lambda,\cali{})$--signatures that corresponds to $\lambda$. 
More precisely, for $(\lambda,\cali{})$--signature $g$,
$\widehat{\lambda}_{\mathcal{I}}(g)$ is true iff
$\lambda$ is true for configurations for which for each class $\nu$,
the number of nodes in state $\istate{}$ is $g(\nu)$,
the number of nodes in state $\sstate{}$ is $|\nus{}| - g(\nu)$,
and the number of nodes in state $\rstate$ is $|\nu| - |\nus|$.
Note that given a pair consisting of 
the table specifying $r$-symmetric predicate $\lambda$
and initial configuration ${\cali}$,
a table specifying function $\widehat{\lambda}_{\mathcal{I}}$ 
can be constructed in linear time.
Thus, without loss of generality, we can henceforth assume that 
such a specification of function $\widehat{\lambda}_{\mathcal{I}}$
is available.

As described in the next result, a compiler can be constructed
that for any fixed $r$, runs in polynomial time,
and solves the $1$-configuration probability problem
for {\bf any} $r$-symmetric configuration constraint.


\begin{theorem}  \label{th:one-step-forecasting}
For any fixed $r$, 
there is a polynomial time algorithm for 
the $1$-configuration probability problem,
given an $r$-symmetric configuration constraint.
\end{theorem}   

\noindent  
\textbf{Proof:}~   
Suppose we are given SIR system $\cals{}$, 
initial configuration \cali{},  and $r$-symmetric configuration constraint $\lambda$.
For each node $v$ of the underlying graph of \cals{},
let $p(v)$ be the probability that node $v$ is in
state $\istate$ at time 1, given that the configuration of \cals{}
at time 0 is \cali.

Recall that for each class $\nu$ of $\kappa_{\lambda}$,
the members of $\nu$ that are in state 
\sstate, \istate{} and \rstate{} in $\cali$
are denoted as \nus{},~ \nui{} and \nur{} . 
Note that at most $|\nus|$ member of $\nu$ 
can be in state \istate{}  at time 1.
Let the members of \nus{} be $\{u^{\nu}_1, \ldots, u^{\nu}_{|\nus|}\}$.  
For $0 \leq j \leq |\nus|$, 
let $R^{\nu}_j = \{u^{\nu}_1, \ldots, u^{\nu}_j\}$.
Note that $R^{\nu}_0$ has no members.
%and let $n^{\nu}_j$ denote the number of nodes of $R^{\nu}_j$ 
%that are in state \sstate{} in $\cali$. 
Let $P_j^{\nu}$, $0 \leq j \leq |\nus|$, 
be a function that assigns to each  value $a$, where $0 \leq a \leq j$,
the probability that exactly $a$ members of  $R^{\nu}_j$ 
are in state \istate{}  at time 1.

Note that the only dependence of these functions $P_j^\nu$
on predicate $\lambda$ is a specification of which nodes of system \cals{} 
are in each of the classes of $\lambda$.
Thus, the computation of these functions $P_j^\lambda$ 
can be done in an identical manner for
all $\lambda$ with the same node classes.

For each class $\nu$, we now show how to calculate $P_j^\nu$, $0 \leq j \leq  |\nus|$,
using dynamic programming.
First, note that since $R_0^\nu$ contains no nodes,
the domain of $P_0^\nu$ consists of the single value $a = 0$.
So, $P_0^\nu(0) = 1$.

After having computed $P_j^\nu$ for some $j$,
we can compute$P_{j+1}^\nu$ as follows.

\begin{enumerate}
\item 
$P_{j+1}^\nu(0) ~=~ P_j^\nu(0) (1 - p(u^{\nu}_{j+1}))$.  

\item  
$P_{j+1}^\lambda(j+1) ~=~ P_j^\lambda(j) p(u^{\nu}_{j+1})$.  

\item For $0 < a \leq j$,
$P_{j+1}^\lambda(a) ~=~ P_j^\lambda(a) (1 - p(u^{\nu}_{j+1})) 
                        + P_j^\lambda(a-1) p(u^{\nu}_{j+1})$.
\end{enumerate}

Let $P^\lambda$ be the function that assigns to each signature $g$ in
$\Gamma_{\mathcal{I}}^{\lambda}$~ the probability that $g$ is the
$(\lambda,\cali{})$--signature associated with the configuration
reached after one step.
Then, $P^\lambda$ can be computed as follows.
For each $g$ in $\Gamma_{\mathcal{I}}^{\lambda}$,

\[ \displaystyle{
P^\lambda(g) ~=~  \prod_{\nu \in \kappa_{\lambda}} P_{|\nus|}^\nu}(g(\nu))
\]

Note that the computation of function $P^\lambda$ can be done in an
identical manner for all  $\lambda$ with the same node classes.

Note further that the set of all configurations reached at at time 1
is partitioned into blocks by their signatures:
all configurations with the same associated 
$(\lambda,\cali{})$--signature 
are in the same block of this partition.
Moreover, all configurations reached at time 1
that have the same signature 
have the same truth value with respect to constraint $\lambda$,
as specified by predicate $\widehat{\lambda}_{\mathcal{I}}$.
Let $\Psi(\lambda, \cali)$ denote the set of all
signatures $g \in \Gamma_{\mathcal{I}}^{\lambda}$~
such that $\widehat{\lambda}_{\mathcal{I}}(g)$ is true.
Thus, given $P^\lambda$, we can solve the 
1-configuration probability problem for $\lambda$
by summing $P^{\lambda}(g)$ over all 
$(\lambda,\cali{})$--signatures $g \in \Psi(\lambda, \cali)$;
that is, the solution is given by 
$\displaystyle{\sum_{g \in \Psi(\lambda, \mathcal{I})} P^{\lambda}(g)}$.
Note that given a table specifying predicate $\widehat{\lambda}_{\mathcal{I}}$,
this summation can be carried out automatically.

\medskip
We now consider the running time of the above algorithm.  
For each node $v$,
$p(v)$ can be computed in time proportional to the number of neighbors
of node $v$, so the time required to compute $p(v)$ for all nodes
$v$ is linear in the size of $G$.  For fixed $r$, the time to
compute functions $P^\lambda$ is
polynomial in $n$.  (If $r = 1$, it is quadratic in $n$.)
\QED
  
\begin{corollary}\label{cor:gen_easy_time_1}
The following problems can be solved in time linear in the size of the underlying graph:
(a) \OneNewInfs, (b) \OneTotInfs{}, (c) \OneVuls{} ~and \\ (d) \OneTotVuls{}.
\end{corollary}

\noindent
\textbf{Proof:}~ Algorithms for these four problems follow directly 
from the general result of Theorem~\ref{th:one-step-forecasting},
but special properties of their predicates enable the algorithms to be made more efficient,
as explained below.

Each of the four above problems corresponds to a 2-symmetric constraint,
where one node class of the constraint
consists of the members of node set $S$. 
The other class consists of the remaining nodes;
we denote this class as $S'$. 
We recall that $S_{\sstate}$ denotes the members of class $S$
that are in state ${\sstate}$ in configuration $\cali{}$,
and $S'_{\sstate}$ denotes the members of class $S'$
that are in state ${\sstate}$ in configuration $\cali{}$.
For any 2-symmetric constraint $\lambda$ involving these two node classes,
a $(\lambda,\cali{})$--signature is a pair of
integers $(a, b)$, where $0 \leq a \leq |S_{\sstate}|$ and $0 \leq b \leq |S'_{\sstate}|$. 

Each of the four above  constraints $\lambda$ has the property that
the value of $\widehat{\lambda}_{\mathcal{I}}((a,b))$
is independent of $b$, so the state of the nodes in $S'$ in a configuration $C$ is irrelevant.
This property enables us to simplify the algorithm by processing only the members of set $S$,
thereby implicitly compressing all 2-dimensional signatures with the same value of $a$
into a single value, as described below.
%Thus the set of compressed signatures is the set of values $a$ such that $0 \leq a \leq |S_{\sstate}|$.

%Let $n' = | S |$.
For each \cali{}--valid configuration,
the associated signature 
can be represented as a single integer $a$ that
equals the number of nodes of set $S_{\sstate}$  
that are in state \istate{} at time 1.
(Note that $0 \leq a \leq |S_{\sstate}|$.)
We call this value $a$ a {\bf compressed} signature.
Suppose that we construct function $P_{|S_{\sstate}|}^S$,
as described in the proof of Theorem~\ref{th:one-step-forecasting}.
Note that $P_{|S_{\sstate}|}^S(a)$
%Let $P^{\lambda}$, $0 \leq j \leq n'$, 
assigns to each compressed
signature value $a$
the probability that $a$ is the
associated $(\lambda,\cali{})$ compressed signature of the configuration
reached after one step.

For each of the four predicates $\lambda$ under consideration,
the set of compressed signatures is the set of values 
$a$ such that $0 \leq a \leq |S_{\sstate}|$.
For each $\lambda$, let $\widehat{\lambda}_{\mathcal{I}}$ denote the corresponding
predicate on compressed signatures.
Let $\Psi(\lambda, \cali)$ denote the set of compressed signatures $a$
%$ \Gamma_{V,\,\mathcal{I}}^{\lambda}$~
such that $\widehat{\lambda}_{\mathcal{I}}(a)$ is true.
Thus, given $P_{|S_{\sstate}|}^S$, we can solve the 
1-configuration probability problem for $\lambda$
by summing $P_{|S_{\sstate}|}^S(a)$ over all 
values $a$ in $\Psi(\lambda, \cali)$;
that is, the solution is given by 
$\displaystyle{\sum_{a \in \Psi(\lambda, \mathcal{I})} P_{|S_{\sstate}|}^S(a)}$.
For each of the four problems $P_{|S_{\sstate}|}^S$ is the same, 
but $\widehat{\lambda}_{\mathcal{I}}$ is different.

%Note that given a table specifying predicate $\widehat{\lambda}_{\mathcal{I}}$,
%this summation can be carried out automatically.

First consider the \OneNewInfs{} problem, where we want to compute 
the probability that at least $q$ members of set $S$ are infected at $t = 1$.
The predicate $(\widehat{\lambda}_{{\textsc{NewInf},S}})_{{\mathcal{I}}}(a)$ 
is true iff $a \geq q$.
Thus, the answer to the \OneNewInfs{} problem is given by
$\displaystyle{\sum_{a=q}
                    ^{|S_{\sstate}|}
                    P_{|S_{\sstate}|}^S(a)}$.



Now consider the \OneTotInfs{} problem,
where we want to compute 
the probability that at least $q$ members of set $S$ are infected by $t = 1$,
and so can have been infected before time 1.
The only difference here is that the
predicate $\widehat{\lambda}_{{\textsc{TotInf},S}_{\mathcal{I}}}(a)$ is 
true iff $q \leq a +  |S_{\istate}| + |S_{\rstate}|$.
Thus, the answer is given by 
$\displaystyle{\sum_{a=q - |S_{\istate}| - |S_{\rstate}|}
                    ^{|S_{\sstate}|}
                    P_{|S_{\sstate}|}^S(a)}$.


Next, consider the \OneVuls{} problem.
Here, predicate $\widehat{\lambda}_{{\textsc{OneVul},S}_{\mathcal{I}}}(a)$ is 
true iff $a = |S|$.
If $|S_{\sstate}| < |S|$, then $\Psi(\lambda, \cali)$ is empty,
so the sum contains no terms, and evaluates to 0.
If $|S_{\sstate}| = |S|$, then the sum contains only term, 
namely 
$P_{|S_{\sstate}|}^S(|S|)$.

Next, consider the 
\OneTotVuls{} problem.
Here, predicate $\widehat{\lambda}_{{\textsc{OneTotVul},S}_{\mathcal{I}}}(a)$ is 
true iff $a +  |S_{\istate}| + |S_{\rstate}| = |S|$.
Thus, the sum contains only one term, namely
$P_{|S_{\sstate}|}^S(|S| - |S_{\istate}| - |S_{\rstate}|)$.
\QED

%% Note that \OneVul{} can be solved more simply by just computing $p_i$ 
%% for the node $v_i$ whose vulnerability is to be computed.



\subsubsection{Two Step Constrained Configuration Problems}
\label{sss:two_step_poly}
  
%\newcommand{\calsp}{\mbox{${\cal S}'$}}  
\newcommand{\calspp}{\mbox{${\cal S}''$}} 


We now present efficient
algorithms for the \TwoNewInfs{},~ \TwoTotInfs{},\\
\TwoVuls{} ~and~ \TwoTotVuls{} 
problems for the case when node set $S$ is relatively small.

For pairwise disjoint sets of nodes $S_1$, $S_2$, and $S_3$,
let $\psi(S_1, S_2, S_3)$
denote the probability that all members of $S_1$ are infected at time 1,
all members of $S_2$ are infected at time 2,
and no member of $S_3$ is infected by time 2.

\begin{theorem}\label{thm:generalized_vulnerability_time_2}
For pairwise disjoint sets of nodes $S_1$, $S_2$, and $S_3$,
where  set $S_2$ is of size $O(\log{n})$,
the probability value $\psi(S_1, S_2, S_3)$ can be computed in polynomial time.
\end{theorem}
\noindent
\textbf{Proof:}~  
If $S_1 \cup S_2 \cup S_3$ contains an index node, that is,
a node which is infected at time 0, clearly $\psi(S_1, S_2, S_3) = 0$.
Thus, without loss of generality we can assume that 
no member of the three given node sets is an index node.

Our algorithm begins by modifying the given SIR system $\cals{}$ and initial configuration, 
if necessary, by combining all the index
nodes into a single new index node, creating a modified SIR system
$\calsp{}$.  (If there is only one index node, then $\calsp{}$ is
the same as $\cals{}$.) Let us call this new index node $u_0$.
Consider any node $v$ such that $v$ is not an index node for $\cals{}$ and  
there is at least one edge between any index node 
for $\cals{}$ and $v$.
Let $E_v = \{e_1, e_2, \ldots, e_r\}$ denote the set of all
edges of \cals{} between $v$ and the index nodes of \cals.
In \calsp, all the edges in $E_v$ are replaced by a 
single new edge $e' = \{u_0, v\}$. 
The transmission probability $p_{e'}$ of $e'$ is the
probability of transmission via at least one of the edges being
replaced; in other words,
\[ \displaystyle{
p_{e'} ~=~ 1- \prod_{i=1}^{r} (1-p_{e_i})}
\]
Note that the probability that any given node is infected at any given time
is the same in $\cals{}$ and $\calsp{}$.

Let $q_1$ denote the probability in system $\calsp{}$
that all members of $S_1$ are infected at time 1.
If $S_1$ contains a node that is not connected to $u_0$, then $q_1 = 0$. 
Otherwise,
$q_1 = \displaystyle{
\prod_{v \in S_1} p_{\{u_0,v\}} }$.

Let $q_2$ denote the probability in system $\calsp{}$
that no member of $S_2 \cup S_3$ is infected at time 1.
To compute $q_2$,
let $S_4$ denote the set of members $v$ of $S_2 \cup S_3$ 
such that system $\calsp{}$ contains the edge $\{u_0, v\}$.
Then, $q_2 = \displaystyle{
\prod_{v \in S_4} (1 - p_{\{u_0,v\}}) }$.

Let $\calspp$ be the SIR system obtained from system $\calsp$ by deleting 
all the edges from node $u_0$ to members of $S_4$,
and changing the probability of any edge between $u_0$ and a member of $S_1$ 
to probability value 1.
Then $\psi(S_1, S_2, S_3)$ for  SIR system $\calsp$ 
is the product of the three factors $q_1$, $q_2$, and
the value of $\psi(\emptyset, S_2, S_3)$ for  SIR system $\calspp$.
%Wlog, we subsequently assume that $q_1 > 0$.
Note that no member of $S_2 \cup S_3$ can be infected at time 1 in $\calspp$.

We now consider computing $\psi(\emptyset, S_2, S_3)$ for  SIR system $\calspp$.
Let $U=\{u_1,\ldots,u_k\}$ denote the set of nodes such that for each
$u_i\in U$, $\calspp{}$ contains the edge $\{u_0, u_i\}$ and also contains an edge $\{u_i, v\}$
for some $v\in (S_2 \cup S_3)$. 
If $k = 0$ and $S_2$ is nonnull, then $\psi(\emptyset, S_2, S_3) = 0$. 
If $k = 0$ and $S_2$ is null, then $\psi(\emptyset, S_2, S_3) = 1$. 
Thus, without loss of generality, we can assume that $k \geq 1$.
For $1 \leq i \leq k$, let $U_i$ denote the set of $i$ nodes $\{u_1,\ldots,u_i\}$.
% let $S_3^i$ denote the members of $S_3$ that are neighbors of $u_i$.

For disjoint node sets $W_1$ and $W_2$ from $\calspp{}$, such that neither node set contains  $u_0$,
let $E(W_1, W_2)$ denote the set of edges in $\calspp{}$
such that one endpoint of the edge is in $W_1$ and the other endpoint is in $\{u_0\} \cup W_2$.
For any set of edges $E'$ from $\calspp{}$,
let $\mathcal{S}''[E']$ denote the SIR system obtained from $\calspp{}$
by retaining the edges in $E'$
and removing the edges not in $E'$.
Note that $\psi(\emptyset, S_2, S_3)$ for SIR system $\calspp$
equals the value of $\psi(\emptyset, S_2, S_3)$ for SIR system 
$\mathcal{S}''[E(U_k, S_2 \cup S_3)]$.

For $1 \leq i \leq k$ and a set of nodes $Y$ in system $\calspp{}$, 
let $N_i(Y)$ denote 
the set of nodes $v\in Y$
such that $\{u_i, v\}$ is an edge in system $\calspp{}$,
i.e., the members of $Y$ that are neighbors of $u_i$.
%If $T= \emptyset$, then $f(i+1, S') = f(i, S')$. 
%Suppose $T \neq \emptyset$. 
(Note that SIR system $\mathcal{S}''[E(\{ u_i  \}, S_2 \cup S_3)]$
is identical to SIR system $\mathcal{S}''[E(\{ u_i  \}, N_i(S_2 \cup S_3))]$.)

For each $i$ such that $1 \leq i \leq k$, $S_2' \subseteq S_2$ and $T \subseteq N_i(S_2')$,
we define $g^i_{S_2'}(T)$ to be
the value of $\psi(\phi, T, N_i(S_2' \cup S_3) - T)$ in SIR system
$\mathcal{S}''[E(\{ u_i \}, S'_2 \cup S_3)]$,
i.e., the probability that
all the nodes in $T$ are infected at time 2, 
and none of the {\em other} nodes in $N_i(S_2' \cup S_3)$ are infected at time 2.
The value of $g^i_{S_2'}(T)$ can be computed as follows.

If $T$ is nonempty, then in order for the nodes in $T$ to be infected at time 2, 
node $u_i$ must be infected at time 1.
So $g^i_{S_2'}(T)$ can be computed as:
\[
g^i_{S_2'}(T) = 
\psi(\phi, T, N_i(S_2' \cup S_3) - T) = 
p_{\{u_0,u_i\}}\prod_{v \in T} p_{\{u_i,v\}} \prod_{v \in N_i(S_2' \cup S_3) - T} (1-p_{\{u_i,v\}}).
\]
If $T$ is null, i.e., $\phi$, then $g^i_{S_2'}(\phi)$ 
can be computed as:
\[
g^i_{S_2'}(\phi) = 
\psi(\phi, \phi, N_i(S_2' \cup S_3)) = ( 1 - p_{ \{u_0,u_i\} }) \, + \,
p_{\{u_0,u_i \}} \prod_{v \in N_i(S_2' \cup S_3)} (1-p_{\{u_i,v\}}).
\]

Note that all the $g^i_{S_2'}$ functions can be computed in time 
$O(2^{|S_2|})$ times a polynomial in the size of the underlying graph.

For $1 \leq i \leq k$ and a node set $S'_2 \subseteq S_2$, let $f(i, S'_2)$ denote
$\psi(\phi, S'_2, S_3)$ in SIR system $\mathcal{S}''[E(U_i, S'_2 \cup S_3)]$,
i.e., the probability that all nodes in $S'_2$ are infected at time 2,
and no node in $S_3$ is infected by time 2.
Then, $\psi(\phi, S_2, S_3)$ 
in system \calspp{} is precisely $f(k, S_2)$.

The set of quantities $f(i, S_2')$ for all $S'_2 \subseteq S_2$,
can be computed by dynamic programming, as discussed below.

First consider each of the quantities $f(1, S'_2)$,
where $S'_2\subseteq S_2$.
If any node in $S'_2$ is not a neighbor of node $u_1$, 
then $f(1, S'_2)=0$. 
Otherwise, when every node in $S'_2$ is a neighbor of $u_1$, 
$f(1, S'_2) = g^1_{S_2'}(S_2'),$
which can be computed as described above (with $T = S_2'$).

Now, assume that $f(i, S'_2)$ has been computed for some $i<k$ and for every $S'_2 \subseteq S_2$.
We then consider $f(i+1, S'_2)$ for each $S'_2 \subseteq S_2$. 
We claim that
\begin{equation} \label{eq:recurrence_for_f}
f(i+1, S_2') = \sum_{T \subseteq N_{i+1}(S_2')} g^{i+1}_{S_2'}(T)f(i, S_2'-T).
\end{equation}
From this claim, it follows that when $|S_2|=O(\log{n})$,
$f(i+1, S_2')$ can be computed in polynomial time,
and consequently  that $f(k, S_2)$ can be computed in polynomial time. 
Putting all these steps together,  
when $|S_2|=O(\log{n})$,
the value of $\psi(S_1, S_2, S_3)$ for system $\cals{}$
can be computed in polynomial time.

We now prove the claim given by Equation~(\ref{eq:recurrence_for_f}). 
Note that $E(U_{i+1}, S_2' \cup S_3)$ consists of two disjoint sets of edges:
$E(U_i, S_2'  \cup S_3)$ and  $E( \{u_{i+1}\}, S_2' \cup S_3')$.
The spread of infections at times 1 and 2 through these two sets of edges is independent.
For a node in $S_3$ to be not infected by time 2 in SIR system $\mathcal{S}''[E(U_{i+1}, S'_2 \cup S_3)]$,
it must be uninfected by time 2 both in system $\mathcal{S}''[E(U_i, S'_2 \cup S_3)]$,
and in system $\mathcal{S}''[E(\{u_{i+1}\}, S'_2 \cup S_3)]$.
For a node in $S_2'$ to be infected at time 2 in SIR system $\mathcal{S}''[E(U_{i+1}, S'_2 \cup S_3)]$,
it must either be infected at time 2 in system $\mathcal{S}''[E(U_i, S'_2 \cup S_3)]$,
or be infected at time 2 in system $\mathcal{S}''[E(\{u_{i+1}\}, S'_2 \cup S_3)]$, or both.

For each $i$ such that $1 \leq i < k$, $S_2' \subseteq S_2$, $T \subseteq N_{i+1}(S_2')$,
and $T' \subseteq N_{i+1}(S_2')$, such that $T \neq T'$,
the event that in SIR system $\mathcal{S}''[E(\{ u_{i+1} \}, S'_2 \cup S_3)]$
all the nodes in $T$ are infected at time 2, 
and none of the nodes in $N_{i+1}((S_2' -T) \cup S_3)$ are infected at time 2,
is distinct from the event that 
all the nodes in $T'$ are infected at time 2, 
and none of the nodes in $N_{i+1}((S_2' -T') \cup S_3)$ are infected at time 2.
Thus, $g^{i+1}_{S_2'}(T)$ and $g^{i+1}_{S_2'}(T')$ correspond to disjoint events.
Consequently, $f(i+1, S_2')$ equals the sum over all $T \subseteq N_i(S_2')$,
of the product of $g^{i+1}_{S_2'}(T)$
and the conditional probability that 
in SIR system $\mathcal{S}''[E(U_{i+1}, S'_2 \cup S_3)]$,
all nodes in $S'_2$ are infected at time 2,
and no node in $S_3$ is infected  at time 2,
given that in SIR system $\mathcal{S}''[E(\{ u_{i+1} \}, S'_2 \cup S_3)]$,
all the nodes in $T$ are infected at time 2, 
and none of the nodes in $N_{i+1}((S_2' -T) \cup S_3)$ are infected at time 2.
This conditional probability is the probability that
in SIR system $\mathcal{S}''[E(U_i, S'_2 \cup S_3)]$,
all nodes in $S'_2 - T$ are infected at time 2,
and no node in $S_3$ is infected by time 2, i.e.,
precisely $f(i, S_2'-T)$.
\QED

\begin{corollary}\label{cor:4probs_time_2}
The following problems can be solved  in polynomial time
for sets $S$ of size $O(\log{n})$: \\
(a) \TwoNewInfs{}, (b) \TwoTotInfs{}, (c) \TwoVuls{} ~and\\
(d) \TwoTotVuls{}.
\end{corollary}
\textbf{Proof:}~  
Let $S_{\sstate}$, $S_{\istate}$, and$S_{\rstate}$ denote the members of node set $S$ 
that are in state $\sstate$,   $\istate$, and  $\rstate$, respectively,
in initial configuration $\cali{}$. 

(a) 
Let $q$ be the required number of nodes from $S$ to be in state $\istate$
at time 2.
Let $\cal T$ be the set of triples of the form $(S_1, S_2, S_3)$
where $S_1$, $S_2$, and $S_3$ are pairwise disjoint node sets such that
$S_1 \cup S_2 \cup S_3 = S_{\sstate}$ and $|S_2| \geq q$.
Then the solution to the given \TwoNewInfs{} problem instance is 
\[
\sum_{(S_1, S_2, S_3) \in {\cal T} } \psi(S_1, S_2, S_3).
\]

(b) 
Let $q$ be the required number of nodes from $S$ to be in state $\istate$ by time 2.
Let $\cal T$ be the set of triples of the form $(S_1, S_2, S_3)$
where $S_1$, $S_2$, and $S_3$ are pairwise disjoint node sets such that
$S_1 \cup S_2 \cup S_3 = S_{\sstate}$ 
and $|S_1| + |S_2| + |S_{\istate}| + |S_{\rstate} |  \geq q$.
Then the solution to the given \TwoNewInfs{} problem instance is 
\[
\sum_{(S_1, S_2, S_3) \in {\cal T} } \psi(S_1, S_2, S_3).
\]

(c)
The solution to the given \TwoVuls{} problem instance is $\psi(\phi, S, \phi)$.

(d)
Let $\cal T$ be the set of pairs of the form $(S_1, S_2)$
where $S_1$ and $S_2$ are disjoint node sets such that
$S_1 \cup S_2 = S_{\sstate}$.
Then the solution to the given \TwoTotVuls{} problem instance is 
\[
\sum_{(S_1, S_2) \in {\cal T} } \psi(S_1, S_2, \phi).
\]
\QED

Corollary~\ref{cor:4probs_time_2} can be generalized, based on the following concept.

\begin{definition}\label{def:S_determined}
A configuration constraint $\lambda$ on a SIR system $\cals{}$ is 
{\em $S$-determined}, where $S$ is  a node set, if 
for every configuration $\calc{}$ of system $\cals{}$,
whether or not $\calc{}$ satisfies $\lambda$ 
is determined by the projection of $\calc{}$ onto the node set $S$.
\end{definition}

Note that the four problems listed in Corollary~\ref{cor:4probs_time_2} 
are each based on an $S$-determined configuration constraint.

\begin{corollary}\label{cor:S_determined_time_2}
For any $S$-determined configuration constraint $\lambda$,
where $S$ is of size $O(\log{n})$,
the 2-configuration probability problem for $\lambda$
can be solved in polynomial time.
\end{corollary}
\textbf{Proof:}~  
Let $S_{\sstate}$, $S_{\istate}$, and$S_{\rstate}$ denote the subsets of node set $S$ 
that are in state $\sstate$,   $\istate$, and  $\rstate$, respectively,
in initial configuration $\cali{}$. 
Every node in $S_{\istate}\,\cup\,S_{\rstate}$ is in state $\rstate$ at time 2.
Thus, the 2-configuration probability problem for $S$-determined 
configuration constraint $\lambda$
can be re-expressed as a 2-configuration probability problem 
for a corresponding $S_{\sstate}$-determined configuration constraint $\lambda'$.

Let $\cal T$ be the set of triples of the form $(S_1, S_2, S_3)$,
where $S_1$, $S_2$, and $S_3$ are pairwise disjoint node sets such that
$S_1 \cup S_2 \cup S_3 = S_{\sstate}$.
For each triple $(S_1, S_2, S_3)$ in $\cal T{}$,
let $\calc{}'(S_1, S_2, S_3)$ be the configuration on the nodes in $S_{\sstate}$,
such that every node in $S_1$ is in state ${\rstate}$,
every node in $S_2$ is in state ${\istate}$,
and every node in $S_3$ is in state ${\sstate}$.
Then, $\psi(S_1, S_2, S_3)$ is the probability that 
the projection onto $S_{\sstate}$ of the configuration of system $\cals{}$ at time 2 
equals $\calc{}'(S_1, S_2, S_3)$.
From, Theorem \ref{thm:generalized_vulnerability_time_2},
$\psi(S_1, S_2, S_3)$ can be computed in polynomial time.

Let $\cal T'$ be the set of members $(S_1, S_2, S_3)$ of $\cal T$,
such that $\lambda'$ is satisfied by $\calc{}'(S_1, S_2, S_3)$.
Since the cardinality of $S_{\sstate}$ is $O(\log{n})$,
the cardinality of $\cal T'$ is polynomial in $n$.
Thus, the 2-configuration probability problem for $\lambda$
can be solved in polynomial time by computing the following sum:
\[
\sum_{(S_1, S_2, S_3) \in {\cal T'} } \psi(S_1, S_2, S_3).
\]
\QED

We note that when the cardinality of node set $S$ is greater than $O(\log{n})$,
but still limited, corresponding versions  of 
Theorem~\ref{thm:generalized_vulnerability_time_2}, 
Corollary~\ref{cor:4probs_time_2}, 
and Corollary~\ref{cor:S_determined_time_2} hold,
but with a running time that is exponential in the cardinality of $S$.
For instance, if $|S|$ is $O(n^{1/2})$,
the running time of the corresponding algorithms is
$O(2^{n^{1/2}})$.

Finally, from the linearity of expectation \cite{MU-2005}, 
we note that the expected number of new infections at any time $t$ equals
the sum of the vulnerabilities of all the nodes  at time $t$.
Thus, we have the following consequence of Corollary~\ref{cor:gen_easy_time_1}(c)
and  Corollary~\ref{cor:4probs_time_2}(c) .

\begin{corollary}\label{cor:expected_number_infected_time_2}
For any SIR system,
the expected number of new infections at times $t = 1$ and 
$t = 2$ can be computed in polynomial time. \QED
\end{corollary}

%\newcommand{\calspi}{\mbox{$\mathcal{S}'_i$}}
%\newcommand{\calsppi}{\mbox{$\mathcal{S}''_i$}}


\subsection{Efficient Approximations for some Forecasting Problems} 
\label{sse:fpras_expected_inf}

\subsubsection{Assumptions and Notation}

We now develop approximation algorithms for the \tTotVuls{} and \tTotInfs{} problems.
As in the previous sections, we assume that the underlying 
graph is $G=(V, E)$. 
%As explained in the proof of Proposition~\ref{pro:vulnerability_time_2},
%without loss of generality, we can assume that
%the initial configuration contains only one node in state \istate;
%we denote this node as $s$.
For convenience, the transmission probability on
edge $\{u, v\}\in E$ is denoted by $p(u,v)$. 
In some cases, we assume
a uniform probability $p(u,v) = p$ for all edges.  


%Let $X(t)$ denote
%the (random) number of nodes infected at time $t$; 
%the values of $X(t)$ for different values of $t$ represent the 
%\textbf{epicurve}.  
%Let $Y(t)=\sum_{t'\leq t} X(t)$ denote the cumulative
%number of infections up to time $t$; the values of $Y(t)$ for
%different values of $t$ represent the 
%\textbf{cumulative epicurve}, and computing $Y(t)$ is the \tTotInf{} problem.
%Let $X_v(t)$ denote the indicator for node
%$v$ getting infected within time $t$; that is, $X_v(t) = 1$ if
%$v$ gets infected within time $t$ and $X_v(t) = 0$ otherwise.
%We refer to $p_{tvul}(v, t)=\Pr[X_v(t)=1]$ as the 
%\textbf{timed vulnerability} of node $v$ at time $t$. Computing
%$p_{tvul}(v, t)$ is the \tVul{} problem.


\subsubsection{Efficient Approximations for \tTotVuls{} and \tTotInfs{}}
\label{sec:vul}

In this section, we develop efficient 
approximations for certain forecasting problems by reducing them
to the problem of computing the probability of satisfying a DNF formula (\dnfsat)
defined in Section~\ref{sss:boolean_sat}.
Theorem~\ref{thm:num_dnf_prob_approx} 
stated in that section (and proven in \cite{karp:jc85}) shows that
for any $\epsilon, \delta\in(0,1)$, there is an efficient algorithm
that produces an $(\epsilon, \delta)$-approximation for \dnfsat.
We rely on that theorem to obtain our approximation results.

%% given the probability that each variable is true.
%% The running time of the DNF approximation algorithm is the following.
%%
%%\begin{proposition}\label{pro:DNF_sampling_running_time}
%%For a given DNF formula containing $n$ variables and $m$ clauses,
%%the running time of the DNF sampling approximation algorithm is 
%%$O(\frac{1}{\epsilon^2}\,n\, m \log{(1/\delta)})$. 
%%\end{proposition}
 
Let $G(V,E)$ be the underlying graph of the given SIR system $\cals{}$.

\begin{definition}\label{def:edge_assignment}
For any edge $e$ in $E$, we let $x(e)$ denote a corresponding Boolean variable,
where $x(e) =1$ indicates that edge $e$ is transmitting,
and $x(e) =0$ indicates that $e$  is not transmitting.
We call $x(e)$ an {\bf edge variable}.
We let $X(E)$ denote the set of Boolean variables for all $e \in E$.
We define an {\bf edge assignment} to be
an assignment of a Boolean value to each variable in $X(E)$.
\end{definition}

We let ${\cal A}_E$ denote the set of all edge assignments.
% to a given edge set $E'$.
For an edge assignment $\alpha \in {\cal A}_E$, 
and Boolean value $b$, we let $\alpha^b$ denote the set of edges $e$ in $E$
such that $\alpha(x(e)) = b$.
We let $p(\alpha)$ denote the probability of the assignment, i.e.,
$p(\alpha) = \displaystyle{ \prod_{e \in \alpha^1} p_e 
 \prod_{e \in \alpha^0} (1 - p_e)   }.$      
 For a set of edge assignments $\Gamma \subseteq {\cal A}_E$,
 we let $p(\Gamma)$ denote the total probability 
 of all the edge assignments in $\Gamma$, 
 i.e., 
 $p(\Gamma) = \displaystyle{ \sum_{\alpha \in \Gamma} p(\alpha) 
                                            }.$     

Given an initial configuration $\cali{}$ of a SIR system $\cals{}$,
we use the phrase ``index node" to denote any
node which is in state \istate{} in the configuration $\cali{}$.  
We define a path in underlying graph $G(V,E)$ 
to be an {\bf infectious path} if the path is simple,
the initial node of the path is an index node
and all other nodes along the path 
are in state \sstate{} in configuration $\cali{}$.
Given a node $v$,
let $\mathcal{P}_t(v)$ denote the set of all infectious paths in $G(V, E)$
such that each path in $\mathcal{P}_t(v)$ satisfies the following
two properties: (i) its length\footnote{The length 
of a path is the number of edges in that path.} is
$t$ and (ii) its final node is $v$;
further, let $\mathcal{P}'_t(v)$ denote the set of 
all infectious paths in $G(V, E)$
whose length is at most $t$ and whose final node is $v$.
Let $\mathcal{P}_t^{max}$ denote 
the maximum value of $|\mathcal{P}'_t(v)|$ over all nodes $v$.
Also, given a subgraph $G'$ of $G$, a path in $G'$ is termed an infectious path
if the path is an infectious path in $G$.

Given an initial configuration $\cali{}$,
an edge assignment $\alpha \in {\cal A}_E$, 
and a non-negative integer $t$,
we let $\calc{}_t(\alpha)$ denote the configuration specified as follows.
Consider each node $v \in V$.
\begin{description}
\item{(a)}
If $v$ is in state \rstate{} in initial configuration $\cali{}$,
then $v$ is in state \rstate{} in $\calc{}_t(\alpha)$.
\item{(b)}
Suppose $v$ is in state \istate{} in $\cali{}$.
If $t=0$,
$v$ is in state \istate{} in $\calc{}_t(\alpha)$;
and if $t >0$,
$v$ is in state \rstate{} in $\calc{}_t(\alpha)$.
\item{(c)}
Suppose $v$ is in state \sstate{} in $\cali{}$.
If graph $G(V, \alpha^1)$ contains an infectious path to $v$ of length less than $t$,
then $v$ is in state \rstate{} in $\calc{}_t(\alpha)$;
if $G(V, \alpha^1)$ contains an infectious path to $v$ of length $t$,
but there is no such shorter path, 
then $v$ is in state \istate{} in $\calc{}_t(\alpha)$;
and if $G(V, \alpha^1)$ contains no infectious path to $v$ of length at most $t$,
then $v$ is in state \sstate{} in $\calc{}_t(\alpha)$.
\end{description}
We observe the following.

\begin{proposition}\label{pro:edge_assn_phase_space}
For a given edge assignment $\alpha \in {\cal A}_E$, the sequence of configurations
$\langle \calc{}_0(\alpha), \calc{}_1(\alpha), \cdots,$ \\
$\calc{}_t(\alpha)\rangle$
is a phase space path for SIR system $\cals{}$. 
\end{proposition}

\noindent
\textbf{Proof:}~
Consider two consecutive configurations in the above sequence, say
 $\calc{}_i(\alpha)$ and  $\calc{}_{i+1}(\alpha)$.
Consider a given node $v \in V$.

Suppose $v$ is in state \rstate{} in $\calc{}_i(\alpha)$.
Then either $v$ is in state \rstate{} in initial configuration $\cali{}$,
or $G(V, \alpha^1)$ contains an infectious path to $v$ of length less than $i$.
In either case,
$v$ is in state \rstate{} in $\calc{}_{i+1}(\alpha)$.

Suppose $v$ is in state \istate{} in $\calc{}_i(\alpha)$.
Then $G(V, \alpha^1)$ contains an infectious path to $v$ of length $i$,
so $v$ is in state \rstate{} in $\calc{}_{i+1}(\alpha)$.

Suppose $v$ is in state \sstate{} in $\calc{}_i(\alpha)$.
Then, $v$ is in state \sstate{} in $\cali{}$,
and $G(V, \alpha^1)$ contains no infectious path to $v$ of length at most $i$.
First, suppose there is a node $u$ such that 
$u$ is in state \istate{} in $\calc{}_i(\alpha)$
and $G(V, \alpha^1)$ contains the edge $(u,v)$.
Then $G(V, \alpha^1)$ contains an infectious path to $u$ of length $i$,
so $G(V, \alpha^1)$ contains an infectious path to $v$ of length $i+1$.
Thus, $v$ is in state \istate{} in $\calc{}_{i+1}(\alpha)$.
Now suppose there is no such node $u$.
Then $G(V, \alpha^1)$ contains no infectious path to $v$ of length at most $i+1$,
So, $v$ is in state \sstate{} in $\calc{}_{i+1}(\alpha)$.

Since the above holds for every node $v \in V$,
configuration $\calc{}_{i+1}(\alpha)$ is a valid successor configuration of $\calc{}_i(\alpha)$.
\QED

We now observe the following.

  \begin{proposition}\label{pro:infectious_path}
  In the phase space path corresponding to 
  an edge assignment $\alpha \in {\cal A}_E$, 
  a given node $v$ is infected by time $t$ iff
  either $v$ is in state \rstate{} in initial configuration $\cali{}$
  or graph $G(V, \alpha^1)$ contains an infectious path to $v$ 
of length at most $t$. \hfill$\Box$
 \end{proposition}

For a given configuration constraint $\lambda$ and non-negative integer $t$,
 we define $f_{\lambda,t}$ to be the Boolean function of the set of edge variables $X(E)$
 specified as follows.
 For a given edge assignment $\alpha \in {\cal A}_E$, 
 $f_{\lambda,t}(\alpha) = \lambda(\calc{}_t(\alpha))$.
 Thus, $f_{\lambda,t}$ is true for $\alpha$ iff $\lambda$ is satisfied by
 configuration $\calc{}_t(\alpha)$ in the phase space path corresponding to $\alpha$.
 We let $P(f_{\lambda,t})$ denote the probability that 
$f_{\lambda,t}$ is true,
 i.e., $P(f_{\lambda,t})$ is the sum of $p(\alpha)$ 
 over all $\alpha \in {\cal A}_E$  such that $f_{\lambda,t}(\alpha)$ is true.
 Then we observe the following.
 
 \begin{proposition}\label{pro:config_problem_edge_assn}
  For a given configuration constraint $\lambda$ and non-negative integer $t$,
 the solution to the $t$-configuration problem for $\lambda$ is equal to the value of
 $P(f_{\lambda,t})$. \hfill$\Box$
 \end{proposition}
 
Suppose that for a given configuration constraint $\lambda$ and given time $t$,
there is an efficient method to construct a relatively concise DNF formula 
for the Boolean function $f_{\lambda,t}$.
Then we can use DNF sampling to approximate the value of $P(f_{\lambda,t})$.
From Proposition~\ref{pro:config_problem_edge_assn},
this approximation is also an approximation to the solution 
to the $t$-configuration problem for $\lambda$.

We first illustrate this approach by considering the \tTotVuls{} problem.
Let node set $S=\{v_1,\ldots, v_k\}$.

Given initial configuration $\cali{}$, 
for the  \tTotVuls{} problem
we can assume without loss of generality that the problem 
instance  has been modified 
by the removal from graph $G$ of all nodes
that are in state \rstate{} in $\cali{}$,
and that no member of $S$ is an index node.

Define a DNF formula $\phi_{\textsc{TotVul}}(S,t)$ as follows:
\begin{enumerate}
\item
The set of Boolean variables is the set of edge variables $X(E)$.
%Create a Boolean variable $x(e)$ for each $e\in E$, with $\Pr[x(e)=1]=p(e)$. \\
(Thus, there are $|E|$ variables.)
\item
For each tuple of infectious paths $(P_1,\ldots,P_k)$, 
where for each $j$, $1 \leq j \leq k$,
there is  a $t' \leq t$,
such that $P_j \in \mathcal{P}_{t'}(v)$,
%with $P_j\in \displaystyle{\cup_{t'\leq t}\mathcal{P}(s, v_j, t')}$, for
%$j=1,\ldots,k$, \\ 
we construct the clause $C(P_1,\ldots,P_k)=\bigwedge_{e\in E'} x(e)$, where
$E'$ is the set of edges occurring in any of the paths $P_1,\ldots,P_k$.
\item
The formula $\phi_{\textsc{TotVul}}(S,t)$ is the disjunction of all such clauses $C(P_1,\ldots,P_k)$.
\end{enumerate}

Since an edge assignment $\alpha \in {\cal A}_E$ satisfies formula $\phi_{\textsc{TotVul}}(S,t)$ 
iff in the graph induced by $\alpha^1$, for every $v$ node in set $S$,
there is an infectious path to $v$ of length at most $t$, we have the following.

 \begin{proposition}\label{pro:formula_TotVul_correct}
 Formula $\phi_{\textsc{TotVul}}(S,t)$ represents the Boolean function 
 $f_{\lambda_{\textsc{TotVul},S},t}$.
 \end{proposition}
 \noindent
\textbf{Proof:}~
From Proposition \ref{pro:infectious_path},
for a given edge assignment $\alpha \in {\cal A}_E$, 
$f_{\lambda_{\textsc{TotVul},S},t}(\alpha)$ is true iff for every node $v$ in $S$,
$G(V, \alpha^1)$ contains an infectious path to $v$ of length at most $t$.
Thus, $f_{\lambda_{\textsc{TotVul},S},t}(\alpha)$ is true iff 
$\alpha$ satisfies at least one clause of DNF formula
$\phi_{\textsc{TotVul}}(S,t)$. 
\QED

\begin{corollary}\label{cor:formula_satprob_TotVul}
 The solution to the \tTotVuls{} problem is equal to
 the probability that formula $\phi_{\textsc{TotVul}}(S,t)$ is satisfied.
\end{corollary}
 
 To obtain an upper bound on the number of clauses in formula $\phi_{\textsc{TotVul}}(S,t)$,
 we consider the number of infectious paths of a given length.
 
 Let $\theta$ denote the number of index nodes. 
 Let $\Delta$ denote the maximum node degree of graph $G$.
 
 \begin{lemma}\label{lem:path_count_t}
 For any $t \geq 2$ and any node $v$, $|\mathcal{P}_t(v)| = O(\theta \Delta (\Delta-1)^{t-2})$.
 Also  $\mathcal{P}_t^{max}$ = \\ $O(\theta \Delta (\Delta-1)^{t-1})$.
 \end{lemma}
 \noindent
\textbf{Proof:}~
A path in $\mathcal{P}_t(v)$ can begin at any of the $\theta$ index nodes.
There is a choice of out most $\Delta$ edges from this initial index node.
For the next $t-2$ nodes along the path,
there is a choice of at most $\Delta-1$ edges to continue the path.
The last edge must go to $v$.
Thus, $v$, $|\mathcal{P}_t(v)| = O(\theta \Delta (\Delta-1)^{t-2})$.

Since $|\mathcal{P}_{t'}(v)| = O(\theta \Delta (\Delta-1)^{t'-2})$,
it follows that $\sum_{t' = 0}^t   |\mathcal{P}_{t'}(v)|$ is 
$O( \theta \Delta (\Delta-1)^{t-1} )$.
\QED

\begin{theorem}
\label{thm:totvul_approx}
For any $\epsilon, \delta\in(0,1)$, 
an $(\epsilon, \delta)$-approximation to the \tTotVuls{} problem
can be obtained in time 
$O(\frac{1}{\epsilon^2}\,|E|\,\theta^{|S|} \,\Delta^{|S|} \,(\Delta-1)^{(t-1)|S|}\log{(1/\delta)})$. 
\end{theorem}
\noindent
\textbf{Proof:}~
The approximation algorithm constructs formula $\phi_{\textsc{TotVul}}(S,t)$,
and then uses DNF sampling to obtain an estimate of $P(\phi_{\textsc{TotVul}}(S,t))$.
The number of variables in $\phi_{\textsc{TotVul}}(S,t)$ is $|E|$.
The number of clauses in $\phi_{\textsc{TotVul}}(S,t)$
is  $O( ( \mathcal{P}_t^{max} )^{|S|} )$.
From Lemma \ref{lem:path_count_t}, 
$\mathcal{P}_t^{max} = O(\theta \Delta (\Delta-1)^{t-1})$.
Therefore, the number of clauses in $\phi_{\textsc{TotVul}}(S,t)$
is  $O( \theta^{|S|} \,\Delta^{|S|} \,(\Delta-1)^{(t-1)|S|} )$.
The result follows from Theorem \ref{thm:num_dnf_prob_approx}. 
 \QED

Note that if in addition to having fixed values for $\epsilon$ and $\delta$, 
if $|S|$ and $t$ are also fixed,
the approximation algorithm runs in polynomial time.

We now consider the \tTotInfs{} problem.
Suppose that for a given problem instance, 
$q$ is the required number of nodes from set $S$ to be infected by time $t$. 
Given initial configuration $\cali{}$, 
for the  \tTotInfs{} problem
we can assume without loss of generality that the the problem instance  has been modified 
by the removal from graph $G$ of all nodes
that are in state \rstate{} in $\cali{}$,
and that no member of $S$ is an index node.

Define a DNF formula $\phi_{\textsc{TotInf}}(S,t)$ as follows:
\begin{enumerate}
\item
The set of Boolean variables is the set of edge variables $X(E)$.
%Create a Boolean variable $x(e)$ for each $e\in E$, with $\Pr[x(e)=1]=p(e)$. \\
(Thus, there are $|E|$ variables.)
\item
For each tuple of infectious paths $(P_1,\ldots,P_q)$ leading to $q$ distinct nodes 
$v_{i_1},  \ldots, v_{i_q}$
in set $S$, 
such for each $j$, $1 \leq j \leq q$,
there is  a $t' \leq t$,
such that $P_j \in \mathcal{P}_{t'}(v_{i_j})$,
%with $P_j\in \displaystyle{\cup_{t'\leq t}\mathcal{P}(s, v_j, t')}$, for
%$j=1,\ldots,k$, \\ 
we construct the clause $C(P_1,\ldots,P_q)=\bigwedge_{e\in E'} x(e)$, where
$E'$ is the set of edges occurring in any of the paths $P_1,\ldots,P_q$.
\item
The formula $\phi_{\textsc{TotInf}}(S,t)$ is the disjunction of all such clauses $C(P_1,\ldots,P_q)$.
\end{enumerate}

 \begin{proposition}\label{pro:formula_TotInf_correct}
 Formula $\phi_{\textsc{TotInf}}(S,t)$ represents the Boolean function 
 $f_{\lambda_{\textsc{TotInf},S},t}$.
 \end{proposition}
 \textbf{Proof:}~
From Proposition \ref{pro:infectious_path},
for a given edge assignment $\alpha \in {\cal A}_E$, 
$f_{\lambda_{\textsc{TotInf},S},t}(\alpha)$ is true iff 
there is a set of at most $q$ distinct nodes in $S$ such that for each of these nodes,
$G(V, \alpha^1)$ contains an infectious path to $v$ of length at most $t$.
Thus, $f_{\lambda_{\textsc{TotInf},S},t}(\alpha)$ is true iff 
$\alpha$ satisfies at least one clause of DNF formula
$\phi_{\textsc{TotInf}}(S,t)$. 
\QED

\begin{corollary}\label{cor:formula_satprob_TotInf}
 The solution to the \tTotInfs{} problem is equal to
 the probability that formula $\phi_{\textsc{TotInf}}(S,t)$ is satisfied.
\end{corollary}
 
 
\iffalse
%%%%%%%%%%%%%%%%%%%%%%%%%%%%%%%%%%%%%%%%%%
%%% This theorem has been stated above.
\begin{theorem}
\label{thm:totvul_approx}
For any $\epsilon, \delta\in(0,1)$, 
an $(\epsilon, \delta)$-approximation to the \tTotVuls{} problem
can be obtained in time 
$O(\frac{1}{\epsilon^2}\,|E|\,\theta^q \,\Delta^q \,(\Delta-1)^{(t-1)q} \, \log{(1/\delta)})$. 
\end{theorem}
%%%%%%%%%%%%%%%%%%%%%%%%%%%%%%%%%%%%%%%%%%
\fi
 
 
\begin{theorem}
\label{thm:totinf_approx}
For any $\epsilon, \delta \in (0,1)$, 
an $(\epsilon, \delta)$-approximation to the \tTotInfs{} problem
can be obtained in time 
$O(\frac{1}{\epsilon^2}\,|E|\, |S|^q \, \theta^q \,\Delta^q \,(\Delta-1)^{(t-1)q} \, \log{(1/\delta)})$. 
\end{theorem}
\noindent
\textbf{Proof:}~
The approximation algorithm constructs formula $\phi_{\textsc{TotInf}}(S,t)$,
and then uses DNF sampling to obtain an estimate of $P(\phi_{\textsc{TotInf}}(S,t))$.
The number of variables in $\phi_{\textsc{TotVul}}(S,t)$ is $|E|$.
Now consider the number of clauses in $\phi_{\textsc{TotVul}}(S,t)$.
Let $C(a,b)$ denote the number of combinations of $b$ objects chosen from a set of $a$ objects.
Note that $C(a,b)$ is $O(a^b)$.
For each choice of $q$ distinct nodes $v_{i_1},  \ldots, v_{i_q}$ in set $S$, 
the number of clauses in $\phi_{\textsc{TotVul}}(S,t)$ 
is  $O( ( \mathcal{P}_t^{max} )^q )$.
Therefore, the total number of clauses in $\phi_{\textsc{TotVul}}(S,t)$ 
is  $O( C(|S|,q) ( \mathcal{P}_t^{max} )^q )$,
which is $O(  |S|^q ( \mathcal{P}_t^{max} )^q )$.
From Lemma \ref{lem:path_count_t}, 
$\mathcal{P}_t^{max} = O(\theta \Delta (\Delta-1)^{t-1})$.
Therefore, the number of clauses in $\phi_{\textsc{TotInf}}(S,t)$
is  $O(  |S|^q \, \theta^q \,\Delta^q \,(\Delta-1)^{(t-1)q} )$.
The result follows from Theorem \ref{thm:num_dnf_prob_approx}. 
 \QED

Note that if in addition to having fixed values for $\epsilon$ and $\delta$, 
if $q$ and $t$ are also fixed,
the approximation algorithm runs in polynomial time.





%We first describe an algorithm for \tTotVuls{} for node set $S=\{v_1,\ldots, v_k\}$, which
%involves the following steps. 

%Let $P(S)$ denote the probability that 
%$\phi(S)$ has a satisfying assignment; that is,~ $P(S)$ ~=~ \\
%$\sum_{x \in \mathcal{X}} \Pr[x]$,~
%where $\mathcal{X} \subseteq \{0,1\}^E$ denotes the set of
%satisfying assignments to $\phi(S)$. \\
%Use the DNF sampling algorithm of \cite{karp:jc85} to estimate
%$P(S)$ within a factor of $(1\pm\epsilon)$ with \\
%probability at least $1-\frac{\delta}{n}$.~
%Let $\hat{P}(S)$ be the resulting estimate.

%We will show (Theorem~\ref{theorem:tvuldnf}) that the estimate
%$\hat{P}(S)$ provides a good approximation to the actual
%probability value $P(S)$.

%Recall from Section~\ref{sec:graphdefs} that
%$\Delta$ denotes the maximum node degree and 
%$\mathcal{P}(s, v, t)$ denotes the set of all paths of length $t$ 
%between $s$ and $v$ in $G$.

%The analysis of the above algorithm uses the following equivalence between
%fixed points in the SIR system and random subgraphs; a restricted form of
%this was used in the proof of Proposition \ref{pro:vulnerability_time_2}.
%Consider a sequence of configurations 
%$\bar{\mathcal{C}}=\langle\mathcal{C}_0,\ldots,\mathcal{C}_T\rangle$ 
%in an SIR system with index node $s$.
%(Note that  $\mathcal{C}_T$ is not necessarily a fixed point.)
%Let $V(\bar{\mathcal{C}}) = \langle V_0, \ldots, V_T \rangle$, where $V_t$
%denotes the set of nodes that get infected
%at each time $t$ in $\bar{\mathcal{C}}$, and let $E(\bar{\mathcal{C}})$ denote
%the set of edges on which the disease spreads. We say a subset $E'\subseteq E$ is
%``consistent'' with $\bar{\mathcal{C}}$ if
%all of the following three conditions hold:
%(1) the set of nodes at distance $t$
%from the index node $s$ in the graph $G[E']$ is precisely 
%the set $V_t$, for $t=0,\ldots,T$, 
%(2) $E(\bar{\mathcal{C}})\subseteq E'$, and
%(3) $E''=E(\bar{\mathcal{C}})$, where $E''=\{(u,v): u\in V_i, v\in V_j, \mbox{ with }i, j\leq T, i\neq j\}$
%is the subset of $E'$ with both endpoints in $\cup_{t\leq T} V_t$, but not within the same set $V_t$
%for any $t$.
%Let $\mathcal{E}(\bar{\mathcal{C}})$ denote
%the collection of all subsets consistent with $E(\bar{\mathcal{C}})$.

%\begin{lemma}
%\label{lemma:percolation}
%Consider a sequence of configurations 
%$\bar{\mathcal{C}}=\langle \mathcal{C}_0,\ldots,\mathcal{C}_T \rangle$ 
%in an SIR system.
%Let $V(\bar{\mathcal{C}})$, $E(\bar{\mathcal{C}})$ and $\mathcal{E}(\mathcal{C})$
%be as defined above. Then,
%$\Pr[\bar{\mathcal{C}}] = \sum_{E'\in \mathcal{E}(\bar{\mathcal{C}})} \Pr[E']$.
%end{lemma}
%\noindent
%\textbf{Proof:}~
%The proof is by identifying the structure of the subsets $E'$ consistent 
%with $\bar{\mathcal{C}}$. For $i, j\in\{0,\ldots,T\}$, let 
%$E_{ij}=\{(u,v): u\in V_i, v\in V_j\}$ be the set of edges with one
%end point in the set $V_i$ and the other in $V_j$. 
%For $i<T$, let $E^+_{i}=\{(u,v): u\in V_i, v\not\in \cup_{j\leq i} V_j\}$.
%Let $E_{ij}(\bar{\mathcal{C}})= E_{ij}\cap E(\bar{\mathcal{C}})$ be the
%subset of edges in $E_{ij}$ on which the infection spread in the
%sequence $\bar{\mathcal{C}}$.
%From the definition of the SIR system (Section \ref{sec:sir_dyn_system}), we have
%\[
%\Pr[\bar{\mathcal{C}}] = \prod_{t<T}
%\prod_{e\in E_{t,t+1}(\bar{\mathcal{C}})} p(e) 
%\prod_{e\in E^+_{i} - E(\bar{\mathcal{C}}), i< T} (1-p(e))
%\]

%By definition of $E'$, for each $v\in V_t$, $t< T$, we have
%$\{e=(u,v)\in E': u\in V_t, v\in V_{t+1}\} = E_{t,t+1}(\bar{\mathcal{C}})$.
%For any $e\in E_{t,t}$, whether or not $e\in E'$ does not affect its
%onsistency with $E(\bar{\mathcal{C}})$. 
%Further, for $e=(u,v)\in E^+_i$, $i<T$, it must be the case that
%$e\not\in E'$; else, node $v$ would be at distance $i+1$ from the source
%in $G[E']$, making $E'$ inconsistent with $\bar{\mathcal{C}}$.
%Let $E_1 = \{e=(u,v): u\in V_T, v\not\in \cup_{t\leq T} V_t\}$ and
%$E_2=\{e=(u,v): u, v\not\in \cup_{t\leq T} V_t\}$. For edges
%$e\in E_1\cup E_2$, 
%whether or not $e\in E'$ does not affect its consistency with
%$E(\bar{\mathcal{C}})$. Therefore, for any $E'\in \mathcal{E}(\bar{\mathcal{C}})$, we have
%\[
%\Pr[E'] = \Pr[\bar{\mathcal{C}}]\prod_{t, e\in E_{tt}\cap E'} p(e)
%\prod_{t, e\in E_{tt} - E'} (1-p(e))
%\prod_{e\in (E_1\cup E_2)\cap E'} p(e)
%\prod_{e\in (E_1\cup E_2)- E'} (1-p(e))
%\]

%From the above discussion, it follows that when the
%above expression is summed up over all $E'\in \mathcal{E}(\bar{\mathcal{C}})$, 
%we get
%\[
%\Pr[\bar{\mathcal{C}}]\prod_{t, e\in E_{tt}}(p(e) + (1-p(e))) \prod_{e\in E_1\cup E_2}(p(e) + (1-p(e))),
%\]
%which is precisely $\Pr[\bar{\mathcal{C}}]$, and the lemma follows.
%\QED

%\begin{theorem}
%\label{theorem:tvuldnf}
%For any $\epsilon, \delta\in(0,1)$, the estimate $\hat{\alpha}(S)$ 
%from the above algorithm gives an
%$(\epsilon, \delta)$-approximation to the \tTotVuls{} problem for set $S$
%in time $O(\frac{1}{\epsilon^2}\,m\,\Delta^{2t|S|}\log{(1/\delta)})$. 
%\end{theorem}
%\noindent
%\textbf{Proof:}~
%Let $\phi(S)$ denote the DNF formula constructed in the above algorithm. Let
%$\mathcal{X}\subseteq \{0,1\}^E$ denote the set of satisfying assignments for $\phi(S)$.
%For each $x\in\mathcal{X}$, we have $\Pr[x] = \Pi_{e: x(e)=1} p(e)$ is the
%probability of the assignment $x$; the total probability that $\phi(S)$ is satisfiable
%equals $\sum_{x\in\mathcal{X}} \Pr[x]$. Next, observe that
%the solution to the \tTotVuls{} problem
%involves computing $\sum_{\bar{\mathcal{C}}}\Pr[\bar{\mathcal{C}}]$, where 
%the summation is over all sequences of configurations
%$\bar{\mathcal{C}}$ in the phase space, which cause all the nodes in $S$ to get infected by time $t$.
%Let $\mathcal{E}=\cup_{\bar{\mathcal{C}}} \mathcal{E}(\bar{\mathcal{C}})$,
%where the union is over all sequences of configurations
%$\bar{\mathcal{C}}$ that are relevant to the \tTotVuls{} problem.
%It is also easy to verify that for distinct $\bar{\mathcal{C}}$ and $\bar{\mathcal{C}'}$,
%the corresponding sets of edges, $\mathcal{E}(\bar{\mathcal{C}})$ and
%$\mathcal{E}(\bar{\mathcal{C}'})$, respectively, are disjoint. Therefore,
%from Lemma \ref{lemma:percolation}, it follows that the solution to the
%\tTotVuls{} problem equals 
%$\sum_{E'\in \mathcal{E}} \Pr[E']$.
%We will prove below that 
%$\sum_{x\in\mathcal{X}} \Pr[x] = \sum_{E'\in \mathcal{E}}\Pr[E']$. 
%Together, these imply that the reduction to DNF sampling correctly gives an approximation
%to the \tTotVuls{} problem. 
%Finally, in the reduction to DNF, the formula $\phi(S)$ has $m$ variables
%and $O(\Delta^{t|S|})$ clauses, since $|\mathcal{P}(s, v, t)|=O(\Delta^t)$
%for each $v\in S$.  Therefore, from the algorithm of \cite{karp:jc85}
%(as discussed in Section \ref{sse:other_def}), the running time is
%$O(\frac{1}{\epsilon^2}\,m\,\Delta^{2t|S|}\log{(1/\delta)})$, and the theorem follows.

%We now prove that 
%$\sum_{x\in\mathcal{X}} \Pr[x] = \sum_{E'\in \mathcal{E}}\Pr[E']$. We show this
%by proving that there exists a one-to-one correspondence between
%$x\in\mathcal{X}$ and a set $E'(x)=\{e: x(e)=1\}\in \mathcal{E}$.
%By construction,
%a clause $C(P_1,\ldots, P_k)$ is satisfied if and only if $x(e)=1$ for all edges
%$e\in E(P_1)\cup\ldots E(P_k)$. 
%Therefore, for each $x\in\mathcal{X}$, there must be a tuple of paths $P_1,\ldots, P_k$
%with $P_j\in\mathcal{P}(s, v_j, t')$ such that $t'\leq t$, and 
%$x(e)=1$ for all $e\in E(P_1)\cup\ldots E(P_k)$. Since each of these paths
%has length at most $t$, it follows that $E'(x)=\{e: x(e)=1\}$ is contained in the set
%$\mathcal{E}$ defined above.  Next, consider
%any $E'\in \mathcal{E}$. By definition, the component containing source node $s$ in the graph
%$G[E']$ contains all the nodes $v_j\in S$ at distance at most $t$ from $s$. This
%implies that there must exist paths $P_j\subset E'$ from $s$ to $v_j$,
%for each $v_j\in S$. Therefore, the assignment $x$ defined by $x(e)=1$ for all
%$e\in E'$ satisfies some clause $C(P_1,\ldots, P_k)$. 
%Further, $\Pr[x] = \Pr[E']$ for the corresponding $x, E'$. Therefore,
%it follows that 
%$\sum_{x\in\mathcal{X}} \Pr[x] = \sum_{E'\in \mathcal{E}'}\Pr[E']$.
%\QED

%%%%%%%%%%%%%%%%%%%%%%%%%%%%%%%%%%%%%%%%%%%%%%%%%%
%\textcolor{red}{This section is not correct, since it uses a solution to the
%\tVul{} problem.}
%\subsubsection{Efficient approximation of the expected number of new infections
%at a given time, and higher moments}
%\label{sec:tnewinf}
%
%
%The \tNewInf{} problem involves computing the probability that 
%the number of nodes in $S$ which get infected at time $t$ is at least $q$.
%Here, we discuss how to compute the expectation of
%number of nodes in $S$ which get infected at time $t$, and its higher moments.
%Obtaining good approximations to \tNewInf{} is a challenging problem.
%
%\begin{corollary}
%\label{cor:ttotinf}
%For any $\epsilon, \delta\in(0,1)$, an $(\epsilon, \delta)$-approximation
%of the $k$-th moment of the number of nodes in $G$ which get infected at time $t$
%can be estimated in
%$O(\frac{8}{\epsilon^2}mn^{k}\Delta^{2tk}k\log{n/\delta})$ time. 
%\end{corollary}
%\textbf{Proof:}~
%For simplicity, we first discuss the proof for $k=1$.
%Let $X(v)$ denote the event that $v$ gets infected at time $t$. Then,
%$E[X(v)] = \Pr[X(v)=1]$ equals the \tVul{} solution for set $S=\{v\}$, and
%the expected number of nodes in $G$ which get infected at time $t$ equals
%$\sum_{v\in V} E[X(v)]$. For each node $v$, we use Theorem \ref{theorem:tvuldnf}
%to compute an $(\epsilon, \delta/n)$-approximation $\hat{\alpha}(v)$ to $E[X(v)]$ such that
%\[
%\Pr[\hat{\alpha}(v) \not\in [(1-\epsilon) E[X(v)], (1+\epsilon) E[X(v)]]]\leq\frac{\delta}{n}
%\]
%in time $O(\frac{8}{\epsilon^2}m\Delta^{2t}\log{n/\delta})$. From a
%union bound, we have
%\[
%\Pr[\sum_{v\in S} \hat{\alpha}(v) \not\in [(1-\epsilon) \sum_{v\in S} E[X(v)], (1+\epsilon) \sum_{v\in S} E[X(v)]] \leq \delta,
%\]
%which implies $\sum_{v\in S} \hat{\alpha}(v)$ is an $(\epsilon, \delta)$-approximation
%to $\sum_{v\in S} E[X(v)]$, within the required time bound.
%
%Next, we consider the computation of $E[(\sum_{v\in S} X(v))^k]$. By linearity, we have
%\[
%E[(\sum_{v\in S} X(v))^k] = \sum_{S'\subseteq S} E[\Pi_{v\in S'} X(v)], 
%\]
%since the $X(v)$'s are binary variables. Observe that $E[\prod_{v\in S'} X(v)]$ is the
%solution to the \tVul{} problem for set $S'$, and Theorem \ref{theorem:tvuldnf} gives an
%$(\epsilon, \delta/n^k)$-approximation to it,
%denoted by $\hat{\alpha}(S')$, in 
%$O(\frac{8}{\epsilon^2}m^{k+1}\Delta^{2kt}k\log{n/\delta})$ time.  This
%implies 
%\[
%\Pr[\hat{\alpha}(S')\not\in [(1-\epsilon)E[\prod_{v\in S'} X(v)], (1+\epsilon)E[\prod_{v\in S'} X(v)]] \leq \delta/n^k.
%\]
%It follows from a union bound that $\sum_{S'\subseteq S} \hat{\alpha}(S')$
%is an $(\epsilon, \delta)$-approximation to $E[(\sum_{v\in S} X(v))^k]$.
%\QED
%

\subsubsection{Exact computation of variance of the number of new infections
at $t=2$ with uniform transmission probability}
\label{sec:exact-t2}

%%%% We decided to delete the section on computing higher order moments
%%In contrast to Section \ref{sec:tnewinf}, which gives approximations to the
%%moments of the number of new infections in $S$ at a given time $t$, we observe that

Here, we show that for $t=2$, the variance of the number of
new infections can be computed exactly,
when the transmission probability is uniform, i.e., $p(e)=p$
for all edges $e\in E$.

\begin{proposition}
The variance of the number of new infections in set $S$ at $t=2$ can be computed in
polynomial time under uniform transmission probability.
\end{proposition}

\noindent
\textbf{Proof:}~
Let $V_1$ be the nodes at distance $1$ from source $s$. Let $V_2$ be the
set of nodes in $S$ at distance 2 from $s$.
Let $X(v)$ denote the event that $v$ gets infected at time $t=2$.
From Corollary~\ref{cor:4probs_time_2}, it follows that the \TwoVuls{}
problem where $S = \{v\}$ can be solved exactly, which implies that $E[X(v)]$ can
be computed exactly. This implies, for all nodes $u, v$, $E[X(u)]E[X(v)]$
can be computed exactly. Therefore, computing the variance of $\sum_{v\in S} X(v)$
reduces to computing $E[X(i)X(j)]$ for all pair of nodes $i, j\in V_2$.


\begin{figure}
\rule{\textwidth}{0.01in}
%%\centering
%%\includegraphics[scale=1.0]{var.pdf}
\begin{center}
\input{anil_figure_3.pdf_t}
\end{center}
\caption{}
\label{fig:vart2}
\rule{\textwidth}{0.01in}
\end{figure}

We focus on a specific pair of nodes $i, j\in V_2$.
As shown in Figure \ref{fig:vart2},
we partition $V_1= A\cup B \cup C$, where: (1) $A$ is the set of nodes $u\in V_1$
such that $(u,i)\in E$, but $(u,j)\not\in E$; (2) $B$ is the set of nodes $u\in V_1$
such that $(u,i), (u,j)\in E$; and (3) $C$ is the set of nodes $u\in V_1$
such that $(u,i)\not\in E$, but $(u,j)\in E$. Let $|A|=n_A$, $|B|=n_B$ and $|C|=n_C$.

We consider the random subgraph $E(p)$ in which each edge is picked with probability $p$.
We define the following random variables:
\begin{itemize}
\item
$Z_1=1$ if there exists a path $s, u, i$ in $E(p)$, for some node $u\in A$ $E(p)$.
We have $\Pr[Z_1]=1 - (1-p^2)^{n_A}$.
\item
$Z_2=$ if there exists a path $s, u, i$ in $E(p)$, for some node $u\in B$.
We have $\Pr[Z_2] = 1 - (1-p^2)^{n_B}$.
\item
$Z_3=$ if there exists a path $s, u, j$ in $E(p)$, for some node
$u\in B$. 
We have $\Pr[Z_3] = 1 - (1-p^2)^{n_B}$.
\item
$Z_4=1$ if there exists a path $s, u, j$ in $E(p)$ for some $u\in C$.
We have $\Pr[Z_4]=1 - (1-p^2)^{n_C}$.
\end{itemize}

Next, we have $\Pr[X_i=X_j=1]=\Pr[Z_1Z_4] + \Pr[Z_1Z_3\bar{Z_4}] + \Pr[\bar{Z_1}Z_2Z_4] + 
\Pr[\bar{Z_1}Z_2Z_3\bar{Z_4}]$, since these induce a mutually disjoint set of events.
Any two events $Z_i, Z_j$, such that $\{i, j\}\neq \{2, 3\}$ are independent.
Therefore, the first three terms of the above sum can be computed easily. Since
$Z_2$ and $Z_3$ are not independent, we describe how to estimate the last term below.

We partition $B$ into sets $B_1, B_2, B_3, B_4$, where $B_i$ is the set
of nodes which are adjacent (or not adjacent) to $i$ and $j$, as shown in Fig \ref{fig2}.
For a given such partition,
let $Y_i$ denote the event that the configuration for the $B_i$ nodes occurs.
Let $n^i_B=|B_i|$.


Finally, we consider $\Pr[Z_2=Z_3=1]$. This happens if either event $Y_1$ occurs, or
we have $\bar{Y_1}\bigwedge Y_2\bigwedge Y_3$. The latter event happens
if $n^1_B=0$, $n^2_B, n^3_B>0$.
Therefore, 
\[\Pr[Z_2\,Z_3] = \sum_{n^1_B, n^2_B, n^3_B, n^4_B} \frac{n_B!}{n^1_B! (n_B-n^1_B)!} p^{3n^1_B} +
                \frac{n_B!}{n^2_B! n^3_B!(n_B-n^2_B-n^3_B)!} 
                 (p^2(1-p))^{n^2_B}(p^2(1-p))^{n^3_B},
\]
where the summation is over all $n^1_B, n^2_B, n^3_B, n^4_B$ that sum to $n_B$.
Clearly, there are at most $n^4$ such combinations. Finally, $Z_2Z_3$ is independent
of $\bar{Z_1}$ and $\bar{Z_4}$. Therefore,
putting these together, we can determine $\Pr[\bar{Z_1}Z_2Z_3\bar{Z_4}]$, and
in turn, $\Pr[X_iX_j]$.
\QED

\begin{figure}
\rule{\textwidth}{0.01in}
%%\centering
%%\includegraphics[scale=1.0]{var.pdf}
\begin{center}
\input{anil_figure_4.pdf_t}
\end{center}
\caption{}
\label{fig2}
\rule{\textwidth}{0.01in}
\end{figure}

\subsubsection{Efficient approximation of the expected total number of infections
in set $S$ up to time $t$}

Recall from Section \ref{sse:prob_formulation} that the \tTotInfs{} problem
involves computing the probability that the number of nodes
in $S$ which are infected by time $t$ is at least $q$. While approximating
\tTotInfs{} ~for an arbitrary set $S$ remains open, we observe that the expected 
total number of infections in $S$
by time $t$ can be approximated by 
simple Monte-Carlo sampling, provided the initial infection occurs in $S$;
this is because when the initially infected node is in $S$,
the total number of infections up to time $t$ in $S$ has bounded variance.  Let
$\epsilon, \delta\in (0,1)$ be parameters.  The basic idea 
is to run $N$ Monte-Carlo
samples of the SIR process on $G$, and compute an estimate in the
following manner:

\begin{enumerate}
\item
For~ $j=1$ ~to ~$\lceil \log{(2/\delta)} \rceil$:
\begin{enumerate}
\item
For $i=1$ to $N=\left\lceil 8n^2/\epsilon^2\right\rceil$: 
use the SIR process to 
find a random fixed point (as discussed in Section \ref{sec:sir_dyn_system}).
Let $Y_{ij}$ denote the cumulative number of infections
in this fixed point.
\item
Let $Y_j =\sum_{i=1}^N Y_{ij}/N$
\end{enumerate}
\item
Return the median $\hat{Y}$ of $\{Y_j: j=1,\ldots,\lceil\log{(2/\delta)}\rceil\}$.
\end{enumerate}


\begin{lemma}
\label{lemma:mcmc-numinf}
Let $\epsilon, \delta\in (0,1)$ be parameters.
For $N\geq \frac{8}{\epsilon^2}n^2\log{(2/\delta)}$, $\hat{Y}$ is
an $(\epsilon, \delta)$-approximation to the
expected total number of infections in $S$ up to time $t$, provided the source of
infection is in $S$.
\end{lemma}
\noindent
\textbf{Proof:}~
Let $Y$ denote the number of infections in $S$ up to time $t$ for a given SIR instance.
Since the source node is in $S$, and always gets infected, we have $Y\geq 1$, so
that $E[Y]\geq 1$, and $var(Y)\leq E[Y^2]\leq n^2$.
From Chebyshev's Inequality, it follows that 
$\Pr[|Y_j - E[Y] | \geq \epsilon E[Y]]\leq 
\frac{\sum_i var(Y_{ij})}{(8n^2/\epsilon^2)\epsilon^2E[Y]^2} \leq 1/8$.

Next, the probability that the median $\hat{Y}$ is less than
$(1-\epsilon)E[Y]$ is at most ${t \choose t/2} 8^{-t/2}\leq
2^{-t}\leq \delta/2$, for $t=\log{(2/\delta)}$.  Similarly, the
probability that $\hat{Y}$ is larger then $(1+\epsilon)E[Y]$
is at most $\delta/2$, and the lemma follows. \QED

The above lemma implies that the Monte-Carlo sampling method 
with $\frac{8}{\epsilon^2}n^2\log{(2/\delta)}$ samples gives a
$(1\pm\epsilon)$-approximation to the \tTotInfs{} problem for set $S$, with
probability at least $1-\delta$, provided the source (the index node) is in $S$.
This can be used for the forecasting problem for specific subpopulations, such
as children (which can be modeled by the choice of $S$). If the source is not
in $S$, the above method does not work, since $E[Y]$ could be very small.
As discussed in Section \ref{sse:related_work}, the approach of
Dagum et al. \cite{dagum:focs95} can be used as an estimator, and is
guaranteed to run in $O(\frac{8}{\epsilon^2}n^2\log{(2/\delta)})$ steps.
