\section{Short-Term Forecasting Problems for SIR Systems}
\label{sec:prob_form}

\iffalse
%%%%%%%%%%%%%%%%%%% These definitions have been moved to prob_names.tex %%%%%%%%%

%%% Macros for problem names.

%%% General forms.
\newcommand{\tNewInfs}{\mbox{$t$}-\textsc{NewInf}\mbox{$(S)$}}
\newcommand{\tNewInfv}{\mbox{$t$}-\textsc{NewInf}\mbox{$(V)$}}
\newcommand{\tTotInfs}{\mbox{$t$}-\textsc{TotInf}\mbox{$(S)$}}
\newcommand{\tTotInfv}{\mbox{$t$}-\textsc{TotInf}\mbox{$(V)$}}
\newcommand{\tPeak}{\mbox{$t$}-\textsc{Peak}}
\newcommand{\tVuls}{\mbox{$t$}-\textsc{Vul}\mbox{$(S)$}}
\newcommand{\tVulv}{\mbox{$t$}-\textsc{Vul}\mbox{$(V)$}}
\newcommand{\tTotVuls}{\mbox{$t$}-\textsc{TotVul}\mbox{$(S)$}}
\newcommand{\tTotVulv}{\mbox{$t$}-\textsc{TotVul}\mbox{$(V)$}}

%%% Versions with t = 2.
\newcommand{\TwoNewInfs}{\mbox{$2$}-\textsc{NewInf}\mbox{$(S)$}}
\newcommand{\TwoNewInfv}{\mbox{$2$}-\textsc{NewInf}\mbox{$(V)$}}
\newcommand{\TwoTotInfs}{\mbox{$2$}-\textsc{TotInf}\mbox{$(S)$}}
\newcommand{\TwoTotInfv}{\mbox{$2$}-\textsc{TotInf}\mbox{$(V)$}}
\newcommand{\TwoPeak}{\mbox{$2$}-\textsc{Peak}}
\newcommand{\TwoVuls}{\mbox{$2$}-\textsc{Vul}\mbox{$(S)$}}
\newcommand{\TwoVulv}{\mbox{$2$}-\textsc{Vul}\mbox{$(V)$}}
\newcommand{\TwoTotVuls}{\mbox{$2$}-\textsc{TotVul}\mbox{$(S)$}}
\newcommand{\TwoTotVulv}{\mbox{$2$}-\textsc{TotVul}\mbox{$(V)$}}

%%% Versions with t = 1.
\newcommand{\OneNewInfs}{\mbox{$1$}-\textsc{NewInf}\mbox{$(S)$}}
\newcommand{\OneNewInfv}{\mbox{$1$}-\textsc{NewInf}\mbox{$(V)$}}
\newcommand{\OneTotInfs}{\mbox{$1$}-\textsc{TotInf}\mbox{$(S)$}}
\newcommand{\OneTotInfv}{\mbox{$1$}-\textsc{TotInf}\mbox{$(V)$}}
\newcommand{\OnePeak}{\mbox{$1$}-\textsc{Peak}}
\newcommand{\OneVuls}{\mbox{$1$}-\textsc{Vul}\mbox{$(S)$}}
\newcommand{\OneVulv}{\mbox{$1$}-\textsc{Vul}\mbox{$(V)$}}
\newcommand{\OneTotVuls}{\mbox{$1$}-\textsc{TotVul}\mbox{$(S)$}}
\newcommand{\OneTotVulv}{\mbox{$1$}-\textsc{TotVul}\mbox{$(V)$}}

%%% Version with t = 3.
\newcommand{\ThrNewInfs}{\mbox{$3$}-\textsc{NewInf}\mbox{$(S)$}}
\newcommand{\ThrNewInfv}{\mbox{$3$}-\textsc{NewInf}\mbox{$(V)$}}
\newcommand{\ThrTotInfs}{\mbox{$3$}-\textsc{Tot}\mbox{$(S)$}}
\newcommand{\ThrTotInfv}{\mbox{$3$}-\textsc{Tot}\mbox{$(V)$}}
%%%%%%%%%%%%%%%%%%% End of problem names %%%%%%%%%
\fi

\subsection{Problems Considered}
\label{sse:prob_formulation}

For simplicity, we will represent an SIR
system by the underlying graph $G(V,E)$, with the
understanding that the transmission probability for each
edge $e \in E$ is also given.
It can be seen from the discussion in Section~\ref{sec:sir_dyn_system}
that the graph and the transmission probability values completely
specify all local transition functions. 
We will also use the phrase ``initial configuration" of the
system to mean the configuration at time $t = 0$.
We now present formal definitions of the forecasting problems 
studied in this paper.
In each case, the goal is to compute the probability 
of a certain event; the long name of a problem describes the
corresponding event.
We begin with a problem whose goal is to compute
the probability that a certain number of 
new infections occur at time $t$. 

\medskip

\noindent
\textbf{(1)~ Number of New Infections in a set $S$ at a Given Time}~ (\tNewInfs)

\medskip
\noindent
\underline{Instance:}~ An SIR system $G(V,E)$ along with its initial
configuration \cali; set $S \subseteq V$; integers $q \leq |S|$ and $t \geq 1$.

\smallskip
\noindent
\underline{Requirement:}~ The probability of the following
event:~ the number of nodes in $S$ which get infected at time 
$t$ is at least $q$.

\medskip
The next problem concerns the total number of 
infections that occur by a given time $t$. 
Using $g(S, j)$ to denote the number of nodes from set $S$ in state \istate{}
at time $j$ (in a stochastic outcome), we note that the total number of infections 
within set $S$ by
time $t$ is given by $\sum_{j=0}^{t} g(S, j)$ (in that outcome).

\medskip
\noindent
\textbf{(2)~ Total Number of Infections in a set $S$ by a Given Time}~ (\tTotInfs)

\medskip
\noindent
\underline{Instance:}~ An SIR system $G(V,E)$ along with its initial
configuration \cali; set $S \subseteq V$; integers $q \leq |V|$ and $t \geq 1$.

\smallskip
\noindent
\underline{Requirement:}~ The probability of the following event:~
the number of nodes
in $S$ which are infected by time $t$ is at least $q$.

\medskip
The next two problems concern the probability that all the nodes
in a specified subset $S$ get infected. 
This probability is referred to as the \textbf{vulnerability} of the set $S$
\cite{BB+2009}.

\medskip
\noindent
\textbf{(3)~ Vulnerability of a set $S$ at a Given Time}~ (\tVuls)

\medskip
\noindent
\underline{Instance:}~ An SIR system $G(V,E)$ along with its initial
configuration \cali;~ a subset of nodes $S \subseteq V$;~ an integer $t \geq 1$.

\smallskip
\noindent
\underline{Requirement:}~ The probability of the following
event:~ \emph{all} the nodes in $S$ get infected ~\emph{at}~ time $t$. 

\medskip
We note that the result of the \tVuls{} problem is the
probability that \emph{all} nodes in $S$ get infected 
\emph{simultaneously at} time $t$.

\medskip
\noindent
\textbf{(4)~ Vulnerability of a set $S$ by a Given Time}~ (\tTotVuls)

\medskip
\noindent
\underline{Instance:}~ An SIR system $G(V,E)$ along with its initial
configuration \cali;~ a subset of nodes $S \subseteq V$;~ an integer $t \geq 1$.

\smallskip
\noindent
\underline{Requirement:}~ The probability of the following
event:~ all the nodes in $S$ get infected ~\emph{by}~ time $t$. 

\medskip
Note the difference
between the problems \tVuls{} and \tTotVuls{}--in the latter problem, 
the nodes in $S$ are not required to get infected
simultaneously at time $t$.

\medskip
Problems \tVuls{} and \tTotVuls{} are special versions 
of \tNewInfs{} and \tTotInfs{} respectively with $q = |S|$.
For simplicity, we will continue to use \tVuls{} and \tTotVuls.

\medskip
When $S = V$ (i.e., the set of all nodes),
the above four problems are denoted by 
\tNewInfv, \tTotInfv, \tVulv{} and \tTotVulv{}
respectively.

\medskip
The next problem considers 
the time at which the number of new
infections (among all the nodes of the network)
reaches a peak (i.e., attains a maximum value).

\medskip
\noindent
\textbf{(5)~ Peak Time}~ (\tPeak)

\medskip
\noindent
\underline{Instance:}~ An SIR system $G(V,E)$ along with its initial
configuration \cali;~ an integer $t \geq 1$.

\smallskip
\noindent
\underline{Requirement:}~ The probability of the following event:~
the largest number of new infections in the network occurs at time $t$.

\medskip

\noindent
\textbf{Note:}~
Concise descriptions of
the problems defined above are shown in Table~\ref{tab:prob_def}.

\medskip
\begin{table}[tbh]
\begin{center}
\begin{tabular}{|p{1.6in}|p{3.9in}|} \hline
\multicolumn{1}{|c|}{\textbf{Problem Name}} & 
\multicolumn{1}{|c|}{\textbf{Description}} \\ \hline\hline   
\tNewInfs{} & {Compute the probability that the number of new infections 
              within a given subset $S$ of nodes \emph{at} time $t$ 
           is at least $q$.} \\ \hline
\tTotInfs{} & {Compute the probability that the total number of 
              infections within a given subset $S$ of nodes \emph{by} time $t$ is 
              at least $q$.} \\ \hline
\tVuls{} & {Compute the probability that \emph{all} the nodes 
                         in a given subset $S$ get infected 
                         \emph{at} time $t$.} \\ \hline
\tTotVuls{}   & {Compute the probability that \emph{all} the nodes 
                             in a given subset $S$ get infected 
                             \emph{by} time $t$.} \\ \hline
\tPeak{}   & {Compute the probability that the number of new infections in the
              network reaches a peak at time $t$.} \\ \hline\hline
\end{tabular}
\end{center}
\caption{Concise descriptions of the forecasting 
problems considered in the paper.
Problems \tVuls{} and \tTotVuls{} are special versions 
of \tNewInfs{} and \tTotInfs{} respectively with $q = |S|$.
When $S = V$ (the set of all nodes in the network), we denote
the first four problems by \tNewInfv, \tTotInfv, \tVulv{} and \tTotVulv{}
respectively.
}
\label{tab:prob_def}
\end{table}

\subsection{Other Definitions} 
\label{sse:other_def}

\subsubsection{Some Graph Theoretic Measures and Notation}
\label{sec:graphdefs}

We now define some graph theoretic measures which will be used
in subsequent sections;
additional information regarding these measures can be
found in \cite{newman_2010}.
Given an undirected graph $G(V,E)$, the \textbf{length of a path} between two
nodes $u$ and $v$ is the number of edges in the path.
The length of a shortest path between $u$ and $v$ will be denoted by
$d_G(u,v)$; when the graph is clear from the context, we will just denote it by $d(u, v)$.
The \textbf{diameter} of a graph, denoted by $\delta(G)$, is given by
\[
\delta(G) ~=~ \max \{d(u,v) ~:~ u,v \in V\}.
\]
For any node $v \in V$, recall that $N_v$ denotes the set of neighbors of $v$
(i.e., the set of nodes which are adjacent to $v$ in $G$).
Further, let $G_v(N_v, E_v)$ denote the subgraph of $G$ induced on
the node set $N_v$.
The \textbf{clustering coefficient} of $v$, denoted by $CC(v)$, 
is the ratio of the number of edges in $G_v$ to the maximum possible
number of edges in $G_v$.
Thus, the value of $CC(v)$ can be computed using the following expression:
\[
CC(v) ~=~ \frac{2|E_v|}{|N_v|(|N_v| - 1)}.
\]
As the name implies, the \textbf{mean clustering coefficient} 
is the average\footnote{
If $v$ is an isolated node (i.e., has no incident edges), the convention
is to set $CC(v) = 0$.
An alternative approach to deal with isolated nodes in
computing the mean clustering coefficient is presented in \cite{Kaiser-2008}.
}
of the clustering coefficients of all the nodes.
We use $\Delta$ to denote the maximum degree of any node in $G$
and $\mathcal{P}(s, v, t)$ to
denote the set of all paths of length $t$ between $s$ and $v$ in $G$.

\subsubsection{Monte Carlo Sampling}
\label{sss:monte_carlo}

Monte Carlo sampling, a standard approach for approximating the quantities
considered in this paper, has been applied in a large number of domains
\cite{hammersley64}.  
To present the relevant details, assume that we want to estimate
the mean $\mu_Z$ of a random variable $Z$.
Let $Z_1$, $Z_2$, $\ldots$, $Z_N$ denote $N$ randomly sampled values of $Z$.
Further, let $S=Z_1+\ldots+Z_N$.  
As discussed in \cite{dagum:focs95}, for 
$N = O((1/\epsilon^2)\ln{(1/\delta}))$, $S/N$ gives an approximation to $\mu_Z$
with \emph{absolute} error $\epsilon$. 
In this paper, we focus on 
obtaining a \emph{relative} approximation, also called an
$(\epsilon, \delta)$-approximation, for given parameters 
$\epsilon$, $\delta$  $\in (0,1)$.
A relative approximation of $\mu_Z$ is an estimate $\widehat{\mu_Z}\:$ such that 
\[
\Pr\left[\,|\widehat{\mu_Z}-\mu_Z| \,\geq\, \epsilon\mu_Z\,\right] 
        ~\leq~ \delta.
\]
\noindent
Unlike the absolute approximation, the running time for
such a relative approximation is a function of $\mu_Z$ and $\sigma^2_Z$
(the variance of $Z$). 
In general, 
$\mu_Z$ and $\sigma^2_Z$ are not known, so precise running time bounds
for Monte Carlo methods are challenging to obtain.
Dagum et al. \cite{dagum:focs95} develop an algorithm that
determines the number of samples $N$ to obtain an $(\epsilon, \delta)$-approximation,
such that $N$ is within a constant factor of the optimal. 
In general, the minimum number of samples need not be 
polynomially bounded; thus, the algorithm
of \cite{dagum:focs95} is not guaranteed to run in polynomial time.


\subsubsection{Boolean Satisfiability Problems} 
\label{sss:boolean_sat}

Some known results regarding two forms
of Boolean satisfiability problems 
will be used in the subsequent sections of this paper.
We mention these problems and results below.

\newcommand{\mtsat}{\mbox{\#M2SAT}}
\newcommand{\dnfsat}{\mbox{\#DNFSAT}}

\medskip
\noindent
\textbf{I.~ Number of Satisfying Assignments to Monotone 2SAT}~ (\mtsat)

\medskip
\noindent
\underline{Instance:}~ A set $X = \{x_1, x_2, \ldots, x_n\}$ of
Boolean variables and a collection 
$C = \{C_1, C_2, \ldots, C_m\}$ clauses,
where each clause $C_j = (x_a \vee x_b)$ for some variables $x_a$ and
$x_b$ $\in X$.

\smallskip
\noindent
\underline{Requirement:}~ Compute the number of assignments 
to the variables in $X$ that satisfy all the clauses in $C$. 

\medskip
It is shown in \cite{Vad-2001} that \mtsat{} is \cnump-hard.
Problems which are \cnump-hard are widely believed to 
computationally intractable \cite{GJ-1979}.
Without loss of generality, we will assume throughout this paper
that any given instance of ~\mtsat{} consists of
at least two variables and at least two clauses.

\medskip
The following result which provides an 
indication of the difficulty of obtaining an approximate solution to the
\mtsat{} problem is established in \cite{Zu-1996}.

\begin{theorem}\label{thm:m2sat_zuckerman}
There is an $\epsilon > 0$ such that if the logarithm of the number of satisfying
assignments to an \mtsat{} instance with $n$ variables can be approximated to within 
a factor of $O(n^{\epsilon})$ in polynomial time, then \textbf{P} = \cnp. \QED
\end{theorem}

We will use the above theorem in establishing the difficulty of approximating
some of the short-term forecasting problems defined above.

\medskip
In \mtsat, the Boolean formula considered is in 
Conjunctive Normal Form (CNF); that is, the formula
has the form $\displaystyle{\bigwedge_{j=1}^m C_j}$, with 
each clause $C_j$ being a disjunction of literals, $1 \leq j \leq m$. 
The following problem uses a dual form of Boolean
functions, namely the Disjunctive Normal Form (DNF).

\medskip

\noindent
\textbf{II.~ Probability of Satisfying a DNF Boolean Formula}~ (\dnfsat)

\smallskip
\noindent
\underline{Instance:}~ A set $X = \{x_1, x_2, \ldots, x_n\}$ of
Boolean variables, a formula 
$F = \displaystyle{\bigvee_{i=1}^{m} D_j}$, where each term $D_j$
is a conjunction of literals formed from the variables in $X$,
$1 \leq j \leq m$; parameters $p_1$, $p_2$, $\ldots$, $p_n$,
where  $p_i \in [0,1]$, $1 \leq i \leq n$, such that 
variable $x_i$ independently takes on the value 1
with probability $p_i$, $1 \leq i \leq n$.

\smallskip
\noindent
\underline{Requirement:}~ Compute the probability that the formula $F$ is satisfied.

\medskip
It is known that \dnfsat{} is \cnump-hard \cite{GJ-1979}.
For a given instance of \dnfsat, let $p^*$ denote the actual
probability that the DNF formula $F$ is satisfied.
For given parameters $\epsilon, \delta\in(0, 1)$, we say that a vale $\hat{p}$
is an $(\epsilon, \delta)$-\textbf{approximation} of $p^*$, if
$\Pr\left[\,|\hat{p} - p^*|\geq \epsilon p^*\right] \leq \delta$. 
Karp et al. \cite{karp:jc85} present the following result
which will be used later in this paper.

\begin{theorem}\label{thm:num_dnf_prob_approx}
There is an $(\epsilon, \delta)$-approximation algorithm to~ \dnfsat{} 
with a running time of\\ $O((1/\epsilon^2)nm^2\log{(1/\delta}))$,
where $n$ and $m$ denote respectively the number of variables
and product terms in the DNF formula. \QED
\end{theorem}

\subsection{A Preliminary Result} 
\label{sse:prelim}

Here, we state and prove a preliminary result which will be used
in the subsequent sections of this paper.

\begin{proposition}\label{pro:sir_fixed_point}
(a) For any SIR system, every pseudo fixed point is a true fixed point.~
(b) Every SIR system with $n$ nodes reaches a fixed point 
in at most $n$ time steps.
\end{proposition}

\noindent
\textbf{Proof of Part (a):}~ Consider any configuration \calc{} of an SIR
system.
If the state of some node $v$ in \calc{} is \istate, then \calc{}
cannot be a pseudo or true fixed point since the state of $v$ will change 
to \rstate{} in the next time step.
Thus, if \calc{} is a pseudo or true fixed point, the state of each
node in \calc{} must be either \sstate{} or \rstate.
In such a configuration, no node can change state; 
that is, \calc{} is a true fixed point.

\smallskip
\noindent
\textbf{Proof of Part (b):}~
From the proof of Part~(a), 
it can be seen that whenever a configuration $\calc_t$ at time $t$ 
transitions into 
a different configuration $\calc_{t+1}$ at time $t+1$, the state 
of at least one node changes from \istate{} to \rstate. 
Since the number of nodes in the system is $n$ and nodes 
reaching state \rstate{} stay in that state forever, 
the system reaches a configuration with no node in state \istate{} 
after at most $n$ transitions. 
Such a configuration is a fixed point. \QED

\subsection{Summary of Contributions}
\label{sse:results_summary}

We now summarize the main contributions of this paper.
\begin{enumerate}
\item For general graphs, we establish computational intractability results for 
the four problems, namely \tNewInfs,~ \tTotInfs,~ \tVuls{}~ and~ \tTotVuls,
even when the time horizon $t$ is as small as 2.
We also prove a result that provides an indication of the difficulty
of obtaining approximate solutions to any of these problems 
for any $t \geq 2$. 
In addition, we observe that our hardness results are tight by showing that the
four problems are efficiently solvable for $t = 1$.
Further, we show that the \tPeak{} problem is computationally
intractable even for $t = 1$. 
We present a randomized approximation scheme for the problem
\tTotVuls, for any fixed $t$ and any set $S$ of fixed size.
We also show that the expected number of new infections at $t = 1$ and
$t = 2$ can  be computed efficiently. 
Table~\ref{tab:gen_results} provides statements of
our results for general graphs.

\item We extend the above intractability results to more
realistic social networks (e.g. networks with low diameter and
clustering coefficient and power-law networks) and to networks
in which all the edges have the same transmission probability.
Statements of these results appear in Table~\ref{tab:realistic_results}.

\item We further extend our results to prove the intractability 
of forecasting various epidemic measures introduced in \cite{TC+2016}.
These results are summarized in Table~\ref{tab:new_measures_results} 
(Section~\ref{sec:extensions}).

\item We also extend our intractability results to
three other epidemic models, namely SI, SIS and
probabilistic threshold.
These results are summarized in Table~\ref{tab:results_for_models} 
(Section~\ref{sec:extensions}).
\end{enumerate}
In a companion paper \cite{Rosenkrantz_etal_2016}, we have shown that
many of the forecasting problems under the SIR model can be solved efficiently
when the treewidth \cite{Bod93} of the underlying graph is bounded. 

\medskip

\begin{table}[tbh]
%%\begin{table}
\begin{center}
\begin{tabular}{|p{1.6in}|p{4.15in}|} \hline
\multicolumn{1}{|c|}{\textbf{Problem(s)}} & 
\multicolumn{1}{|c|}{\textbf{Result(s)}} \\ \hline\hline   
{\tNewInfs,\newline
\tTotInfs,\newline 
\tVuls{}~and \newline
\tTotVuls{}}
              & {(a) \cnump-hard for any~ $t \geq 2$ ~(Part~(1) of 
                     Theorem~\ref{thm:gen_hardness}). 

                \smallskip
                (b) For any $t \geq 2$, there is an $\epsilon > 0$ such
                     that unless \textbf{P} = \cnp, the quantity 
                     $\log{(2^n\,p^*)}$ cannot be efficiently approximated to
                     within the factor $n^{\epsilon}$, where $p^*$ is the solution
                     value and $n$ is the maximum number of nodes that can
                     get infected at $t = 1$. (Part~(2) of \newline  
                     Theorem~\ref{thm:gen_hardness}). %\newline
               } \\ \hline
{\tPeak{}}  & {\cnump-hard for  any $t \geq 1$~ (Part~(3) of
               Theorem~\ref{thm:gen_hardness}).
              } \\ \hline\hline
{\tNewInfs,\newline
 \tTotInfs, \newline
\tVuls{}~and \newline
\tTotVuls{}}
            & {Efficiently solvable for~ $t = 1$~
                     (Part~(1) of Corollary~\ref{cor:gen_easy_time_1}).} \\ \hline
{Compute the expected number of new infections at time $t$}              
            & {Efficiently solvable for~ $t = 1$~ and~ $t = 2$~
                     (Corollary~\ref{cor:expected_number_infected_time_2}).} \\ \hline
{{\tTotVuls{}} and \newline
{\tTotInfs{}}}  & {Randomized approximation scheme for 
                 any fixed $t$ and any set $S$ of fixed size~ 
                 (Section~\ref{sec:poly_versions}).
                 %(Theorems \ref{thm:tvuldnf} and \ref{thm:totinfdnf}).
              } \\ \hline\hline
\end{tabular}
\end{center}
\caption{Results for forecasting problems over general graphs\bigskip} 
\label{tab:gen_results}
\end{table}

\begin{table}[tbh]
%%\begin{table}
\begin{center}
\begin{tabular}{|p{1.6in}|p{3.9in}|} \hline
\multicolumn{1}{|c|}{\textbf{Problem}} & 
\multicolumn{1}{|c|}{\textbf{Result(s)}} \\ \hline\hline   
{\TwoNewInfv{}} & {\cnump-hard even when both the diameter 
                      and the average clustering coefficient are 1
                       (Part~(1) of Theorem~\ref{thm:dia_cc}).
                 } \\ \hline
{\TwoTotInfv{}} & {\cnump-hard even when both the diameter 
                       and the average clustering coefficient are 1
                       (Part~(2) of Theorem~\ref{thm:dia_cc}).
                 } \\ \hline
{\tNewInfv{}}   & {(a)  \cnump-hard for any $t \geq 2$ 
                      even when all edge probability values \newline
                      are less than 1 ~(Part~(1) of 
                      Theorem~\ref{thm:gen_hardness_prob_not_one}). \newline
                  (b)  \cnump-hard for any $t \geq 3$ even when 
                       all edge probability values \newline
                       are equal to $0.5$~
                       (Theorem~\ref{thm:gen_hardness_prob_half}). \newline
                  (c)  \cnump-hard for any $t \geq 2$ even 
                       for power-law graphs \newline 
                       (Theorem~\ref{thm:hardness_power_law}).
                 } \\ \hline
{\tTotInfv{}}   & {(a)  \cnump-hard for any $t \geq 2$ 
                       even when all edge probability values \newline
                       are less than 1 ~(Part~(2) of
                       Theorem~\ref{thm:gen_hardness_prob_not_one}). \newline
                  (b) \cnump-hard for any $t \geq 3$ even when 
                      all edge probability values \newline 
                      are equal to $0.5$~
                      (Theorem~\ref{thm:gen_hardness_prob_half}).
                 } \\ \hline\hline
\end{tabular}
\end{center}
\caption{Extensions of intractability results to more realistic networks}
\label{tab:realistic_results}
\end{table}


\subsection{Related Work}
\label{sse:related_work}

Several researchers have studied the dynamics of various diseases
under the SIR model 
(see e.g. \cite{marathe:cacm13,Britton-2009,BB+2009,Brauer-2008,
Salathe-Jones-2010,Ferrari-etal-2006} and the references cited therein).
Only a limited amount of work has been reported on
the computational complexity of problems that arise 
in the context of SIR disease dynamics. 
To our knowledge, we are the first to address complexity issues concerning
the short-term forecasting problems under the SIR model.
In contrast to our work (where the focus
is on short-term events), all of the known complexity results are for the case
when the epidemic has run its course and the network has reached a fixed point. 
We now summarize the known results.
When $t$ is not fixed,
Shapiro and Delgado-Eckert \cite{SD-2012} point out that the 
\cnp-hardness of the \tVuls{} problem follows from its direct relationship to
the two-terminal reliability problem for networks \cite{Ball-1980,Colbourn-1987},
even when the set $S$ consists of just a single node.
Several groups of researchers \cite{SD-2012,PS-2012,LZ-2009} 
have observed that computing the expected number of infections 
under the SIR model is computationally intractable.
Laumanns and Zenklusen \cite{LZ-2009} present two additional results.
They show that there is an efficient randomized approximation scheme 
for the problem of computing the expected number of infections.
They also show that even obtaining a constant factor approximation for the 
probability of a large number of infections (i.e., an infection size 
$\geq \alpha n$, where $\alpha < 1$ is a constant and $n$ is the number
of nodes in the graph) is computationally intractable.
Peyrard and Sabbadin \cite{PS-2012} consider a version where there is a
nonnegative cost associated with each node and the goal is to compute
the expected infection cost. 
They give a divide-and-conquer algorithm for the problem and compare its
performance experimentally with an approximation algorithm based 
on Monte Carlo simulations.

\medskip
For the sake of completeness, we briefly discuss papers that 
address forecasting problems in other contexts.
Empirical methods to forecast civil unrests
and other population-related events based on social media data are discussed in 
\cite{Korkmaz_etal_2016,Muthaiah_etal_2016}.
Cheng et al. \cite{Cheng_etal_2014} study the problem
of predicting the sizes of cascades in social networks such as Facebook and Twitter.
Martin et al. \cite{Martin_etal_2016} attribute the difficulty 
of predicting behaviors of complex social networks two factors, namely
the insufficiency of data or models and the unpredictability of
complex social systems.
These observations are consistent with those made by 
Drake \cite{Drake-2005,Drake-2006} in the context of 
epidemiological models.
Krishnan et al. \cite{Krishnan_etal_2016} consider information
propagation in networks where initially two or more nodes in the
network  have the information.
Using a large Twitter data set, 
they present a comprehensive set of experimental results
for predicting the size and shape of the forest created
by the diffusion process.

